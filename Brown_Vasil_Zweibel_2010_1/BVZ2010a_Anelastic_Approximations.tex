\documentclass[12pt]{article} % document type and language

\usepackage{amsmath, bm, mathtools, cancel, empheq, ulem}
%\usepackage{newtxmath} 
\usepackage[margin=1in]{geometry}

\date{April 03, 2019}
\author{Loren Matilsky}
\title{Brown, Vasil, and Zweibel (2010a): Anelastic Approximations}
\newcommand{\pderiv}[2]{\frac{\partial#1}{\partial#2}}
\newcommand\numberthis{\addtocounter{equation}{1}\tag{\theequation}}
\newcommand{\pomega}{\varpi}
\newcommand{\ugrad}{\bm{u}\cdot\nabla}
\allowdisplaybreaks
\begin{document}
	\maketitle
	\section{Basic equations}
	The Euler continuity equation (fully compressible) is
	\begin{subequations}\label{eq:cont}
		\begin{align}
		\pderiv{\rho}{t}&=-\nabla\cdot(\rho\bm{u})\label{eq:conta}\\
		&=-\rho\bm\nabla\cdot\bm{u} -\bm{u}\cdot\nabla\rho\Longrightarrow\nonumber\\
		\pderiv{\rho}{t} + \bm{u}\cdot\nabla\rho &= -\rho\nabla\cdot\bm{u},\label{eq:contb}\\
		\text{or}\ \ \ \ \ \frac{D\ln{\rho}}{Dt} &= -\nabla\cdot\bm{u},\label{eq:contc}
		\end{align}
	\end{subequations}
	where $D/Dt \coloneqq \partial/\partial t + \bm{u}\cdot\nabla$ is the convective (or substantive) derivative. 
	
	The Euler momentum equation is
	\begin{align}
	\pderiv{\bm{u}}{t} + \bm{u}\cdot\nabla\bm{u} = \frac{D\bm{u}}{Dt} = 
	 -\frac{1}{\rho}\big{[}\nabla P - \rho\bm{g}\big{]},
	 \label{eq:fc_mom}
	\end{align}
	and for an ideal gas,
	\begin{subequations}
	\begin{align}
	P &= \mathcal{R}\rho T = (\gamma - 1)\rho\mathcal{E},\label{eq:idgas1}\\
	\text{or}\ \ \ \ \ \ln{P} &= \ln{\rho} + \ln{T} + \text{const.}\label{eq:idgas2}
	\end{align}
	\end{subequations}
	with $\mathcal{E}$ the specific internal energy and $\gamma=c_p/c_v$ the ratio of specific heats. The latter equality comes from the fact that in an ideal gas,
	\begin{align}
	\gamma = 1 + \frac{2}{f}\nonumber,
	\end{align}
	where $f$ is the number of degrees of freedom (\text{d.o.f.}) in the gas ($f=3$ for a gas with three translational \text{d.o.f.} only). In that case $\mathcal{E}=(f/2)\mathcal{R} T=\mathcal{R} T/(\gamma - 1)$, leading to the second equality in \eqref{eq:idgas1}.
	
	\textit{Starting with the assumption} that the motions are adiabatic and the gas is ideal, we have $D(T/\rho^{\gamma-1})/Dt = 0\Longrightarrow D\big{[}\ln(T/\rho^{\gamma-1})\big{]}/Dt=0$, or $D\ln T/Dt - (\gamma-1) D\ln\rho/Dt=0$. Combined with \eqref{eq:contc}, this yields
	\begin{align}
	\frac{D\ln T}{Dt} = (\gamma - 1)\frac{D\ln\rho}{Dt} = -(\gamma - 1)\nabla\cdot\bm{u}.
			\label{eq:fc_en_T}
	\end{align}
	Similarly, since for adiabatic motions $D(P/\rho^\gamma)/Dt = 0$, we also have
	\begin{align}
		\frac{D\ln P}{Dt} = \gamma\frac{D\ln\rho}{Dt} = -\gamma\nabla\cdot\bm{u}.
		\label{eq:fc_en_P}
	\end{align}
	The fundamental thermodynamic relation is
	\begin{align*}
	TdS = d\mathcal{E} + PdV.
	\end{align*}
	With $V\coloneqq1/\rho$, we have $dV = -d\rho/\rho^2$, and since $\mathcal{E}=c_vT$ for an ideal gas,
	\begin{align*}
	TdS = c_vdT-\frac{P}{\rho^2}d\rho
	\end{align*}
	Or, dividing through by $c_pT$,
	\begin{subequations}\label{eq:ftr}
	\begin{align*}
	\frac{dS}{c_p}&= \frac{c_v}{c_p}\frac{dT}{T} - \frac{P}{c_p\rho T}d\ln\rho\\
	&= \frac{1}{\gamma}d\ln T - \underbrace{\frac{\mathcal{R}}{c_p}}_{\frac{c_p-c_v}{c_p}=\frac{\gamma - 1}{\gamma}}d\ln\rho\\
	&\boxed{= \frac{1}{\gamma}d\ln T - \frac{\gamma-1}{\gamma}d\ln\rho}\label{eq:ftra}\numberthis\\
	&= \frac{1}{\gamma}\underbrace{(d\ln P - d\ln\rho)}_{\text{by \eqref{eq:idgas2}}} - \frac{\gamma - 1}{\gamma}d\ln\rho\\
	&= \frac{1}{\gamma}(d\ln P - \cancel{d\ln\rho}-\gamma d\ln\rho + \cancel{d\ln\rho})\\
	&\boxed{= \frac{1}{\gamma}d\ln P - d\ln\rho}\label{eq:ftrb}\numberthis
	\end{align*}
	\end{subequations}
	Thus, by ``going backwards," we can (of course) re-derive the adiabatic assumptions from either \eqref{eq:fc_en_T} or \eqref{eq:fc_en_P}. For example, using \eqref{eq:fc_en_T},
	\begin{align}
	\frac{1}{c_p}\frac{DS}{Dt} &= \frac{1}{\gamma}\frac{D\ln T}{Dt} - \frac{\gamma-1}{\gamma}\frac{D\ln\rho}{Dt}\nonumber\\
	&= \frac{1}{\gamma}\big{[}-(\gamma-1)\nabla\cdot\bm{u}\big{]}+ \bigg{(}\frac{\gamma-1}{\gamma}\bigg{)}\nabla\cdot\bm{u} = 0
	\label{eq:fc_en_S}
	\end{align}
	
	The \textit{four} equations \eqref{eq:cont}--\eqref{eq:fc_en_T} form a complete system. There is a one-to-one correspondence between  \eqref{eq:fc_en_T},  \eqref{eq:fc_en_P}, and \eqref{eq:fc_en_S} through the fundamental thermodynamic relation, assuming an ideal gas. \textit{Each is simply a statement that the motions are adiabatic}.
	
	\section{Different forms for the buoyancy term}
	We specialize to a hydrostatic atmosphere,
	\begin{align}
	\nabla P_0 = \rho_0\bm{g},
	\label{eq:hydr}
	\end{align}
	assume the anelastic approximation,
	\begin{align}\label{eq:ans_cont}
	\nabla\cdot(\rho_0\bm{u})=0,
	\end{align}
	and assume small thermodynamic fluctuations, e.g.,
	\begin{align}
	\rho_1\coloneqq\rho - \rho_0 \ll \rho_0.
	\end{align}
	The equation of state (EOS) thus becomes linear in the thermodynamic perturbations, i.e., the ideal gas law \eqref{eq:idgas1} leads to 
	\begin{align*}
	(P_0 + P_1) &= \mathcal{R}(\rho_0 + \rho_1)(T_0+T_1)\Longrightarrow\\
	\{\cancel{P_0} + P_1 &= \mathcal{R}\big{[}\bcancel{\rho_0T_0}+ \rho_0T_1 + T_0\rho_1\big{]}\}\\
	-\{\cancel{P_0} &= \bcancel{\mathcal{R}\rho_0T_0}\}\\
	\text{or}\ \ \ \ \ P_1 &= \mathcal{R}\big{[}\rho_0T_1 +T_0 \rho_1\big{]}\\
	\text{or}\ \ \ \ \ \frac{P_1}{P_0} &= \frac{\mathcal{R}\rho_0T_1}{\mathcal{R}\rho_0T_0} + \frac{\mathcal{R}T_0\rho_1}{\mathcal{R}\rho_0T_0} \Longrightarrow\\ \frac{P_1}{P_0}&= \frac{\rho_1}{\rho_0} + \frac{T_1}{T_0},
	\end{align*} where we have assumed that the reference state also satisfies the ideal gas law independently. 
	
	Furthermore, the fundamental thermodynamic relation applies with the fluctuating quantities taking the place of differentials and reference state values taking the place of the thermodynamic variables themselves, i.e., 
	\begin{align*}
	\frac{dS}{c_p}=\frac{1}{\gamma}\frac{dP}{P} - \frac{d\rho}{\rho}\Longrightarrow\frac{S_1}{c_p}=\frac{1}{\gamma}\frac{P_1}{P_0}-\frac{\rho_1}{\rho_0}
	\end{align*}
	The preceding results can be combined in the form
	\begin{align}
	\frac{\rho_1}{\rho_0} = \frac{P_1}{P_0} - \frac{T_1}{T_0} = \frac{P_1}{\gamma P_0} - \frac{S_1}{c_p}.
	\label{eq:eos}
	\end{align}
	We now use \eqref{eq:hydr} and \eqref{eq:eos} to write the buoyancy acceleration term in \eqref{eq:fc_mom} as 
	\begin{align*}
	\mathbf{a}_{\rm{buoy}} \coloneqq\frac{-\nabla P +\rho\bm{g}}{\rho} &\approx 
	\frac{-\nabla P +\rho\bm{g}}{\rho_0}\\
	&= \frac{-\cancel{\nabla P_0} - \nabla P_1 + \cancel{\rho_0\bm{g}} + \rho_1\bm{g}}{\rho_0}\\
	&= -\frac{1}{\rho_0}\nabla P_1 + \frac{\rho_1}{\rho_0}\bm{g}\\
	&=  -\frac{1}{\rho_0}\nabla P_1 + \bigg{[}\frac{P_1}{\gamma P_0} - \frac{S_1}{c_p}\bigg{]} \bm{g}\\
	&= -\frac{1}{\rho_0}\nabla P_1 - \frac{S_1}{c_p}\bm{g} + \frac{P_1}{\gamma P_0}\bm{g}.
	\end{align*}
	We note that the fundamental thermodynamic relation \eqref{eq:ftr} can be integrated (assuming constant specific heats) from some initial state $(S_i, P_i, \rho_i)$ to the reference state, yielding
	\begin{align*}
	\frac{S_0 - S_i}{c_p} = \frac{1}{\gamma}(\ln{P_0} - \ln{P_i}) - (\ln\rho_0 - \ln\rho_i)
	&= \frac{1}{\gamma}(\ln{T_0} - \ln{T_i}) - \frac{\gamma - 1}{\gamma}(\ln\rho_0 - \ln\rho_i)\\
	\text{or}\ \ \ \ \ \frac{S_0}{c_p} &= \ln\big{(}P_0^{1/\gamma}\big{)}- \ln\rho_0 + \text{const.}\\
	&= \ln\big{(}T_0^{1/\gamma}\big{)}- \ln\big{[}\rho_0^{(\gamma - 1)/\gamma}\big{]} + \text{const.}\Longrightarrow\\
	\nabla\bigg{(}\frac{S_0}{c_p}\bigg{)} &= \frac{1}{\gamma}\nabla\ln P_0 - \nabla\ln\rho_0\\
	&= \frac{1}{\gamma}\nabla\ln T_0  - \frac{\gamma-1}{\gamma}\nabla\ln\rho_0.
	\tag{$\star$}
	\label{eq:tdr_ref}
	\end{align*}
	Using the preceding relations \eqref{eq:tdr_ref} and the expression for hydrostatic balance \eqref{eq:hydr} we can write the pressure term in buoyancy acceleration as
	\begin{align}
	\frac{P_1}{\gamma P_0}\bm{g} = \frac{P_1}{\rho_0}\frac{\rho_0\bm{g}}{\gamma P_0} \underbrace{=}_{\eqref{eq:hydr}} \frac{P_1}{\rho_0}\frac{\nabla P_0}{\gamma P_0} = \frac{P_1}{\rho_0} \frac{\nabla\ln P_0}{\gamma} \underbrace{=}_{\eqref{eq:tdr_ref}} \frac{P_1}{\rho_0}\Bigg{[}\nabla\bigg{(}\frac{S_0}{c_p}\bigg{)} + \nabla\ln\rho_0\Bigg{]}.
	\end{align}
	Thus, 
	\begin{align*}
	\mathbf{a}_{\rm{buoy}} &= -\frac{1}{\rho_0}\nabla P_1 - \frac{S_1}{c_p}\bm{g} + \frac{P_1}{\rho_0} \Bigg{[}\nabla\bigg{(}\frac{S_0}{c_p}\bigg{)} + \nabla\ln\rho_0\Bigg{]} \\
	&= \underbrace{-\frac{1}{\rho_0}\nabla P_1 + \frac{P_1}{\rho_0^2}\nabla\rho_0}_{-\nabla\big{(}\frac{P_1}{\rho_0}\big{)}} + \frac{P_1}{\rho_0}\nabla\bigg{(}\frac{S_0}{c_p}\bigg{)}-\frac{S_1}{c_p}\bm{g}\\
	&= -\nabla\pomega+\pomega\nabla\bigg{(}\frac{S_0}{c_p}\bigg{)} - \frac{S_1}{c_p}\bm{g}\tag{$\star\star$},
	\label{eq:abuoy1}
	\end{align*}
	where we have defined the ``reduced pressure" (``pomega")
	\begin{align}
	\pomega\coloneqq\frac{P_1}{\rho_0}.
	\end{align}
	Similarly, we can write the buoyancy acceleration in terms of the \textit{temperature fluctuation} (using the equation of hydrostatic balance \eqref{eq:hydr} and the linearized EOS \eqref{eq:eos}):
	\begin{align*}
	\mathbf{a}_{\rm{buoy}} &= -\frac{1}{\rho_0}\nabla P_1 +\bigg{[}\frac{P_1}{P_0}-\frac{T_1}{T_0}\bigg{]}\bm{g}\\
	&= \underbrace{-\frac{1}{\rho_0}\nabla P_1 + \frac{P_1}{P_0}\bigg{(}\frac{1}{\rho_0}\nabla P_0\bigg{)}}_{\substack{
		-\big{[}\frac{1}{\rho_0}\nabla P_1 - \frac{P_1}{\rho_0}\underbrace{\nabla\ln P_0}_{\nabla\ln T_0 + \nabla\ln\rho_0}\big{]} \\
		= -\frac{1}{\rho_0}\nabla P_1+ \frac{P_1}{\rho_0}\big{[}\nabla\ln T_0 + \frac{\nabla \rho_0}{\rho_0} \big{]}} } - \frac{T_1}{T_0}\bm{g}\\
	&= -\frac{1}{\rho_0}\nabla P_1 + \frac{P_1}{\rho_0^2}\nabla\rho_0 + \frac{P_1}{\rho_0}\nabla \ln T_0- \frac{T_1}{T_0}\bm{g}\\
	&= -\nabla\bigg{(}\frac{P_1}{\rho_0}\bigg{)}+\frac{P_1}{\rho_0}\nabla\ln T_0 - \frac{T_1}{T_0}\bm{g}\\
	& = -\nabla\pomega +\pomega\nabla\ln T_0 - \frac{T_1}{T_0}\bm{g}\tag{$\star\star\star$}
	\label{eq:abuoy2}
	\end{align*}
	
	\section{Anelastic Navier-Stokes equations}
	Assuming a time-independent and motionless reference state (i.e., $\partial\rho_0/\partial t=0$ and $\bm{u}_0\equiv 0$), the fully compressible continuity equation \eqref{eq:contb} becomes
	\begin{align}
	\pderiv{\rho_1}{t} + \cancelto{O(\epsilon^2)?}{\ugrad\rho_1}+ \ugrad\rho_0 = -\cancelto{O(\epsilon^2)?}{\rho_1\nabla\cdot\bm{u}} -\rho_0\nabla\cdot\bm{u}.
	\label{eq:fc_cont_lintd}
	\end{align}
	BVZ seem to get rid of the terms which on the LHS of \eqref{eq:fc_cont_lintd} would correspond to $\nabla\cdot(\rho_1\bm{u})$; however, this seems maybe dangerous (even with linearized thermodynamics), especially since eventually we will be taking the limit $\nabla\cdot(\rho_1\bm{u})=0$. Furthermore, I don't think the terms removed from \eqref{eq:fc_cont_lintd} are actually higher -order, since $\bm{u}$ won't be small once the convection starts (are ``nonlinear in the velocities" as BVZ explicitly note). Throwing away the fluctuating terms also seems inconsistent with keeping the $\partial\rho_1/\partial t$ term on the LHS. But maybe I am just confused?
	
	Using \eqref{eq:abuoy1}, the fully compressible momentum equation \eqref{eq:fc_mom} with linearized thermodynamics becomes 
	\begin{align}
	\frac{D\bm{u}}{Dt} = -\nabla\pomega + \pomega\nabla\bigg{(}\frac{S_0}{c_p}\bigg{)} - \frac{S_1}{c_p}\bm{g},
	\label{eq:fc_mom_lintd}
	\end{align}
	and with ($\partial S_0/\partial t =0$), the energy equation \eqref{eq:fc_en_S} becomes
	\begin{align}
	\pderiv{S_1}{t} + \ugrad S_1 + \ugrad S_0 = 0
	\label{eq:fc_en_lintd}
	\end{align}
	
	As previously shown in deriving forms for $\mathbf{a}_{\rm{buoy}}$, the fully compressible momentum equation (with linearized thermodynamics) is
	\begin{align}
	\pderiv{\bm{u}}{t}+(\ugrad)\bm{u}=\frac{-\nabla P_1 + \rho_1\bm{g}}{\rho_0}
	\label{eq:ans_mom}
	\end{align}
	and using \eqref{eq:abuoy1} we again reproduce the fully compressible momentum equation \eqref{eq:fc_mom_lintd}, which is unchanged in anelastic Navier-Stokes (ANS):
		\begin{align}
	\frac{D\bm{u}}{Dt} = -\nabla\pomega + \pomega\nabla\bigg{(}\frac{S_0}{c_p}\bigg{)} - \frac{S_1}{c_p}\bm{g}.
	\label{eq:ans_mom2}
	\end{align}
	We can use an ``integrating factor" to write
	\begin{align*}
	e^{-(S_0/c_p)}\Bigg{[}-\nabla\pomega+\pomega\nabla\bigg{(}\frac{S_0}{c_p}\bigg{)}\Bigg{]}&= -\Bigg{[}e^{-(S_0/c_p)}\nabla\pomega - \pomega \underbrace{e^{-(S_0/c_p)}\nabla\bigg{(}\frac{S_0}{c_p}\bigg{)}}_{-\nabla\big{[} e^{-(S_0/c_p)}\big{]}}\Bigg{]}\\
	&= -\big{\{}e^{-(S_0/c_p)}\nabla\pomega + \pomega\nabla\big{[} e^{-(S_0/c_p)}\big{]}\big{\}}\\
	&= -\nabla\big{[}e^{-(S_0/c_p)}\pomega\big{]},
	\end{align*}
	and thus
	\begin{align*}
 -\nabla\pomega+\pomega\nabla\bigg{(}\frac{S}{c_p}\bigg{)}
 =  -e^{(S_0/c_p)}\nabla\big{[}e^{-(S_0/c_p)}\pomega\big{]}.
	\end{align*}
	Thus, \eqref{eq:ans_mom2} can be rewritten
	\begin{align}
	\frac{D\bm{u}}{Dt}=-e^{(S_0/c_p)}\nabla\big{[}e^{-(S_0/c_p)}\pomega\big{]}
	- \frac{S_1}{c_p}\bm{g}
	\label{eq:ans_mom3}
	\end{align}
	From the relations in \eqref{eq:tdr_ref}, we can define a potential temperature as
	\begin{align}
	\Theta_0\coloneqq e^{(S_0/c_p)} = \frac{P_0^{1/\gamma}}{\rho_0},
	\end{align}
	and from the linearized EOS,
	\begin{align*}
	\Theta_1 \coloneqq \Theta-\Theta_0 &= e^{(S_0+S_1)/c_p}-e^{(S_0/c_p)}\\
	&= e^{(S_0/c_p)}\big{[}e^{(S_1/c_p)}-1\big{]}\\
	&\approx \Theta_0[(1+S_1/c_p) - 1] = \Theta_0\bigg{(}\frac{S_1}{c_p}\bigg{)},
	\end{align*}
	or
	\begin{align}
	\frac{\Theta_1}{\Theta_0}=\frac{S_1}{c_p}.
	\end{align}
	We thus rewrite the ANS momentum equation \eqref{eq:ans_mom3}
	\begin{align}
	\frac{D\bm{u}}{Dt}=-\Theta_0\nabla\big{(}\pomega\Theta_0^{-1}\big{)}-\frac{\Theta_1}{\Theta_0}\bm{g}.
	\end{align}
	We also have the identical energy equation as in FC (Equation \eqref{eq:fc_en_lintd}):
	\begin{align}
	\pderiv{S_1}{t} + \ugrad S_1  = - \ugrad S_0.
	\end{align}
	\textbf{Note:} The ANS equations differ from the FC equations \textit{only} through the continuity equation \eqref{eq:ans_cont}, i.e., $\nabla\cdot(\rho_0\bm{u})=0$ in ANS. In both sets of equations (at least in this work by BVZ), the thermodynamics are linearized. It thus seems that the assumption of anelasticity \eqref{eq:ans_cont} is separate from the assumption of small thermodynamic perturbations \eqref{eq:eos}. 
	
	\section{Lantz-Braginsky-Roberts (LBR) equations}
	Dropping the entropy gradient term in the ANS momentum equation (as is justifiable for a nearly adiabatic reference state) \eqref{eq:ans_mom2} leads to the LBR momentum equation
	\begin{align}
		\frac{D\bm{u}}{Dt} = -\nabla\pomega - \frac{S_1}{c_p}\bm{g},
		\label{eq:lbr_mom}
	\end{align}
	or
	\begin{align}
	\frac{D\bm{u}}{Dt} = -\nabla\pomega - \frac{\Theta_1}{\Theta_0}. 
		\label{eq:lbr_mom2}	
	\end{align}
	
	\section{Rogers-Glatzmaier (RG) equations (continued July 16, 2020)}
	Using the form of the buoyancy term in \eqref{eq:abuoy2}, we can write the ANS momentum equation \eqref{eq:ans_mom2} in the alternate form
	\begin{align}
	\frac{D\bm{u}}{Dt} = -\nabla\pomega +\pomega\nabla\ln T_0 - \frac{T_1}{T_0}\bm{g},
	\label{eq:rg_mom}
	\end{align} 
	which is called the Rogers-Glatzmaier (RG) anelastic momentum equation. \textit{\textbf{Note:} The RG momentum equation \eqref{eq:rg_mom} is identical mathematically to the ANS equation \eqref{eq:ans_mom2}}. Also, I am now continuing this document over a year after starting it, on July 16, 2020!
	
	The temperature-based statement of adiabaticity \eqref{eq:fc_en_T}, may be rewritten (dropping only the term $\partial T_0/\partial t$ and writing $T=T_0+T_1$ as usual)
	\begin{align}\label{eq:rg_en}
	\pderiv{T_1}{t} + \bm{u}\cdot \nabla T_1 = -\bm{u}\cdot\nabla T_0 - (\gamma - 1)(T_0+T_1)\nabla\cdot\bm{u},
	\end{align}
	the energy equation in RG. 
	
	Also, since $-\nabla\pomega +\pomega\nabla\ln T_0 = -T_0[(1/T_0)\nabla\pomega - (1/T_0^2)\pomega\nabla T_0] = -T_0\nabla(\pomega/T_0)$, we can rewrite the RG momentum equation \eqref{eq:rg_mom} as
	\begin{align}
	\frac{D\bm{u}}{Dt} = -T_0\nabla\bigg{(}\frac{\pomega}{T_0}\bigg{)} - \frac{T_1}{T_0}\bm{g}.\label{eq:rg_mom2}
	\end{align}
	
	We can manipulate the RG energy equation further to cast things in terms of the entropy gradient. Dividing equation \eqref{eq:rg_en} by $T_0$ and using the anelastic continuity equation \eqref{eq:ans_cont} in the form $\nabla\cdot\bm{u}=-\bm{u}\cdot\nabla\ln \rho_0$,
	\begin{align*}
	\pderiv{}{t}\bigg{(}\frac{T_1}{T_0}\bigg{)}+\bm{u}\cdot\nabla\bigg{(}\frac{T_1}{T_0}\bigg{)} + \frac{T_1}{T_0^2}\bm{u}\cdot\nabla T_0 &= -\bm{u}\cdot\nabla\ln T_0 + (\gamma - 1)\bigg{(}1 + \frac{T_1}{T_0}\bigg{)}\bm{u}\cdot\nabla\ln \rho_0\\
	\text{or}\ \ \ \ \ \frac{D}{Dt}\bigg{(}\frac{T_1}{T_0}\bigg{)} &= -\bigg{(}1 + \frac{T_1}{T_0}\bigg{)}\bm{u}\cdot\nabla\ln T_0 + (\gamma - 1)\bigg{(}1 + \frac{T_1}{T_0}\bigg{)}\bm{u}\cdot\nabla\ln \rho_0\\
	&= - \gamma \bigg{(}1 + \frac{T_1}{T_0}\bigg{)}\bm{u}\cdot\bigg{(}\underbrace{\frac{1}{\gamma}\nabla\ln T_0 - \frac{\gamma - 1}{\gamma}\nabla\ln \rho_0}_{=\ \nabla(S_0/c_p)\ \text{by}\ \eqref{eq:tdr_ref}}\bigg{)}\\
	\text{or}\ \ \ \ \ \frac{D}{Dt}\bigg{(}\frac{T_1}{T_0}\bigg{)} &=  - \gamma \bigg{(}1 + \frac{T_1}{T_0}\bigg{)}\bm{u}\cdot\nabla\bigg{(}\frac{S_0}{c_p}\bigg{)}.\numberthis\label{eq:rg_en2}
	\end{align*}
	\textbf{Note:} formally, it is clear that the RG equations are mathematically equivalent to the ANS equations. Am I missing something?
\end{document}