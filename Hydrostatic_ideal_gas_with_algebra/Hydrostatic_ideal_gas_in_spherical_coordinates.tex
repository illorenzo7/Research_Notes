\documentclass[12pt]{article} % document type and language

\usepackage{amsmath, bm, mathtools, cancel, empheq, ulem, mathrsfs}
%\usepackage{newtxmath} 
\usepackage[margin=1in]{geometry}

\date{April 18, 2019}
\author{Loren Matilsky}
\title{Hydrostatic, ideal gas reference states in spherical coordinates}
\newcommand{\pderiv}[2]{\frac{\partial#1}{\partial#2}}
\newcommand{\bigfrac}[2]{\bigg{(}\frac{#1}{#2}\bigg{)}}
\newcommand{\mbigfrac}[2]{\bigg{(}{-\frac{#1}{#2}}\bigg{)}}
\newcommand\numberthis{\addtocounter{equation}{1}\tag{\theequation}}
\newcommand{\pomega}{\varpi}
\newcommand{\ugrad}{\bm{u}\cdot\nabla}
\newcommand{\cv}{c_{\rm{v}}}
\newcommand{\cp}{c_{\rm{p}}}
\newcommand{\orr}{\text{or}\ \ \ \ \ }
\newcommand{\andd}{\text{and}\ \ \ \ \ }
\newcommand{\tz}{\tilde{\zeta}}
\newcommand{\five}{\ \ \ \ \ }
\newcommand{\ei}{{\rm{Ei}}}
\newcommand{\tr}{\tilde{r}}
\newcommand{\ek}{e^{-k[(r/r_i)-1]}}
\newcommand{\ekt}{e^{-k[(\tr/r_i)-1]}}
\newcommand{\ekp}{e^{k[(r/r_i)-1]}}
\allowdisplaybreaks
\begin{document}
	\maketitle
	\section{Basic assumptions}
	We wish to derive the thermodynamic state for an ideal, hydrostatic gas in spherical coordinates, assuming a known specific entropy stratification $s(r)$. In mathematical terms, this means that
	\begin{align}
	\frac{dp}{dr} &= -\rho g(r)\label{eq:hydr},\\
	p(r) &= \rho(r)\mathcal{R} T(r),\label{eq:idgas}
	\end{align}
	and 
	\begin{align}
	s^\prime(r) \coloneqq \frac{ds}{dr}
	\label{def:sgrad}
	\end{align}
	is a known function of radius. Here $p(r)$ is the pressure, $\rho(r)$ the density, $T(r)$ the temperature, $g(r)$ the gravitational acceleration (unspecified for now except that it is spherically symmetric), and $\mathcal{R}$ the gas constant. We further assume an ideal gas with three translational degrees of freedom (d.o.f.) throughout, so that
	\begin{subequations}	\label{eq:heats}
	\begin{align}
	\cv &=\frac{3}{2}\mathcal{R},\\
	\cp &= \cv + \mathcal{R} =\frac{5}{2}\mathcal{R},\\
	\gamma &\coloneqq \frac{\cp}{\cv} = \frac{5}{3},\\
	\text{and}\ \ \ \ \ \mathcal{R} &= (\gamma-1)\cv,
	\end{align}
	\end{subequations}
	which in turn implies that 
	\begin{subequations}\label{eq:heats2}
	\begin{align}
	\cv &= \frac{\mathcal{R}}{\gamma - 1}\label{eq:cv_from_gamma}\\
	\andd \cp &= \frac{\gamma\mathcal{R}}{\gamma - 1}\label{eq:cp_from_gamma}.
	\end{align}
	\end{subequations}
	For quasistatic processes, the First Law of Thermodynamics states that
	\begin{align}
	Tds=de+pdv= \cv dT - \frac{p}{\rho^2}d\rho,
	\label{eq:firstlaw1}
	\end{align}
	where $e$ is the specific energy, $v=1/\rho$ is the specific volume, and the second equality follows from the fact that $e=\cv T$ for an ideal gas (we assume the variation of $\cv$ with $T$---at least in the infinitesimal sense---is negligible). Noting that $p/\cv \rho T = \rho\mathcal{R} T/\cv \rho T = \mathcal{R}/\cv=\gamma - 1$, we can divide \eqref{eq:firstlaw1} by $e=\cv T$ to yield
	\begin{align}
	\frac{ds}{\cv}=\frac{dT}{T} - (\gamma - 1)\frac{d\rho}{\rho} = d\ln{T} - (\gamma - 1)d\ln{\rho}. 
	\label{eq:firstlaw2}	
	\end{align}
	\textit{As long as} there are consistently only three d.o.f. throughout the domain of interest (i.e., $T$ is not so high that rotational and vibrational d.o.f. are excited, for example), then $\cv,\cp$, and $\gamma$ are constants and relation \eqref{eq:firstlaw2} can be applied across radius to yield
	\begin{align}
	\frac{d\ln{T}}{dr} - (\gamma - 1)\frac{d\ln{\rho}}{dr} = \frac{s^\prime}{\cv}.
	\label{eq:firstlaw3}
	\end{align}
	Taking the radial derivative of the logarithm of the ideal gas law \eqref{eq:idgas} also gives
	\begin{align}
	\frac{d\ln{p}}{dr} = \frac{d\ln{\rho}}{dr} + \frac{d\ln{T}}{dr}. 
	\label{eq:idgasdr}
	\end{align}
	To isolate the temperature derivative, we plug \eqref{eq:idgasdr} into \eqref{eq:firstlaw3} use the hydrostatic equation \eqref{eq:hydr} to give
	\begin{align*}
	\frac{d\ln{T}}{dr} - (\gamma - 1)\frac{d\ln{\rho}}{dr} &= \frac{d\ln{T}}{dr} - (\gamma -1)\bigg{(}\frac{d\ln{p}}{dr} - \frac{d\ln{T}}{dr}\bigg{)}\\
	 &=\gamma \frac{d\ln{T}}{dr} - (\gamma - 1)\frac{(-\rho g)}{p}\\
	 &= \gamma \frac{d\ln{T}}{dr} + \frac{\gamma - 1}{\mathcal{R}}\frac{g}{T}
	\end{align*}
	and thus
	\begin{align*}
	\frac{\gamma}{T}\frac{dT}{dr} + \frac{\gamma - 1}{\mathcal{R}}\frac{g}{T} &= \frac{s^\prime}{\cv},\\
	\orr \frac{dT}{dr} + \frac{\gamma - 1}{\gamma \mathcal{R}}g &= \frac{Ts^\prime}{\cp}.
	\end{align*}
	Using \eqref{eq:cp_from_gamma}, our equation for temperature in the standard form for ordinary differential equations is
	\begin{align}
	\boxed{
	\frac{dT}{dr} - \bigg{[}\frac{s^\prime(r)}{\cp}\bigg{]}T(r) = -\frac{g(r)}{\cp}.
}
\label{eq:tdiffeq}
	\end{align}
	The preceding equation is ``nice" and can be multiplied by the integrating factor $e^{-s/\cp}$ to yield
	\begin{align}
	\frac{d}{dr}(e^{-s/\cp}T)&=-\frac{ge^{-s/\cp}}{\cp},\nonumber\\
	\orr e^{-s(r)\cp}T(r) &=-\frac{1}{\cp}\int_{r_0}^r g(\tilde{r})e^{-s(\tilde{r})/\cp}d\tilde{r} + A,\nonumber\\
	\orr T(r) &= -\frac{e^{s(r)/\cp}}{\cp}\int_{r_0}^r g(\tilde{r})e^{-s(\tilde{r})/\cp}d\tilde{r} + Ae^{s(r)/\cp}.\nonumber
	\end{align}
	Here we have defined $r_0$ as an arbitrary radius from which to start the integration (in practice it will often be convenient to set $r_0=r_i$, the radius at the inner boundary of the shell). The integration constant $A$ depends on the temperature at the chosen radius, $T_0\coloneqq T(r_0)$. We can see that $T_0 = 0 + Ae^{s_0/\cp}$, or $A=T_ie^{-s_0/\cp}$, where $s_0$ is the entropy at $r_0$. Thus, the full expression for $T(r)$ is
	\begin{align}
	\boxed{
	T(r) = -\frac{e^{s(r)/\cp}}{\cp}\int_{r_0}^r g(\tilde{r})e^{-s(\tilde{r})/\cp}d\tilde{r} + T_0e^{[s(\tilde{r})-s_0]/\cp}.
	\label{eq:tgeneral}}
	\end{align}
	Note that in equation \eqref{eq:tgeneral}, adding a constant $\sigma$ to the entropy ($s\longrightarrow s+\sigma$) has no effect on the value of $T$. \textit{The absolute value of the entropy nowhere appears in the final state; only the relative stratification of entropy is important}. 
	
	Once $T(r)$ is obtained, we can obtain $p(r)$ through the hydrostatic assumption:
	\begin{align*}
	\frac{dp}{dr}=-\rho g = -\frac{p}{\mathcal{R}T}g, \ \ \ \ \ \orr \frac{d\ln{p}}{dr} = -\frac{1}{\mathcal{R}}\frac{g}{T}.
	\end{align*}
	Integrating, we find
	\begin{align*}
	\ln{p} - \ln{p_0} = \ln{(p/p_0)} = -\frac{1}{\mathcal{R}}\int_{r_0}^r\frac{g(\tilde{r})}{T(\tilde{r})}d\tilde{r},
	\end{align*}
	or
	\begin{align*}
	p(r) = p_0\exp{\bigg{[}-\frac{1}{\mathcal{R}}\int_{r_0}^r\frac{g(\tilde{r})}{T(\tilde{r})}d\tilde{r}\bigg{]}},
	\end{align*}
	where $p_0$ is the pressure at $r_0$. Using \eqref{eq:tdiffeq}, we can further simplify:
	\begin{align*}
	\int_{r_0}^r \frac{g(\tr)}{T(\tr)}d\tr &= \int_{r_0}^r\bigg{[}s^\prime(\tr) - \cp \frac{d\ln T}{d\tr}\bigg{]}d\tr = s(r) - s_0 - \cp \ln(T/T_0)\Longrightarrow \\
	p(r) &= p_0\exp\bigg{[}-\frac{s(r) - s_0}{\mathcal{R}} + \frac{\cp}{\mathcal{R}}\ln\bigg{(}\frac{T}{T_0}\bigg{)}\bigg{]},
	\end{align*}
	or
	\begin{align}
	\boxed{
	p(r)=p_0\exp\bigg{[}-\frac{s(r) - s_0}{\mathcal{R}}\bigg{]}\bigg{[}\frac{T(r)}{T_0}\bigg{]}^{\gamma/(\gamma - 1)}.
}
\label{eq:pgeneral}
	\end{align}
	Here $p_0$ is the pressure at $r_0$, and we have used \eqref{eq:cp_from_gamma} to write $\cp/\mathcal{R} = \gamma/(\gamma-1)$.
	
	Finally, $\rho(r)$ is now determined from the ideal gas law \eqref{eq:idgas}:
	\begin{align}
	\boxed{
	\rho(r) =\rho_0\exp\bigg{[}-\frac{s(r) - s_0}{\mathcal{R}}\bigg{]}\bigg{[}\frac{T(r)}{T_0}\bigg{]}^{1/(\gamma - 1)},
}
\label{eq:rhogeneral}
	\end{align}
	where $\rho_0=p_0/\mathcal{R}T_0$ is the density at $r_0$. 
	
	\textit{The full state of the atmosphere is purely determined by the entropy stratification $s^\prime(r)$, the gravitational acceleration $g(r)$, and the two constants $T_0$ and $p_0$. (We assume we already know the specific heats and gas constant).}
	
	The state specified in equations \eqref{eq:tgeneral}--\eqref{eq:rhogeneral} is completely general, relying only on the assumptions of a hydrostatic, ideal gas. To move forward, we must specify $s^\prime(r)$ and $g(r)$. For the convection zone of a star (and much of the radiative zone below), it is safe to assume that the gravitational acceleration comes purely from a central spherically distributed mass, i.e., 
	\begin{align}
	g(r)=\frac{GM}{r^2},
	\label{eq:gpointmass}
	\end{align}
	where $G$ is the Universal Gravitational Constant and $M$ is the stellar mass. 
	\section{Specific atmospheres}
	Assuming equations \eqref{eq:tgeneral}--\eqref{eq:gpointmass}, we now derive atmospheres for various cases of the entropy stratification $s^\prime(r)$. 
	\subsection{Adiabatic atmosphere}
	Considering the mathematical form of \eqref{eq:tgeneral}, the simplest atmosphere to derive is one that is adiabatic (constant-entropy):
	\begin{align}
	s=s_0\equiv\rm{const.} \ \ \ \ \ \andd s^\prime \equiv 0.
	\label{eq:sconst}
	\end{align}
	In this case all the exponentials in \eqref{eq:tgeneral} cancel to 1, and after choosing $r_0=r_i$ (with $T_i\coloneqq T(r_i)$, $p_i\coloneqq p(r_i)$, etc.), we find 
	\begin{align*}
	T(r) = -\frac{GM}{\cp}\int_{r_i}^r\frac{d\tilde{r}}{\tilde{r}^2} + T_i = T_i +  \frac{GM}{\cp}\frac{1}{\tilde{r}}\bigg{|}_{r_i}^r = T_i + \frac{GM}{\cp}\bigg{(}\frac{1}{r} - \frac{1}{r_i}\bigg{)},
	\end{align*}
	which can be rearranged to give 
	\begin{align}
	\boxed{
	T(r) = T_i\bigg{[}a_0\bigfrac{r_i}{r} + (1-a_0)\bigg{]}  \ \ \ \ \ \text{(adiabatic atmosphere)},
}
	\label{eq:tad}
	\end{align}
	where 
	\begin{align}
	a_0\coloneqq \frac{GM}{\cp T_i r_i}.
	\end{align}
	
	Then \eqref{eq:pgeneral} $\Longrightarrow$
	\begin{align}
	\boxed{
	p(r) = p_i \bigg{[}a_0\bigfrac{r_i}{r} + (1-a_0)\bigg{]}^{\gamma/(\gamma-1)} \ \ \ \ \ \text{(adiabatic atmosphere).}
}
\label{eq:pad}
	\end{align}
	Finally, plugging \eqref{eq:tad} and \eqref{eq:pad} into the ideal gas law $\rho=p/\mathcal{R} T$ yields
	\begin{align}
\boxed{
	\rho(r) = \rho_i \bigg{[}a_0\bigfrac{r_i}{r} + (1-a_0)\bigg{]}^{1/(\gamma-1)} \ \ \ \ \ \text{(adiabatic atmosphere),}
}
\label{eq:rhoad}
\end{align}
where $\rho_i = p_i/\mathcal{R}T_i$ is the density at the inner boundary.

\subsection{Polytropes}
The starting assumption for a ``polytrope" is that the temperature gradient is divergenceless---i.e., there is a constant flux due to radiative diffusion throughout the fluid layer. In 
spherical coordinates this condition becomes
\begin{align*}
\frac{\kappa_r}{r^2}\frac{d}{dr}\bigg{(}r^2\frac{dT}{dr}\bigg{)} = 0 \Longleftrightarrow r^2\frac{dT}{dr}=\text{const.} \Longleftrightarrow \frac{dT}{dr} = \frac{\text{const.}}{r^2},
\end{align*}
or
\begin{align}
T(r) = \frac{A}{r} + B.
\label{eq:tconstflux1}
\end{align}
Here $\kappa_r$ is the radiative diffusivity; the polytrope assumes the diffusivity \textit{cannot} vary with radius---an assumption that obviously breaks down in a real star with, say, a Kramer's opacity where $\kappa_r\sim \rho^\alpha T^\beta$ ($\alpha$ and $\beta$ are dimensionless constants). The integration constants $A$ and $B$ in \eqref{eq:tconstflux1} are arbitrary, but one is eliminated from the condition $T(r_i) = T_i$, leaving
\begin{align}
\boxed{
T(r) = T_i \bigg{[}a\bigg{(}\frac{r_i}{r}\bigg{)} + (1 - a)\bigg{]} \ \ \ \ \ \text{(polytropic atmosphere)},
}
\label{eq:tpolytrope1}
\end{align}
where the new constant $a\coloneqq A/T_ir_i$ is dimensionless. Note the similarity of the argument $[a(r_i/r) + (1-a)]$ in \eqref{eq:tpolytrope1} (which was derived by assuming a divergenceless temperature gradient) to the argument in \eqref{eq:tad} (which was derived by assuming the gas was adiabatic). 

We again use \eqref{eq:pgeneral} to find $p(r)$, yielding
\begin{align}
\boxed{
p(r) = p_i\bigg{[}a\bigg{(}\frac{r_i}{r}\bigg{)} + (1 - a)\bigg{]}^{n+1}\ \ \ \ \ \text{(polytropic atmosphere)},
}
\label{eq:ppolytrope1}
\end{align}
where we have defined the \textit{polytropic index} $n$ to be
\begin{align}
n\coloneqq \frac{1}{a}\frac{GM}{\mathcal{R}T_i r_i} - 1 \Longleftrightarrow a = \frac{GM}{(n+1)\mathcal{R}T_i r_i}.
\label{def:polytropicindex}
\end{align}
From the ideal gas law, the density is then
\begin{align}
\boxed{
\rho(r) = \rho_i\bigg{[}a\bigg{(}\frac{r_i}{r}\bigg{)} + (1 - a)\bigg{]}^n \ \ \ \ \ \text{(polytropic atmosphere)}. 
}
\label{eq:rhopolytrope1}
\end{align}

We can also easily compute the logarithmic derivatives of the reference variables by defining 
\begin{align}
\zeta(r) = a\bigg{(}\frac{r_i}{r}\bigg{)} + (1-a), \ \ \ \ \ \frac{d\zeta}{dr} = -\frac{ar_i}{r^2}
\end{align}
and computing
\begin{empheq}[box=\fbox]{align}
\frac{d\ln{T}}{dr} &= - \frac{ar_i}{r^2\zeta},\\
\frac{d\ln{p}}{dr} &=- (n+1)\frac{ar_i}{r^2\zeta},\ \ \ \ \  \ \ \ \ \  \text{(polytropic atmosphere)}\\
\frac{d\ln{\rho}}{dr} &= - n\frac{ar_i}{r^2\zeta},\\
\frac{d^2\ln{\rho}}{dr^2}  &= nar_i\bigg{[}\frac{2}{r^2\zeta} - \frac{ar_i}{r^4\zeta^2}\bigg{]}, \ \ \ \ \ \text{etc...}
\end{empheq}
The equations \eqref{eq:tad}--\eqref{eq:rhoad} for the adiabatic atmosphere are a special case of equations \eqref{eq:tpolytrope1}--\eqref{eq:rhopolytrope1} for the polytropic atmosphere, provided we define the polytropic constants for an adiabatic atmosphere as 
\begin{align}
n_0\coloneqq \frac{1}{\gamma-1}\ \ \ \ \ \orr a_0\coloneqq \frac{GM}{\cp T_ir_i}.
\label{def:n0}
\end{align}

Note that for the polytrope, in addition to the two constants $p_i$ and $T_i$, there is another constant $n$ (or equivalently $a$), which we now show is equivalent to specifying the entropy gradient. Plugging equations \eqref{eq:tpolytrope1}--\eqref{def:n0} into the First Law of Thermodynamics \eqref{eq:firstlaw3} yields
\begin{align}
\boxed{
s^\prime(r) = \bigg{(}\frac{n}{n_0} - 1\bigg{)}\bigg{[}\frac{\cv}{r+(1-a)r^2/(ar_i)}\bigg{]} \ \ \ \ \ \text{(polytropic atmosphere).}
}
\label{eq:sprimepolytrope}
\end{align}
From the preceding equation, it is obvious that $n=n_0$ corresponds to an adiabatic atmosphere (as we have already shown), while
\begin{align}
&n > n_0 \Longleftrightarrow s^\prime > 0 \text{ (atmosphere is stable to convection)}\\
\andd &n < n_0 \Longleftrightarrow s^\prime < 0 \text{ (atmosphere is unstable to convection).}
\end{align}

If the actual entropy is desired instead of the entropy gradient, \eqref{eq:sprimepolytrope} may be integrated via partial fractions to yield
\begin{align}
\boxed{
s(r) = \cv\bigg{(}\frac{n}{n_0} - 1\bigg{)}\bigg{\{}\ln{\bigg{(}\frac{r}{r_i}\bigg{)}} - \ln{\bigg{[}a+(1-a)\bigg{(}\frac{r}{r_i}\bigg{)}\bigg{]}}\bigg{\}}\ \ \ \ \ \text{(polytropic atmosphere)},
}
\label{eq:spolytrope}
\end{align}
where we have chosen $s_i=0$ for simplicity.

Assuming the entropy stratification \eqref{eq:spolytrope}, one could of course derive the polytropic equations \eqref{eq:tpolytrope1}--\eqref{eq:rhopolytrope1} via the general equations \eqref{eq:tgeneral}--\eqref{eq:rhogeneral}. But this would be a rather contrived form of the entropy profile a priori, and its only benefit would be to enforce a divergenceless temperature gradient, so it makes the most sense to start from the divergenceless temperature gradient (constant-radiative-flux) equation \eqref{eq:tconstflux1}. 

\subsubsection{Polytrope in terms of overall density stratification}
In the Rayleigh code and Jones et al. (2011), the polytrope is initiated using the inner density $\rho_i$ and the number of density scale heights $N_\rho$, as opposed to the two inner-boundary values $p_i$ and $T_i$. We can recast equations \eqref{eq:tpolytrope1}--\eqref{eq:rhopolytrope1} in terms of $\rho_i$ and $N_\rho$ by defining the density at the inner and outer boundaries as $\rho_i$ and $\rho_o$, noting the definition $e^{N_\rho}\coloneqq \rho_i/\rho_o$, and computing
\begin{align*}
e^{N_\rho/n} = (\rho_i/\rho_o)^{1/n} = \frac{1}{a(r_i/r_o) + (1-a)} &= \frac{1}{1 - (1 - \beta)a} \Longleftrightarrow e^{N_\rho/n} - (1-\beta)e^{N_\rho/n}a = 1\\
 \Longleftrightarrow  (1 - \beta)e^{N_\rho/n}a &= e^{N_\rho/n} - 1 \Longleftrightarrow a = \frac{e^{N_\rho/n} - 1 }{(1 - \beta)e^{N_\rho/n}},
\end{align*}
where we have defined the aspect ratio of the shell,
\begin{align}
\beta\coloneqq \frac{r_i}{r_o} < 1.
\label{def:beta}
\end{align}
Noting the definition of $a$ \eqref{def:polytropicindex}, we see that 
\begin{align}
\frac{GM}{(n+1)\mathcal{R}T_i r_i} =  \frac{e^{N_\rho/n} - 1 }{(1 - \beta)e^{N_\rho/n}} \ \ \ \ \ \orr T_i = \frac{(1 - \beta)e^{N_\rho/n}} {e^{N_\rho/n} - 1 }\frac{GM}{(n+1)\mathcal{R} r_i}.
\label{eq:ti_from_nrho}
\end{align}
Thus, instead of specifying $T_i$ and $p_i$, we can specify $\rho_i$ and $N_\rho$; $T_i$ is then determined through equation \eqref{eq:ti_from_nrho}, and $p_i=\rho_i\mathcal{R}T_i$. 

\subsubsection{Polytrope with respect to the center of the shell}
Jones et al. (2011) defines the polytrope using the variables at the center of the shell (i.e., in our notation $r_0=r_c\coloneqq(r_i+r_o)/2$. In that case, equations  \eqref{eq:tpolytrope1}--\eqref{eq:rhopolytrope1} become
\begin{align}
	T(r) &= T_c\bigg{[}a_c\bigg{(}\frac{r_c}{r}\bigg{)} + (1 - a_c)\bigg{]},\label{eq:tpolytrope2}\\
	p(r) &= p_c\bigg{[}a_c\bigg{(}\frac{r_c}{r}\bigg{)} + (1 - a_c)\bigg{]}^{n+1},\label{eq:ppolytrope2}\\
	\rho(r) &= \rho_c\bigg{[}a_c\bigg{(}\frac{r_c}{r}\bigg{)} + (1 - a_c)\bigg{]}^n,\label{eq:rhopolytrope2}
\end{align} 
where 
\begin{align}
a_c \coloneqq \frac{GM}{(n+1)\mathcal{R}T_c r_c}.
\label{def:ac}
\end{align}
Using a similar computation as the one leading to \eqref{eq:ti_from_nrho}, one can show that 
\begin{align}
a_c = \frac{2\beta(e^{N_\rho/n}-1)}{(1-\beta)(\beta e^{N_\rho/n} + 1)}.
\label{eq:ac_fromn}
\end{align}
Defining 
\begin{align}
d&\coloneqq r_o-r_i,\label{def:d}\\
c_1&\coloneqq \bigg{(}\frac{r_c}{d}\bigg{)}a_c = \frac{(1+\beta)\beta(e^{N_\rho/n}-1)}{(1-\beta)^2(\beta e^{N_\rho/n} + 1)},\label{def:c1}\\
c_0&\coloneqq 1-a_c = \frac{(1+\beta)(1-\beta e^{N_\rho/n})}{(1-\beta)(\beta e^{N_\rho/n} + 1)}, \label{def:c0}\\
\andd \tz(r) &\coloneqq a_c\bigg{(}\frac{r_c}{r}\bigg{)} + (1-a_c) = c_0 + c_1\bigg{(}\frac{d}{r}\bigg{)},\label{def:tildezeta}
\end{align}
equations \eqref{eq:tpolytrope2}--\eqref{eq:rhopolytrope2} become
\begin{align}
T(r)=T_c\tz(r),\ \ \ \ \ p(r)=p_c[\tz(r)]^{n+1},\ \ \ \ \ \rho(r)=\rho_c[\tz(r)]^n. 
\label{eq:polytrope3}
\end{align}
With some more algebra, we can also see that if we define
\begin{align}
\tz_o \coloneqq \tz(r_o) = \frac{1+\beta}{\beta e^{N_\rho/n} + 1}, 
\end{align}
then 
\begin{align}
c_0 = \frac{2\tz_o - \beta - 1}{1-\beta} \ \ \ \ \ \andd c_1 = \frac{(1+\beta)(1-\tz_o)}{(1-\beta)^2},
\label{eq:c01_from_zeta0}
\end{align}
yielding exactly the formulation for the polytrope in Jones et al. (2011). 

Both the formulation with respect to the center of the shell \eqref{def:tildezeta}--\eqref{eq:c01_from_zeta0} and the formulation with respect to the inner boundary \eqref{eq:tpolytrope1}--\eqref{eq:rhopolytrope1} (also \eqref{eq:ti_from_nrho} if specifying $\rho_i$ and $N_\rho$ instead of $p_i$ and $T_i$) are mathematically equivalent. However, given the substantial extra notation involved in defining the constants $c_0$ and $c_1$, it seems preferable to use the latter formulation. 

\subsection{Two polytropes stitched together}
In simulations involving a convection zone (CZ) on top of a non-convecting (stable) radiation zone (RZ), a common approximation is to use two polytropes: one with $n_1\approx n_0$ in the CZ (domain 1) and one with $n_2>n_0$ in the RZ (domain 2). We denote the inner and outer boundaries by $r_i$ and $r_o$ (as before) and the ``middle boundary," or interface between the convective layer and stable region, by $r_m$. If we take $r_0=r_m$ and specify $N_\rho$, $n_1$, and $\rho_m \coloneqq \rho(r_m)$, the CZ polytrope is given by
\begin{align}
T_1(r) &= T_m\zeta_1(r), \five p_1(r) = p_m[\zeta_1(r)]^{n_1+1}, \five \rho(r) = \rho_m[\zeta_1(r)]^{n_1},\label{eq:poly1}\\
 \text{where}\five \zeta_1(r)&\coloneqq a_1\bigg{(}\frac{r_m}{r}\bigg{)} + (1-a_1),\five a_1 \coloneqq \frac{e^{N_\rho/n_1}-1}{(1-\beta_1)e^{N_\rho/n_1}},\nonumber\\
 T_m &\coloneqq \frac{1-\beta_1}{e^{N_\rho/n_1}-1}\bigg{[}\frac{GM}{(n_1+1)\mathcal{R} r_m}\bigg{]},\five p_m\coloneqq \rho_m\mathcal{R}T_m, \five\andd \beta_1\coloneqq \frac{r_m}{r_o}.
\end{align}
The entropy profile is obtained from \eqref{eq:sprimepolytrope} and \eqref{eq:spolytrope}, yielding 
\begin{align}
s_1(r) &= \cv\bigg{(}\frac{n_1}{n_0} - 1\bigg{)}\bigg{\{}\ln{\bigg{(}\frac{r}{r_m}\bigg{)}} - \ln{\bigg{[}a_1+(1-a_1)\bigg{(}\frac{r}{r_m}\bigg{)}\bigg{]}}\bigg{\}},\nonumber\\
\andd  s_1^\prime(r) &= \bigg{(}\frac{n_1}{n_0} - 1\bigg{)}\bigg{[}\frac{\cv}{r+(1-a_1)r^2/(a_1r_m)}\bigg{]}.
\end{align}
Note that if $n_1=n_0$, $s^\prime(r)=s(r)\equiv 0$. 

The only free parameter to specify the RZ polytrope is $n_2$, since for a continuous double-polytrope, the values of the thermodynamic variables at the top of the RZ must be $\rho_m$, $T_m$, $p_m$, and $s_m=0$. We thus again set $r_o=r_m$ for domain 2 and obtain
\begin{align}
T_2(r) &= T_m\zeta_2(r), \five p_2(r) = p_m[\zeta_2(r)]^{n_2+1}, \five \rho(r) = \rho_m[\zeta_2(r)]^{n_2},\label{eq:poly2}\\
\text{where}\five \zeta_2(r)&\coloneqq a_2\bigg{(}\frac{r_m}{r}\bigg{)} + (1-a_2)\five\andd a_2\coloneqq \frac{GM}{(n_2+1)\mathcal{R}T_mr_m}.
\end{align}
The entropy profile for domain 2 is 
\begin{align}
s_2(r) &= \cv\bigg{(}\frac{n_2}{n_0} - 1\bigg{)}\bigg{\{}\ln{\bigg{(}\frac{r}{r_m}\bigg{)}} - \ln{\bigg{[}a_2+(1-a_2)\bigg{(}\frac{r}{r_m}\bigg{)}\bigg{]}}\bigg{\}},\nonumber\\
\andd  s_2^\prime(r) &= \bigg{(}\frac{n_2}{n_0} - 1\bigg{)}\bigg{[}\frac{\cv}{r+(1-a_2)r^2/(a_2r_m)}\bigg{]}.
\end{align}

In order to smoothly match the two polytropes together, we define ``smooth Heaviside functions" with width $\delta r_m$, $\delta$ being a dimensionless constant:
\begin{align}
f_1(r) &= \frac{1}{2}\bigg{[}1+\tanh{\bigg{(}\frac{r - r_m}{\delta r_m}\bigg{)}}\bigg{]},\\
f_2(r) &= \frac{1}{2}\bigg{[}1-\tanh{\bigg{(}\frac{r - r_m}{\delta r_m}\bigg{)}}\bigg{]} = 1 - f_1(r),
\end{align}
In practice, usually $\delta \sim 0.01$ results in a transition that is sufficiently smooth and rapid. 

We then define the thermodynamic profiles via
\begin{subequations}
	\begin{align}
	\rho(r) &= f_1(r)\rho_1(r) + f_2(r)\rho_2(r),\\
	p(r) &= f_1(r)p_1(r) + f_2(r)p_2(r),\\
	T(r) &= f_1(r)T_1(r) + f_2(r)T_2(r),\\
	\andd s(r) &= f_1(r)s_1(r) + f_2(r)s_2(r)\label{eq:sdoublepoly},
	\end{align}
\end{subequations}
yielding asymptotic behavior (i.e., where $|r-r_i|/r_i \gg \delta$) that is a polytrope of index $n_1$ in domain 1 and a polytrope of index 2 in domain 2. One disadvantage of smoothing the two polytropes in this way is that the resulting atmosphere is not exactly in hydrostatic equilibrium nor exactly satisfies the ideal gas law. A better approach might be to specify $s(r)$ via \eqref{eq:sdoublepoly} and then numerically integrate equations \eqref{eq:tgeneral} and \eqref{eq:pgeneral}. 

\subsection{Atmosphere with constant entropy gradient}
A simple example of a fluid layer that is stable to convection is one with a constant ``stiffness," or entropy gradient. In this case, we can write
\begin{align}
s(r) = k\cp\bigg{(}\frac{r}{r_i} - 1\bigg{)}\five \andd s^\prime(r) = \frac{k\cp}{r_i}, 
\end{align}
where $k$ is a dimensionless constant ($>0$) representing the stiffness of the stable region. We can then use \eqref{eq:tgeneral} to find $T(r)$:
\begin{align}
T(r) = -\frac{GM}{\cp}\big{\{}e^{k[(r/r_i) - 1]}\big{\}}\underbrace{\int_{r_i}^r \frac{e^{-k[(r/r_i) - 1]}d\tilde{r} }{\tilde{r}^2}}_{\coloneqq \mathscr{I}(r)}\ +\ T_ie^{k[(r/r_i)-1]}.
\end{align}
The integral $\mathscr{I}(r)$ does not have an analytic solution; however, it can be recast in terms of the exponential integral function 
\begin{align}
E_n(x) \coloneqq \int_1^\infty\frac{e^{-xt}}{t^n}dt = x^{n-1}\int_x^\infty \frac{e^{-t}}{t^n}dt,
\label{def:en}
\end{align}
yielding
\begin{align*}
\mathscr{I}(r) &= e^k\int_{r_i}^r \frac{e^{-k\tr/r_i}}{\tr^2}d\tr = e^k\int_k^{kr/r_i}\frac{e^{-t}(r_i/k)}{(r_i/k)^2t^2}dt\\
&= \frac{ke^k}{r_i}\int_k^{kr/r_i}\frac{e^{-t}}{t^2}dt = \frac{ke^k}{r_i}\bigg{[}\underbrace{\int_k^\infty\frac{e^{-t}}{t^2}dt}_{E_2(k)/k} - 
\underbrace{\int_{kr/r_i}^\infty\frac{e^{-t}}{t^2}dt}_{E_2(kr/r_i)/(kr/r_i)}  \bigg{]}\\
&= \frac{e^kE_2(k)}{r_i} - \frac{e^kE_2(kr/r_i)}{r} = \frac{e^k}{r_i}\bigg{[}E_2(k) - \bigg{(}\frac{r_i}{r}\bigg{)}E_2\bigg{(}\frac{kr}{r_i}\bigg{)}\bigg{]}
\end{align*}
and
\begin{align*}
T(r) = -e^k\frac{GM}{\cp r_i}\bigg{[}E_2(k) - \bigg{(}\frac{r_i}{r}\bigg{)}E_2\bigg{(}\frac{kr}{r_i}\bigg{)}\bigg{]}\ekp + T_i\ekp
\end{align*}
or
\begin{subequations}\label{eq:tconstsgrad}
\begin{empheq}[box=\fbox]{align}
T(r) &= T_i\bigg{\{}\bigg{[}e^ka_0\bigg{(}\frac{r_i}{r}\bigg{)} E_2\bigg{(}\frac{kr}{r_i}\bigg{)} + (1-e^kE_2(k)a_0)\bigg{]}\ekp\bigg{\}},\\
\five \text{with}\five a_0 &\coloneqq \frac{GM}{\cp T_i r_i} \ \text{(again)}.\\
\five &\text{(constant entropy gradient).}\nonumber
\end{empheq}
\end{subequations}
Using \eqref{eq:pgeneral} and \eqref{eq:rhogeneral} then yields
	\begin{empheq}[box=\fbox]{align}
p(r) =\ &p_i\exp{\bigg{[}-\frac{\gamma}{\gamma-1}k\bigg{(}\frac{r}{r_i}\bigg{)}\bigg{]}}\nonumber\\
	&\times\bigg{\{}\bigg{[}e^ka_0\bigg{(}\frac{r_i}{r}\bigg{)} E_2\bigg{(}\frac{kr}{r_i}\bigg{)} + (1-e^kE_2(k)a_0)\bigg{]}\ekp\bigg{\}}^{\gamma/(\gamma-1)}\\
\andd \rho(r) =\ &\rho_i\exp{\bigg{[}-\frac{\gamma}{\gamma-1}k\bigg{(}\frac{r}{r_i}\bigg{)}\bigg{]}}\nonumber\\
&\times
\bigg{\{}\bigg{[}e^ka_0\bigg{(}\frac{r_i}{r}\bigg{)} E_2\bigg{(}\frac{kr}{r_i}\bigg{)} + (1-e^kE_2(k)a_0)\bigg{]}\ekp\bigg{\}}^{1/(\gamma-1)}\\
\five &\text{(constant entropy gradient).}\nonumber
\end{empheq}
Clearly if $k=0$, we recover equations \eqref{eq:tad}--\eqref{eq:rhoad} for an adiabatic atmosphere (note that $E_n(0)\equiv 1$ for all $n$). 
\end{document}