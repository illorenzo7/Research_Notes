% document type and language
\documentclass[12pt]{article}

% standard packages
\usepackage{amsmath, bm, empheq, mathrsfs, natbib, cancel}
% label equations by section first, then equation in that section
\numberwithin{equation}{section}

% normal margins
\usepackage[margin=1in]{geometry}

% plane blue hyperlinks
\usepackage[colorlinks]{hyperref}
\hypersetup{
	colorlinks = true,
	linkcolor=blue,
	urlcolor=blue,
	citecolor=blue
}

% common macros
% Not sure where the following came from...maybe remove
\newcommand{\twochoices}[2]{\left\{ \begin{array}{lcc}
        \displaystyle #1 \\ \vspace{-10pt} \\
        \displaystyle #2 \end{array} \right. } %}

\newcommand{\threechoices}[3]{\left\{ \begin{array}{lcc}
        #1 \\ #2 \\ #3 \end{array} \right. }    %}

\newcommand{\fourchoices}[4]{\left\{ \begin{array}{lcc}
        #1 \\ #2 \\ #3 \\ #4 \end{array} \right. }      %}

\newcommand{\twovec}[2]{\left(\begin{array}{c} #1 \\ #2 \end{array}\right)}
\newcommand{\threevec}[3]{\left(\begin{array}{c} #1 \\ #2 \\ #3 \end{array}\right)}
\newcommand{\twomatrix}[4]{\left(\begin{array}{cc} #1 & #2 \\ #3 & #4 \end{array}\right)}

% MY MACROS

% MY EMAIL
\newcommand{\myemail}{loren.matilsky@gmail.com}

% MATH OPERATORS
\newcommand{\pderiv}[2]{\frac{\partial#1}{\partial#2}}
\newcommand{\matderiv}[1]{\frac{D#1}{Dt}}
\newcommand{\pderivline}[2]{\partial#1/\partial#2}
\newcommand{\parenfrac}[2]{\left(\frac{#1}{#2}\right)}
\newcommand{\brackfrac}[2]{\left[\frac{#1}{#2}\right]}
\newcommand{\bracefrac}[2]{\left\{\frac{#1}{#2}\right\}}
\newcommand{\av}[1]{\left\langle#1\right\rangle}
\newcommand{\avsph}[1]{\left\langle#1\right\rangle_{\rm{sph}}}
\newcommand{\avspht}[1]{\left\langle#1\right\rangle_{ {\rm sph}, t}}
\newcommand{\avt}[1]{\left\langle#1\right\rangle_{t}}
\newcommand{\avphi}[1]{\left\langle#1\right\rangle_{\phi}}
\newcommand{\avphit}[1]{\left\langle#1\right\rangle_{\phi,t}}
\newcommand{\avvol}[1]{\left\langle#1\right\rangle_{\rm{v}}}

\newcommand{\avalt}[1]{\langle#1\rangle}
\newcommand{\avaltsph}[1]{\langle#1\rangle_{\rm{sph}}}
\newcommand{\avaltt}[1]{\langle#1\rangle_{\rm{t}}}
\newcommand{\avaltphi}[1]{\langle#1\rangle_{\phi}}
\newcommand{\avaltphit}[1]{\langle#1\rangle_{\phi,t}}
\newcommand{\avaltvol}[1]{\langle#1\rangle_{\rm{v}}}

\newcommand{\sn}[2]{#1\times10^{#2}}
\newcommand{\define}{\coloneqq}
\newcommand{\definealt}{\equiv}
%\newcommand{\define}{\equiv}

% TEXT OPERATORS
\newcommand{\five}{\ \ \ \ \ }
\newcommand{\orr}{\text{or}\five }
\newcommand{\andd}{\text{and}\five }
\newcommand{\where}{\text{where}\five }
\newcommand{\with}{\text{with}\five }

% VECTOR SHORCUTS

% operators
\newcommand{\curl}{\nabla\times}
\newcommand{\Div}{\nabla\cdot}
\newcommand{\lap}{\nabla^2}
\newcommand{\dotgrad}{\cdot\nabla}
\newcommand{\ugrad}{\bm{u}\dotgrad}

% unit vectors
\newcommand{\e}{\hat{\bm{e}}}
\newcommand{\er}{\e_r}
\newcommand{\et}{\e_\theta}
\newcommand{\ep}{\e_\phi}
\newcommand{\el}{\e_\lambda}
\newcommand{\ez}{\e_z}
\newcommand{\exi}{\e_\xi}
\newcommand{\eeta}{\e_\eta}
\newcommand{\epol}{\e_{\rm{pol}}}

% REFERENCE STATE AND THERMO. VARIABLES
% may want to append this
\newcommand{\ofr}{(r)}

% reference-state constantsF_{\rm nr}
\newcommand{\cv}{c_{\rm{v}}}
\newcommand{\cp}{c_{\rm{p}}}
\newcommand{\cpcap}{C_{\rm{p}}}
\newcommand{\cvcap}{C_{\rm{v}}}
\newcommand{\cs}{c_{\rm s}}
\newcommand{\gasconst}{\mathcal{R}}
\newcommand{\gammaone}{\Gamma_1}
\newcommand{\omref}{\Omega_0}
\newcommand{\omrefvec}{\bm{\Omega}_0}

% total thermal variables (may wish to switch this stuff around later--I've always hated this notation of "no subscripts" = perturbation
\newcommand{\tot}{_{\rm{tot}}}
\newcommand{\rhotot}{\rho\tot}
\newcommand{\tmptot}{T\tot}
\newcommand{\prstot}{P\tot}
\newcommand{\stot}{S\tot}
\newcommand{\dsdrtot}{\frac{dS\tot}{dr}}
\newcommand{\dsdrtotline}{dS\tot/dr}

% reference or mean state
\newcommand{\rhoover}{\overline{\rho}}
\newcommand{\tmpover}{\overline{T}}
\newcommand{\prsover}{\overline{P}}
\newcommand{\entrover}{\overline{S}}
\newcommand{\inteover}{\overline{U}}
\newcommand{\enthover}{\overline{h}}
\newcommand{\heatover}{\overline{Q}}
\newcommand{\coolover}{\overline{C}}
\newcommand{\nsqover}{\overline{N^2}}
\newcommand{\gover}{\overline{g}}
\newcommand{\nuover}{\overline{\nu}}
\newcommand{\kappaover}{\overline{\kappa}}
\newcommand{\etaover}{\overline{\eta}}
\newcommand{\muover}{\overline{\mu}}
\newcommand{\deltaover}{\overline{\delta}}
\newcommand{\cpover}{\overline{\cpcap}}
\newcommand{\cvover}{\overline{\cvcap}}
\newcommand{\cssqover}{\overline{\cs^2}}

\newcommand{\rhotilde}{\tilde{\rho}}
\newcommand{\tmptilde}{\tilde{T}}
\newcommand{\prstilde}{\tilde{P}}
\newcommand{\entrtilde}{\tilde{S}}
\newcommand{\intetilde}{\tilde{U}}
\newcommand{\enthtilde}{\tilde{h}}
\newcommand{\heattilde}{\tilde{Q}}
\newcommand{\cooltilde}{\tilde{C}}
\newcommand{\nsqtilde}{\tilde{N^2}}
\newcommand{\gtilde}{\tilde{g}}
\newcommand{\nutilde}{\tilde{\nu}}
\newcommand{\kappatilde}{\tilde{\kappa}}
\newcommand{\etatilde}{\tilde{\eta}}
\newcommand{\mutilde}{\tilde{\mu}}
\newcommand{\deltatilde}{\tilde{\delta}}
\newcommand{\cptilde}{\tilde{\cpcap}}
\newcommand{\cvtilde}{\tilde{\cvcap}}
\newcommand{\cssqtilde}{\tilde{\cs^2}}

% perturbations from reference state
\newcommand{\rhoprime}{{\rho^\prime}}
\newcommand{\tmpprime}{{T^\prime}}
\newcommand{\prsprime}{{P^\prime}}
\newcommand{\entrprime}{{S^\prime}}
\newcommand{\inteprime}{{U^\prime}}
\newcommand{\enthprime}{{h^\prime}}

\newcommand{\rhohat}{\hat{\rho}}
\newcommand{\tmphat}{\hat{T}}
\newcommand{\prshat}{\hat{P}}
\newcommand{\entrhat}{\hat{S}}
\newcommand{\intehat}{\hat{U}}
\newcommand{\enthhat}{\hat{h}}

\newcommand{\rhoone}{\rho_1}
\newcommand{\tmpone}{T_1}
\newcommand{\prsone}{P_1}
\newcommand{\entrone}{S_1}
\newcommand{\inteone}{U_1}
\newcommand{\enthone}{h_1}

\newcommand{\pomega}{\varpi}

\newcommand{\fluxnr}{F_{\rm nr}}
\newcommand{\fluxnrtilde}{\widetilde{F_{\rm nr}}}

\newcommand{\grav}{g}
\newcommand{\vecg}{\bm{g}}
\newcommand{\geff}{g_{\rm{eff}}}
\newcommand{\vecgeff}{\bm{g}_{\rm{eff}}}

\newcommand{\heat}{Q}
\newcommand{\buoyfreq}{N}
\newcommand{\nsq}{N^2}

% reference-state derivatives
\newcommand{\dlnrho}{\frac{d\ln\rhoover}{dr}}
\newcommand{\dlntmp}{\frac{d\ln\tmpover}{dr}}
\newcommand{\dlnprs}{\frac{d\ln\prsover}{dr}}
\newcommand{\dsdr}{\frac{d\overline{S}}{dr}}

\newcommand{\dlnrholine}{d\ln\rhoover/dr}
\newcommand{\dlntmpline}{d\ln\tmpover/dr}
\newcommand{\dlnprsline}{d\ln\prsover/dr}
\newcommand{\dsdrline}{d\overline{S}/dr}

\newcommand{\hrho}{H_\rho}
\newcommand{\hprs}{H_{\rm{p}}}

% FLUID VARIABLES

% vector fields
\newcommand{\vecu}{\bm{u}}
\newcommand{\vecb}{\bm{B}}
\newcommand{\vecom}{\bm{\omega}}
\newcommand{\vecj}{\bm{\mathcal{J}}}

\newcommand{\upol}{\vecu_{\rm{pol}}}
\newcommand{\bpol}{\vecb_{\rm{pol}}}

\newcommand{\urad}{u_r}
\newcommand{\ut}{u_\theta}
\newcommand{\up}{u_\phi}
\newcommand{\ul}{u_\lambda}
\newcommand{\uz}{u_z}

\newcommand{\omr}{\omega_r}
\newcommand{\omt}{\omega_\theta}
\newcommand{\omp}{\omega_\phi}
\newcommand{\oml}{\omega_\lambda}
\newcommand{\omz}{\omega_z}

\newcommand{\br}{B_r}
\newcommand{\bt}{B_\theta}
\newcommand{\bp}{B_\phi}
\newcommand{\bl}{B_\lambda}
\newcommand{\bz}{B_z}

\newcommand{\jr}{\mathcal{J}_r}
\newcommand{\jt}{\mathcal{J}_\theta}
\newcommand{\jp}{\mathcal{J}_\phi}
\newcommand{\jl}{\mathcal{J}_\lambda}
\newcommand{\jz}{\mathcal{J}_z}

% spherical coordinates
\newcommand{\cost}{\cos\theta}
\newcommand{\sint}{\sin\theta}
\newcommand{\cott}{\cot\theta}
\newcommand{\rsint}{r\sint}
\newcommand{\orsint}{\frac{1}{\rsint}}
\newcommand{\orsintline}{(1/\rsint)}
\newcommand{\rt}{r\theta}


% SIMULATION GEOMETRY
%s subscripts
\newcommand{\minn}{_{\rm{min}}}
\newcommand{\maxx}{_{\rm{max}}}
\newcommand{\inn}{_{\rm{in}}}
\newcommand{\out}{_{\rm{out}}}
\newcommand{\bott}{_{\rm{bot}}}
\newcommand{\midd}{_{\rm{mid}}}
\newcommand{\topp}{_{\rm{top}}}
\newcommand{\bcz}{_{\rm{bcz}}}
\newcommand{\ov}{_{\rm{ov}}}
\newcommand{\rms}{_{\rm{rms}}}
\newcommand{\const}{_{\rm{const}}}

\newcommand{\lmax}{{\ell_{\rm{max}}}}

% SOLAR AND STELLAR VARIABLES
\newcommand{\rsun}{R_\odot}
\newcommand{\rtach}{r_t}
\newcommand{\lsun}{L_\odot}
\newcommand{\omsun}{\Omega_\odot}

\newcommand{\msun}{M_\odot}
\newcommand{\rstar}{R_*}
\newcommand{\lstar}{L_*}
\newcommand{\mstar}{M_*}
\newcommand{\omstar}{\Omega_*}

\newcommand{\rearth}{R_\oplus}
\newcommand{\omearth}{\Omega_\oplus}
\newcommand{\mearth}{M_\oplus}

% TORQUE DEFINITIONS
\newcommand{\taurs}{\tau_{\rm{rs}}}
\newcommand{\taurad}{\tau_{\rm{rad}}}
\newcommand{\taums}{\tau_{\rm{ms}}}
\newcommand{\taumm}{\tau_{\rm{mm}}}
\newcommand{\taumc}{\tau_{\rm{mc}}}
\newcommand{\tauv}{\tau_{\rm{v}}}
\newcommand{\taumag}{\tau_{\rm{mag}}}

% stellar time scales
\newcommand{\pes}{{P_{\rm{ES}}}}
\newcommand{\pessun}{{P_{ {\rm ES}, \odot}}}
\newcommand{\pnu}{{P_{\nu}}}
\newcommand{\pkappa}{{P_{\kappa}}}
\newcommand{\peta}{{P_{\eta}}}
\newcommand{\prot}{{P_{\rm{rot}}}}
\newcommand{\pequil}{{P_{\rm{eq}}}}
\newcommand{\tequil}{{t_{\rm{eq}}}}
\newcommand{\tmax}{{t_{\rm{max}}}}
\newcommand{\pcyc}{{P_{\rm{cyc}}}}
\newcommand{\omcyc}{{\omega_{\rm{cyc}}}}
\newcommand{\pcycm}{{P_{{\rm cyc}, m}}}
\newcommand{\omcycm}{{\omega_{{\rm{cyc}}, m}}}
% NON-DIMENSIONAL NUMBERS

% input non-d
\newcommand{\nrho}{N_\rho}

\newcommand{\ra}{{\rm{Ra}}}
\newcommand{\ramod}{\ra^*}
\newcommand{\raf}{\ra_{\rm{F}}}
\newcommand{\rafmod}{\raf^*}

\newcommand{\pr}{{\rm{Pr}}}
\newcommand{\prm}{{\rm{Pr_m}}}

\newcommand{\di}{{\rm{Di}}}

\newcommand{\ek}{{\rm{Ek}}}
\newcommand{\ta}{{\rm{Ta}}}

%\newcommand{\he}{{\rm{He}}}

\newcommand{\bu}{{\rm{Bu}}}
\newcommand{\bumod}{{\rm{Bu^*}}}
\newcommand{\bvisc}{{\rm{Bu_{visc}}}}
\newcommand{\brot}{{\rm{Bu_{rot}}}}

% output non-d
\newcommand{\ro}{{\rm{Ro}}}
\newcommand{\lo}{{\rm{Lo}}}
\newcommand{\roc}{{\rm{Ro_c}}}

\newcommand{\re}{{\rm{Re}}}
\newcommand{\rem}{{\rm{Re_m}}}

% FLUX ALIASES
\newcommand{\flux}{{\bm{\mathcal{F}}}}
\newcommand{\fcond}{\flux_{\rm{cond}}}
\newcommand{\frad}{\flux_{\rm{rad}}}
\newcommand{\fluxscalar}{{\mathcal{F}}}
\newcommand{\fcondscalar}{\fluxscalar_{\rm{cond}}}
\newcommand{\fenthscalar}{\fluxscalar_{\rm{enth}}}

% UNITS
\newcommand{\gram}{{\rm{g}}}
\newcommand{\cm}{{\rm{cm}}}
\newcommand{\second}{{\rm{s}}}
\newcommand{\gauss}{{\rm{G}}}
\newcommand{\kelv}{{\rm{K}}}
\newcommand{\unitent}{{\rm{erg\ g^{-1}\ K^{-1}}}}
\newcommand{\uniten}{\rm{erg}\ \cm^{-3}}
\newcommand{\unitprs}{\rm{dyn}\ \cm^{-2}}
\newcommand{\unitrho}{\gram\ \cm^{-3}}
\newcommand{\stoke}{\rm{cm^2\ s^{-1}}}

% MEAN FIELD THEORY
\newcommand{\meanb}{\overline{\bm{B}}}
\newcommand{\flucb}{\bm{B}^\prime}
\newcommand{\totb}{\bm{B}}

\newcommand{\meanv}{\overline{\bm{v}}}
\newcommand{\flucv}{\bm{v}^\prime}
\newcommand{\totv}{\bm{v}}

\newcommand{\emf}{\bm{\mathcal{E}}}
\newcommand{\meanemf}{\overline{\bm{\mathcal{E}}}}
\newcommand{\meanbpol}{\overline{\bm{B}_{\rm{pol}}}}

% SIMULATION CODES
\newcommand{\rayleigh}{\texttt{Rayleigh}}
\newcommand{\rayleigha}{\texttt{Rayleigh 0.9.1}}
\newcommand{\rayleighb}{\texttt{Rayleigh 1.0.1}}

\newcommand{\eulag}{\texttt{EULAG}}
\newcommand{\eulagmhd}{\texttt{EULAG-MHD}}
\newcommand{\ash}{\texttt{ASH}}
\newcommand{\rsst}{\texttt{RSST}}
\newcommand{\rtdt}{\texttt{R2D2}}
\newcommand{\pencil}{\texttt{Pencil}}

% other macros
\newcommand{\nad}{n_{\rm{ad}}}
\newcommand{\dimm}{_{\rm{dim}}}
\newcommand{\cz}{_{\rm{CZ}}}
\newcommand{\wl}{_{\rm{WL}}}
\newcommand{\dcool}{\delta_{\rm{cool}}}
% date, author, title
\date{\today}
\author{Loren Matilsky}
\title{Non-Dimensionalization of an Anelastic Stable--Unstable Layer in {\rayleigh}}

%\allowdisplaybreaks
\begin{document}
	\maketitle
	\section{General Equations Solved in {\rayleigh}}
	In general (with rotation and magnetism), {\rayleigh} evolves in time a set of coupled PDEs for the 3D vector velocity $\vecu$, vector magnetic field $\vecb$, pressure perturbation $P$ (perturbation away from the ``reference" or ``background" state), and entropy perturbation $S$. Note that $S$ can also be interpreted as a temperature perturbation in Boussinesq mode. For more details, see {\rayleigh}'s \href{https://rayleigh-documentation.readthedocs.io/en/latest/doc/source/User_Guide/physics_math_overview.html#the-system-of-equations-solved-in-rayleigh}{Documentation}. 
	
	We use standard spherical coordinates $(r,\theta,\phi)$ and cylindrical coordinates $(\lambda,\phi,z) = (r\sint,\phi, r\cost)$, and $\e_q$ in general denotes a position-dependent unit vector in the direction of increasing $q$. The full PDE-set is then:
	\begin{align}
	\Div[f_1\ofr\vecu] &= 0\label{eq:contgen},\\
	\Div \vecb &= 0,
\end{align}
\begin{subequations}\label{eq:momgen}
	\begin{align}
		f_1\ofr\left[\matderiv{\vecu}+c_1\ez\times\vecu\right] =&\ c_2f_2\ofr S\er - c_3 f_1\ofr\nabla\left[ \frac{P}{f_1\ofr} \right], \nonumber\\
		& +c_4(\curl\vecb)\times\vecb+c_5\Div\bm{D},\\
		\where D_{ij} \define&\ 2f_1\ofr f_3\ofr \left[e_{ij} - \frac{1}{3}(\Div\vecu) \delta_{ij} \right]\label{eq:dstressgen}\\
		\andd e_{ij} \define&\ \frac{1}{2}\left(\pderiv{u_i}{x_j} + \pderiv{u_j}{x_i} \right),\label{eq:estressgen}
	\end{align}
\end{subequations}
\begin{align}\label{eq:engen}
	f_1\ofr f_4\ofr \left[\matderiv{S} + f_{14}\ofr u_r\right] =&\ c_6\Div[f_1\ofr f_4\ofr f_5\ofr \nabla S] \nonumber \\
	&+ c_{10}f_6(r) + c_8 c_5 D_{ij}e_{ij} +c_9c_7f_7\ofr|\curl\vecb|^2,
\end{align}
\begin{align}\label{eq:indgen}
	\andd \pderiv{\vecb}{t} = \curl\left[\vecu\times\vecb - c_7 f_7\ofr\curl\vecb\right],
\end{align}
where $D/Dt\define \partial/\partial t+\vecu\cdot\nabla$ denotes the material derivative. 
	The spherically-symmetric, time-independent reference (or background) functions $f_i(r)$ and constants $c_j$ set the fluid approximation to be made. {\rayleigh} has built-in modes to set the $f$'s and $c$'s for single-layer (i.e., either convectively stable or unstable, but not both) Boussinesq or Anelastic spherical shells. More complex systems (coupled stable--unstable systems or alternative non-dimensionalizations) require the user to manually change the $f$'s and $c$'s. This can be done by editing an input binary file that {\rayleigh} reads upon initialization. The $c$'s can also be changed in the ASCII text-file (i.e., the \texttt{main\_input} file). 
	
	\section{Dimensional Anelastic Equations}
	We begin by writing down the full dimensional anelastic fluid equations, as they are usually implemented in {\rayleigh} (\texttt{reference\_type = 2}). We differ slightly from ``tradition" by assuming at the outset that there is both volumetric heating (preferentially in the bottom of the layer), $\qref\ofr$ and volumetric cooling (preferentially at the top of the layer), $\cref\ofr$ (our convention is that both $\qref$ and $\cref$ are positive; the cooling gets subtracted).
	
	This form of the anelastic approximation in a spherical shell is derived in, or more accurately, attributed to (since {\rayleigh} ``updates" the background state slightly differently than the cluge-y \texttt{ASH} implementation), two common sources: \citet{Gilman1981} and \citet{Clune1999}. {\rayleigh}'s dimensional anelastic equation-set is then:
	\begin{align}
		\Div[\rhoref\ofr\vecu] &= 0\label{eq:contdim},\\
		\Div \vecb &= 0,
	\end{align}
	\begin{subequations}\label{eq:momdim}
	\begin{align}
		\rhoref\ofr\left[\matderiv{\vecu}+2\Omega_0\ez\times\vecu\right] = &  \left[\frac{\rhoref\ofr \gref\ofr}{\cp}\right] S\er-\rhoref\ofr\nabla\left[ \frac{P}{\rhoref\ofr} \right], \nonumber\\
		&+ \frac{1}{\mu}(\curl\vecb)\times\vecb+\Div\bm{D},\\
		\where D_{ij} &\define 2\rhoref\ofr\nuref\ofr \left[e_{ij} - \frac{1}{3}(\Div\vecu) \delta_{ij} \right]\label{eq:dstressdim}\\
		\andd e_{ij} &\define \frac{1}{2}\left(\pderiv{u_i}{x_j} + \pderiv{u_j}{x_i} \right),\label{eq:estressdim}
	\end{align}
	\end{subequations}
	\begin{align}\label{eq:endim}
		\rhoref\ofr\tmpref\ofr \left[\matderiv{S} + \frac{d\sref}{dr} u_r\right] = &\ \Div[\rhoref\ofr \tmpref\ofr \kapparef\ofr \nabla S] \nonumber \\
		&+ \qref(r) -\cref\ofr + D_{ij}e_{ij} + \frac{\etaref\ofr}{\mu}|\curl\vecb|^2,
	\end{align}
	\begin{align}\label{eq:inddim}
	\andd \pderiv{\vecb}{t} = \curl\left[\vecu\times\vecb - \eta\ofr\curl\vecb\right].
	\end{align}
	Here, the thermal variables $\rho$, $T$, $P$, and $S$ refer to the density, temperature, pressure, and entropy (respectively). The overbars denote the spherically-symmetric, time-independent background state. The lack of an overbar on a thermal variable indicates the (assumed small) perturbation from the background (for the entropy, $S/\cp$ is assumed small).
	
	Other background quantities that appear are the gravity $\gref\ofr$, the momentum, thermal, and magnetic diffusivities [$\nuref\ofr$, $\kapparef\ofr$, and $\etaref\ofr$ , respectively], the internal heating or cooling $\qref\ofr$, the frame rotation rate $\Omega_0$, the specific heat at constant pressure $\cp$, and the vacuum permeability $\mu$ ($=4\pi$ in c.g.s. units). The equations are written in a frame rotating with angular velocity $\Omega_0$ and the centrifugal force is neglected. 
	
	Note that the internal heating and cooling functions $\qref\ofr$ and $\cref\ofr$ are reference-state quantities (and thus assumed spherically-symmetric and time-independent) but should be interpreted as $\qref-\cref=-\Div\frad$, where $\frad$ is the radiative heat flux. Properly, $\qref$ should be proportional to the radiative diffusivity $\kappa_{\rm rad}$ (which takes on a specific form in the radiative diffusion approximation, derivable from the opacity) and to the gradient of the total temperature $\tmpref + T$; and $\cref$ should be calculated using complicated near-surface physics. 
	
	In {\rayleigh}, a convective layer is usually driven by a combination of internal heating and the thermal boundary conditions (which are conditions on $S$), that together ensure that an imposed energy flux is transported throughout the layer in a steady state. (Note that energy could also be forced across the layer by fixing the entropy $S$ at each boundary, such that an ``adverse" (negative) radial entropy gradient is obtained in a steady state).   \textbf{In the Jupiter models, which will have both internal heating and cooling, we will set $\pderivline{S}{r}\equiv0$ at both the top and bottom boundary (no conduction in or out), and the flux of energy across the system will be imposed purely by the combination $\qref-\cref$}. 
	
	%Note that Equation \eqref{eq:estressdim} is only valid in a Cartesian coordinate system ($x_1$, $x_2$, $x_3$) (with $i$ and $j$ running over 1, 2, 3) and is translated into spherical coordinates before being used in Rayleigh. 
	
	Also, we recall the relation
	\begin{align}
		\dsdr = \cp \frac{\nsqref\ofr}{\gref\ofr},
	\end{align}
	where $\nsqref\ofr$ is the squared buoyancy frequency, which we will use in favor of $\dsdrline$ in subsequent equations. 
	
	Note that the original equations in \citet{Gilman1981} and \citet{Clune1999} were derived assuming a nearly-adiabatic background state (i.e., $\dsdrline\approx0$). \citet{Brown2012} and \citet{Vasil2013} have raised concerns about using various anelastic approximations in stable layers due to non-energy-conserving gravity waves. Should we be concerned?
	
	\section{Non-Dimensional Scheme}
	We now non-dimensionalize Equations \eqref{eq:contdim}--\eqref{eq:inddim}, according to the following scheme:
	\begin{subequations}\label{eq:ndscheme}
	\begin{align}
		\nabla &\rightarrow \frac{1}{H}\nabla,\\
		t &\rightarrow \tau t,\\
		\vecu &\rightarrow \frac{H}{\tau} \vecu,\\
		S &\rightarrow (\Delta S) S,\\
		P &\rightarrow \tilde{\rho} \frac{H^2}{\tau^2} P,\\
		\vecb &\rightarrow (\mu\tilde{\rho})^{1/2}\frac{H}{\tau} \vecb,\\ 
		\rhoref\ofr &\rightarrow \tilde{\rho} \rhoref\ofr, \\
		\tmpref\ofr &\rightarrow \tilde{T} \tmpref\ofr,\\
		\gref\ofr &\rightarrow \tilde{g} \gref\ofr,\\
		\nsqref\ofr &\rightarrow \widetilde{N^2} \nsqref\ofr,\\
		\nuref\ofr &\rightarrow \tilde{\nu}\nuref\ofr,\\
		\kapparef\ofr &\rightarrow \tilde{\kappa}\kapparef\ofr,\\
		 \etaref\ofr &\rightarrow \tilde{\eta}\etaref\ofr,\\ 
		\qref\ofr &\rightarrow \tilde{C} \qref\ofr,\\
		\andd \cref\ofr &\rightarrow \tilde{Q} \cref\ofr.
	\end{align}
	\end{subequations}
	Here, $H$ is a typical length-scale, $\tau$ a typical time-scale, and $\Delta S$ a typical (\textit{estimated}) entropy scale (in Rayleigh-B\'enard-type convection, the true entropy difference is imposed directly, but we will set the true value indirectly via heating and cooling functions). On the right-hand-sides of Equation \eqref{eq:ndscheme} and in the following non-dimensionalizations, all fluid variables, coordinates, and background-state quantities are understood to be non-dimensional. The tildes refer to``typical values" of the (dimensional) reference-state functions. These typical values will be a volume-average over the convection zone (CZ) of the shell, except for $\widetilde{N^2}$, which will be a volume-average over the stably stratified weather layer (WL). Since cooling takes out what heating dumps in, we will normalize such that $\tilde{C}=\tilde{Q}$. 
	
	Below, we will assume the time-scale is either a thermal diffusion time (i.e., $\tau=H^2/\tilde{\nu}$) or a rotational time-scale [i.e., $\tau=(2\Omega_0)^{-1}$]. %To describe the reference state, we will consider three cases for a given function's ``typical value": Its value at the inner shell boundary, its value at the outer shell boundary, or its value volume-averaged over the shell. 
	
	\section{Non-Dimensional Equations, Non-Rotating ($\tau=H^2/\tilde{\kappa}$)}
	In this case, Equations \eqref{eq:contdim}--\eqref{eq:inddim} become 
	\begin{align}
	\Div[\rhoref\ofr\vecu] &= 0\label{eq:contndvisc},\\
	\Div \vecb &= 0,
\end{align}
\begin{subequations}\label{eq:momndvisc}
	\begin{align}
		\rhoref\ofr\left[\matderiv{\vecu}+\frac{\pr}{\ek}\ez\times\vecu\right] = &\ \pr\ra\rhoref\ofr\gref\ofr S\er         -\rhoref\ofr\nabla\left[ \frac{P}{\rhoref\ofr} \right], \nonumber\\
		&\ +(\curl\vecb)\times\vecb +\pr\Div\bm{D},\\
		\where D_{ij} &\define 2\rhoref\ofr\nuref\ofr \left[e_{ij} - \frac{1}{3}(\Div\vecu) \delta_{ij} \right]\\
		\andd e_{ij} &\define \frac{1}{2}\left(\pderiv{u_i}{x_j} + \pderiv{u_j}{x_i} \right),
	\end{align}
\end{subequations}
\begin{align}\label{eq:enndvisc}
	\rhoref\ofr\tmpref\ofr\left[ \matderiv{S} + \frac{\bu}{\ra}\frac{\nsqref\ofr}{\gref\ofr} u_r\right]  = &\ \Div[\rhoref\ofr \tmpref\ofr \kapparef\ofr \nabla S] \nonumber \\
	&+ \qref(r) -\cref\ofr + \frac{\di}{\ra} D_{ij}e_{ij} + \frac{\di}{\prm\ra} \etaref\ofr|\curl\vecb|^2,
\end{align}
\begin{align}\label{eq:indndvisc}
	\andd \pderiv{\vecb}{t} = \curl\left[\vecu\times\vecb - \frac{\pr}{\prm}\etaref\ofr\curl\vecb\right].
\end{align}	

The non-dimensional numbers appearing are:
\begin{subequations}
\begin{align}
	\ra &\define \frac{\tilde{g} H^3}{\tilde{\nu} \tilde{\kappa}} \frac{\Delta S}{\cp}\five\text{(Rayleigh number)},\\ 
	\pr &\define \frac{\tilde{\nu}}{\tilde{\kappa}}\five\text{(Prandtl number)},\\
	\prm &\define \frac{\tilde{\nu}}{\tilde{\eta}}\five\text{(magnetic Prandtl number)},\\
	\ek &\define \frac{\tilde{\nu}}{2\Omega_0H^2}\five\text{(Ekman number)},\\	
	\bu &\define \frac{\widetilde{N^2}H^4}{\tilde{\nu}\tilde{\kappa}}\five\text{(buoyancy number)},\\
	\andd \di &= \frac{\tilde{g}H}{\cp\tilde{T}}\five\text{(dissipation number)},
\end{align}
\end{subequations}
Note that in our convention, the dissipation number is not an independent control parameter, but a function of the non-dimensional parameters characterizing the reference state (this will be seen in Section \ref{sec:ref}). 

Note that we have chosen the entropy-scale (and thus the Rayleigh number) based on the internal heating:
\begin{subequations}
\begin{align}
	\Delta S &\define \frac{\tilde{Q}\tau}{\tilde{\rho}\tilde{T}} = \frac{\tilde{Q}H^2}{\tilde{\rho}\tilde{T}\tilde{\kappa}}\label{eq:dsnonrot}\\
	\andd \ra &\define \frac{\tilde{g} \tilde{Q} H^3\tau}{\tilde{\rho}\tilde{T}\cp\tilde{\nu} \tilde{\kappa}} =  \frac{\tilde{g} \tilde{Q} H^5}{\tilde{\rho}\tilde{T}\cp\tilde{\nu} \tilde{\kappa}^2},\label{eq:ranonrot}\\ 
\end{align}
\end{subequations}
where the first equality in each equation is general (it holds for any choice of time-scale $\tau$) and the second equality is specific to the non-rotating case. Essentially, we have assumed that the heating (or cooling) operates on the time-scale $\tau$ before the fluid parcel buoyantly moves to another part of the shell, carrying with it an entropy perturbation $\Delta S$ (in the non-rotating case, this should happen on the thermal dissipation time-scale $\tau=H^2/\tilde{\kappa}$). 

The user is thus free to choose the shapes of $\qref\ofr$ and $\cref\ofr$, but not their amplitude, since they must have unity volume-averages over the CZ. 

The buoyancy number $\bu$ is the ratio of the typical squared buoyancy frequency to the thermal and viscous diffusion times. It is essentially a ``second (stable) Rayleigh number", and measures the stiffness of the stable layer (recall $\widetilde{N^2}$ refers to the typical value of $\nsqref\ofr$ in the WL). The buoyancy number is independent of the Rayleigh number, which estimates the ultimate instability of the CZ. 

\section{Non-Dimensional Equations, Rotating [$\tau=(2\Omega_0)^{-1}$]}
In the previous section, $t$ (and things with time in the dimensions) was implied to mean $(\tilde{\kappa}/H^2) t_{\rm{dim}}$, where $t_{\rm{dim}}$ was the dimensional time. We now want to use a new non-dimensional time, $t_{\rm{new}}= \Omega_0t_{\rm{dim}} = (\pr/\ek)t$. We can thus find the new equations easily from Equations \eqref{eq:contndvisc}--\eqref{eq:indndvisc}. Every place we see a time dimension, we recall $t=(\ek /\pr)t_{\rm{new}}$, so we multiply the place where the time-dimension appears by $(\ek/\pr)$ and drop the ``new" subscript [e.g., $t\rightarrow (\ek/\pr)\ t$, $\vecu\rightarrow(\pr/\ek)\vecu$, etc.]  

Note that we should now choose a different typical entropy-scale and corresponding Rayleigh number:
\begin{subequations}
	\begin{align}
		\Delta S^*&\define \frac{\tilde{Q}\tau}{\tilde{\rho}\tilde{T}} = \frac{\tilde{Q}}{2\Omega_0\tilde{\rho}\tilde{T}}\label{eq:dsrot}\\
		\andd \ra &\define \frac{\tilde{g} \tilde{Q} H^3\tau}{\tilde{\rho}\tilde{T}\cp\tilde{\nu} \tilde{\kappa}} =  \frac{\tilde{g} \tilde{Q} H^3}{2\Omega_0\tilde{\rho}\tilde{T}\cp\tilde{\nu} \tilde{\kappa}}.\label{eq:rarot}
	\end{align}
\end{subequations}
The reasoning here is that under the influence of rapid rotation, the life-times of upflows or downflows are no longer set by the thermal dissipation time, but by the rotation period. Thus, the heating or cooling of a fluid parcel occurs on a shorter time-scale, leading to a smaller entropy difference across the shell than in the non-rotating case with the same amount of heating. Of course, neither Equations \eqref{eq:dsnonrot} or \eqref{eq:dsrot} are particularly convincing estimates and there is a large degree of uncertainty in the actual magnitude of typical entropy perturbations. We can only see how good these estimates are (after the fact) by checking if the achieved (non-dimensional) entropy difference across the shell winds up being close to unity. 

We thus find (after rearranging terms),
\begin{align}
	\Div[\rhoref\ofr\vecu] &= 0\label{eq:contndrot},\\
	\Div \vecb &= 0,
\end{align}
\begin{subequations}\label{eq:momndrot}
	\begin{align}
		\rhoref\ofr\left[\matderiv{\vecu} + \ez\times\vecu\right] = &\ \ramod \rhoref\ofr\gref\ofr S\er         -\rhoref\ofr\nabla\left[ \frac{P}{\rhoref\ofr} \right], \nonumber\\
		&+(\curl\vecb)\times\vecb + \ek \Div\bm{D},\\
		\where D_{ij} &\define 2\rhoref\ofr\nuref\ofr \left[e_{ij} - \frac{1}{3}(\Div\vecu) \delta_{ij} \right]\\
		\andd e_{ij} &\define \frac{1}{2}\left(\pderiv{u_i}{x_j} + \pderiv{u_j}{x_i} \right),
	\end{align}
\end{subequations}
\begin{align}\label{eq:enndrot}
	\rhoref\ofr\tmpref\ofr\left[ \matderiv{S}+ \frac{\bumod}{\ramod} \frac{\nsqref\ofr}{\gref\ofr} u_r\right]  = &\ \frac{\ek}{\pr} \Div[\rhoref\ofr \tmpref\ofr \kapparef\ofr \nabla S] \nonumber \\
	&+ \frac{\ek}{\pr} \qref(r) + \frac{\di\ek}{\ramod} D_{ij}e_{ij} + \frac{\di\ek}{\prm\ramod} \etaref\ofr|\curl\vecb|^2,
\end{align}
\begin{align}\label{eq:indndrot}
	\andd \pderiv{\vecb}{t} = \curl\left[\vecu\times\vecb - \frac{\ek}{\prm} \eta\ofr\curl\vecb\right],
\end{align}	
where we have introduced two modified non-dimensional numbers:
\begin{subequations}
	\begin{align}
		\ramod &\define \frac{\ek^2}{\pr}\ra =  \frac{\tilde{g} }{H\Omega_0^2} \frac{\Delta S}{\cp}=\frac{\tilde{g}\tilde{Q}}{(2\Omega_0)^3\tilde{\rho}\tilde{T}\cp H},\\ 
		\andd \bumod &\define \frac{\ek^2}{\pr}\bu = \frac{\widetilde{N^2}} {4\Omega_0^2} \sim \frac{\tilde{g} }{H\Omega_0^2} =  \frac{1}{\text{geometric\ oblateness}}.
	\end{align}
\end{subequations}

Note that although the ``$\dsdrline$-terms" in the non-dimensionalizations have seemingly different definitions, they are similar, since
\begin{align}
	\frac{\bu}{\ra} \sim \frac{\cp}{\Delta S}\five \andd 	\frac{\bumod}{\ramod} \sim \frac{\cp}{\Delta S^*}.
\end{align}
The only difference is in the different estimates $\Delta S$ and $\Delta S^*$. 

In terms of {\rayleigh}'s $f$'s and $c$'s, we compare Equations \eqref{eq:contndrot}--\eqref{eq:indndrot} to \eqref{eq:contgen}--\eqref{eq:indgen} and find:
\begin{align*}
	f_1&\rightarrow \rhoref\\
	f_2&\rightarrow\rhoref\gref \\
	f_3 &\rightarrow \nuref\\
	f_4 &\rightarrow \tmpref\\
	f_5 &\rightarrow \kapparef\\
	f_6 &\rightarrow \qref - \cref\\
	f_7 &\rightarrow \etaref\\
	\vdots\\
	f_{14} &\rightarrow \frac{\bumod}{\ramod}\frac{\nsqref}{\gref}
\end{align*}
\section{Hydrostatic, Ideal-Gas, Jovian Stable--Unstable Layer}
To model a background CZ and WL in Jupiter, we consider a spherical shell composed of an ideal, hydrostatic gas extending between inner radius $r\inn$ and outer radius $r\out$. An assumed transition in convective stability occurs near an intermediate radius $r_0$, over width $\delta$. More specifically, we choose quartic matching of the entropy gradient between the two layers:
\begin{align}
	\frac{d\sref}{dr} &=\psi\wl(r; r_0,\delta),
\end{align}
where
\begin{equation}
  \psi\wl(r;r_0, \delta)\define \begin{cases}
		0 & r\leq r_0 \\
		1 - \left[1 - \Big{(}\frac{r-r_0}{\delta}\Big{)}^2\right]^2 & r_0  < r < r_0+ \delta\\
		1 & r\geq r_0+\delta.
	\end{cases}
	\label{eq:dsdr}
\end{equation}
We also define
\begin{equation}
	\psi\cz(r;r_0, \delta)\define 1 - \psi\wl(r;r_0-\delta,\delta) =  \begin{cases}
		1 & r\leq r_0 \\
		1 - \left[1 - \Big{(}\frac{r-r_0}{\delta}\Big{)}^2\right]^2 & r_0 -\delta  < r < r_0\\
		0 & r\geq r_0.
	\end{cases}
	\label{eq:dsdr}
\end{equation}
$\psi\wl$ thus ``senses" only the WL, and $\psi\cz$ ``senses" only the CZ. 

With this formulation, the CZ is strictly unstable (really, marginally stable, but becomes unstable from the heating and cooling). This ensures that none of the stable gradient ``leaks" into the CZ, as happens with (e.g.) tanh matching. We assume a centrally-concentrated mass so that $\gref\ofr\propto 1/r^2$. 

It can then be shown that five non-dimensional parameters fully characterize the shell geometry, $\rhoref\ofr$, and $\tmpref\ofr$:
\begin{subequations}
	\begin{align}
		\alpha&\define \frac{r\out-r_0}{r_0-r\inn} \five\text{(WL-to-CZ\ aspect ratio)},\\ 
		\beta &\define \frac{r\inn}{r_0} \five\text{(CZ aspect ratio)},\\
		\gamma&\define \frac{\cp}{\cv}\five\text{(specific-heat ratio)},\\
		\delta &\five\text{(stability transition width)},\\	
		N_\rho &\define \ln\left[\frac{\rhoref(r\inn)}{\rhoref(r\out)}\right]\five\text{(number of density scale-heights across CZ)},
	\end{align}
\end{subequations}
where $\cv$ is the specific heat at constant volume. 

We choose $H$ to be the thickness of the CZ ($r_0-r\inn=1$ and $r\out-r_0=\alpha$). Thus,
\begin{subequations}
	\begin{align}
		r\inn &=\frac{1-\beta}{\beta},\\
		r_0 &= \frac{1}{1-\beta},\\
		\andd r\out &=  \frac{1}{1-\beta} + \alpha
	\end{align}
\end{subequations}
If we choose $\alpha=0.25$ and $\beta=0.9$, then $(r\inn,r_0,r\out)=(9, 10, 10.25)$. 

With the requirement that the volume-average of $\gref\ofr$ over the CZ be unity, we require
\begin{align}
	\gref\ofr &= \left[ \frac{1-\beta^3}{3(1-\beta)^3}    \right]\frac{1}{r^2}.
\end{align}

It can then be shown from the ideal-gas and hydrostatic conditions (e.g., \citealt{Matilsky2023d}) that
\begin{align}\label{eq:tmphat}
	\tmpref &= e^{\sref}\left[\tmpref(r_0) - \di\int_{r_0}^r \gref(x)  e^{-\sref(x)}dx\right].
\end{align}
and
\begin{align}\label{eq:rhohat}
	\rhoref &= \rhoref(r_0) \exp{\left[-\left(\frac{\gamma}{\gamma-1}\right)\sref\right]}\tmpref^{1/(\gamma-1)},
\end{align}
where (nastily)
\begin{align}\label{eq:digory}
	\di \define \frac{\tilde{g}H}{\cp\tilde{T}} &=  \frac{3 \beta (1 - \beta)^2 (1 - e^{-\nrho/n})} 
	{ (3\beta/2) (1 - \beta^2) (1 - e^{-\nrho/n}) - (1-\beta^3)(\beta-e^{-\nrho/n})},\\
	\tmpref(r_0)&= \frac{(1-\beta^3)(1-\beta)}{(3\beta/2)(1-\beta^2)(e^{\nrho/n}-1) - (1-\beta^3)(\beta e^{\nrho/n}-1)},\\
	\andd n &\define \frac{1}{1-\gamma}
\end{align}
is the polytropic index of the CZ (note that since $\dsdrline\equiv0$ in the CZ, the stratification of the CZ winds up being an adiabatic polytrope). We have also assumed (without loss of generality) that $\sref(r_0)=0$, so that $\sref\equiv0$ in the CZ). The constant $\rhoref(r_0)$ ends up not having an analytical expression but can be easily found by integration over the CZ of Equation \eqref{eq:rhohat} (and I should maybe just find $\di$ and $\tmpref(r_0)$ from integration, too, since I am error-prone).  

To follow \citet{Jones2011} \citet{Heimpel2022} somewhat, we assume $n=2$. Apparently this is a common choice for Jupiter that better approximates its weird equation of state without breaking the ideal gas law. It results in the somewhat strange result $\gamma=3/2$. 

We also simplify our lives and assume constant diffusivities:
\begin{align}
	\nuref\ofr=\kapparef\ofr=\etaref\ofr\equiv 1.
\end{align}

For the heating, we use the typical {\rayleigh} profile $\qref\propto \rhoref \tmpref$, but ensure that the heating is fully contained in the CZ:
\begin{align}
	\qref\ofr \propto \rhoref\ofr\tmpref\ofr \psi\cz(r; r_0,\delta)
\end{align} 

For the cooling, we choose a cooling width $\dcool$:
\begin{align}
	\cref\ofr \propto \exp\left[\frac{r-r_0}{\dcool}\right] \psi\cz(r; r_0,\delta)
\end{align}

We thus fully define the following geometry and Jovian-ish reference state:
\begin{subequations}
	\begin{align}
		\alpha &= 0.25,\\
		\beta &= 0.9,\\
		\gamma &= 3/2,\\
		\delta=\dcool &= 0.1\five \text{(may mess with these)},\\
		\andd N_\rho &= 3.
	\end{align}
\end{subequations}
\clearpage
\newpage
%\bibliography{/Users/loren/Desktop/Paper_Library/000_bibtex/library_propstyle, 
	\bibliography{/Users/loren/Desktop/Paper_Library/000_bibtex/library, 
			/Users/loren/Desktop/Paper_Library/000_bibtex/proceedings,
			/Users/loren/Desktop/Paper_Library/000_bibtex/books}
	\bibliographystyle{aasjournal}

\end{document}