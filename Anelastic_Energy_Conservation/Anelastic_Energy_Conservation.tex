\documentclass[12pt]{article}

% standard packages
\usepackage{amsmath, amssymb,bm, empheq, mathrsfs, natbib, cancel}

% normal margins
\usepackage[margin=1in]{geometry}

% plane blue hyperlinks
\usepackage[colorlinks]{hyperref}
\hypersetup{
	colorlinks = true,
	linkcolor=blue,
	citecolor=blue
}

% common macros
% Not sure where the following came from...maybe remove
\newcommand{\twochoices}[2]{\left\{ \begin{array}{lcc}
        \displaystyle #1 \\ \vspace{-10pt} \\
        \displaystyle #2 \end{array} \right. } %}

\newcommand{\threechoices}[3]{\left\{ \begin{array}{lcc}
        #1 \\ #2 \\ #3 \end{array} \right. }    %}

\newcommand{\fourchoices}[4]{\left\{ \begin{array}{lcc}
        #1 \\ #2 \\ #3 \\ #4 \end{array} \right. }      %}

\newcommand{\twovec}[2]{\left(\begin{array}{c} #1 \\ #2 \end{array}\right)}
\newcommand{\threevec}[3]{\left(\begin{array}{c} #1 \\ #2 \\ #3 \end{array}\right)}
\newcommand{\twomatrix}[4]{\left(\begin{array}{cc} #1 & #2 \\ #3 & #4 \end{array}\right)}

% MY MACROS

% MY EMAIL
\newcommand{\myemail}{loren.matilsky@gmail.com}

% MATH OPERATORS
\newcommand{\pderiv}[2]{\frac{\partial#1}{\partial#2}}
\newcommand{\matderiv}[1]{\frac{D#1}{Dt}}
\newcommand{\pderivline}[2]{\partial#1/\partial#2}
\newcommand{\parenfrac}[2]{\left(\frac{#1}{#2}\right)}
\newcommand{\brackfrac}[2]{\left[\frac{#1}{#2}\right]}
\newcommand{\bracefrac}[2]{\left\{\frac{#1}{#2}\right\}}
\newcommand{\av}[1]{\left\langle#1\right\rangle}
\newcommand{\avsph}[1]{\left\langle#1\right\rangle_{\rm{sph}}}
\newcommand{\avspht}[1]{\left\langle#1\right\rangle_{ {\rm sph}, t}}
\newcommand{\avt}[1]{\left\langle#1\right\rangle_{t}}
\newcommand{\avphi}[1]{\left\langle#1\right\rangle_{\phi}}
\newcommand{\avphit}[1]{\left\langle#1\right\rangle_{\phi,t}}
\newcommand{\avvol}[1]{\left\langle#1\right\rangle_{\rm{v}}}

\newcommand{\avalt}[1]{\langle#1\rangle}
\newcommand{\avaltsph}[1]{\langle#1\rangle_{\rm{sph}}}
\newcommand{\avaltt}[1]{\langle#1\rangle_{\rm{t}}}
\newcommand{\avaltphi}[1]{\langle#1\rangle_{\phi}}
\newcommand{\avaltphit}[1]{\langle#1\rangle_{\phi,t}}
\newcommand{\avaltvol}[1]{\langle#1\rangle_{\rm{v}}}

\newcommand{\sn}[2]{#1\times10^{#2}}
\newcommand{\define}{\coloneqq}
\newcommand{\definealt}{\equiv}
%\newcommand{\define}{\equiv}

% TEXT OPERATORS
\newcommand{\five}{\ \ \ \ \ }
\newcommand{\orr}{\text{or}\five }
\newcommand{\andd}{\text{and}\five }
\newcommand{\where}{\text{where}\five }
\newcommand{\with}{\text{with}\five }

% VECTOR SHORCUTS

% operators
\newcommand{\curl}{\nabla\times}
\newcommand{\Div}{\nabla\cdot}
\newcommand{\lap}{\nabla^2}
\newcommand{\dotgrad}{\cdot\nabla}
\newcommand{\ugrad}{\bm{u}\dotgrad}

% unit vectors
\newcommand{\e}{\hat{\bm{e}}}
\newcommand{\er}{\e_r}
\newcommand{\et}{\e_\theta}
\newcommand{\ep}{\e_\phi}
\newcommand{\el}{\e_\lambda}
\newcommand{\ez}{\e_z}
\newcommand{\exi}{\e_\xi}
\newcommand{\eeta}{\e_\eta}
\newcommand{\epol}{\e_{\rm{pol}}}

% REFERENCE STATE AND THERMO. VARIABLES
% may want to append this
\newcommand{\ofr}{(r)}

% reference-state constantsF_{\rm nr}
\newcommand{\cv}{c_{\rm{v}}}
\newcommand{\cp}{c_{\rm{p}}}
\newcommand{\cpcap}{C_{\rm{p}}}
\newcommand{\cvcap}{C_{\rm{v}}}
\newcommand{\cs}{c_{\rm s}}
\newcommand{\gasconst}{\mathcal{R}}
\newcommand{\gammaone}{\Gamma_1}
\newcommand{\omref}{\Omega_0}
\newcommand{\omrefvec}{\bm{\Omega}_0}

% total thermal variables (may wish to switch this stuff around later--I've always hated this notation of "no subscripts" = perturbation
\newcommand{\tot}{_{\rm{tot}}}
\newcommand{\rhotot}{\rho\tot}
\newcommand{\tmptot}{T\tot}
\newcommand{\prstot}{P\tot}
\newcommand{\stot}{S\tot}
\newcommand{\dsdrtot}{\frac{dS\tot}{dr}}
\newcommand{\dsdrtotline}{dS\tot/dr}

% reference or mean state
\newcommand{\rhoover}{\overline{\rho}}
\newcommand{\tmpover}{\overline{T}}
\newcommand{\prsover}{\overline{P}}
\newcommand{\entrover}{\overline{S}}
\newcommand{\inteover}{\overline{U}}
\newcommand{\enthover}{\overline{h}}
\newcommand{\heatover}{\overline{Q}}
\newcommand{\coolover}{\overline{C}}
\newcommand{\nsqover}{\overline{N^2}}
\newcommand{\gover}{\overline{g}}
\newcommand{\nuover}{\overline{\nu}}
\newcommand{\kappaover}{\overline{\kappa}}
\newcommand{\etaover}{\overline{\eta}}
\newcommand{\muover}{\overline{\mu}}
\newcommand{\deltaover}{\overline{\delta}}
\newcommand{\cpover}{\overline{\cpcap}}
\newcommand{\cvover}{\overline{\cvcap}}
\newcommand{\cssqover}{\overline{\cs^2}}

\newcommand{\rhotilde}{\tilde{\rho}}
\newcommand{\tmptilde}{\tilde{T}}
\newcommand{\prstilde}{\tilde{P}}
\newcommand{\entrtilde}{\tilde{S}}
\newcommand{\intetilde}{\tilde{U}}
\newcommand{\enthtilde}{\tilde{h}}
\newcommand{\heattilde}{\tilde{Q}}
\newcommand{\cooltilde}{\tilde{C}}
\newcommand{\nsqtilde}{\tilde{N^2}}
\newcommand{\gtilde}{\tilde{g}}
\newcommand{\nutilde}{\tilde{\nu}}
\newcommand{\kappatilde}{\tilde{\kappa}}
\newcommand{\etatilde}{\tilde{\eta}}
\newcommand{\mutilde}{\tilde{\mu}}
\newcommand{\deltatilde}{\tilde{\delta}}
\newcommand{\cptilde}{\tilde{\cpcap}}
\newcommand{\cvtilde}{\tilde{\cvcap}}
\newcommand{\cssqtilde}{\tilde{\cs^2}}

% perturbations from reference state
\newcommand{\rhoprime}{{\rho^\prime}}
\newcommand{\tmpprime}{{T^\prime}}
\newcommand{\prsprime}{{P^\prime}}
\newcommand{\entrprime}{{S^\prime}}
\newcommand{\inteprime}{{U^\prime}}
\newcommand{\enthprime}{{h^\prime}}

\newcommand{\rhohat}{\hat{\rho}}
\newcommand{\tmphat}{\hat{T}}
\newcommand{\prshat}{\hat{P}}
\newcommand{\entrhat}{\hat{S}}
\newcommand{\intehat}{\hat{U}}
\newcommand{\enthhat}{\hat{h}}

\newcommand{\rhoone}{\rho_1}
\newcommand{\tmpone}{T_1}
\newcommand{\prsone}{P_1}
\newcommand{\entrone}{S_1}
\newcommand{\inteone}{U_1}
\newcommand{\enthone}{h_1}

\newcommand{\pomega}{\varpi}

\newcommand{\fluxnr}{F_{\rm nr}}
\newcommand{\fluxnrtilde}{\widetilde{F_{\rm nr}}}

\newcommand{\grav}{g}
\newcommand{\vecg}{\bm{g}}
\newcommand{\geff}{g_{\rm{eff}}}
\newcommand{\vecgeff}{\bm{g}_{\rm{eff}}}

\newcommand{\heat}{Q}
\newcommand{\buoyfreq}{N}
\newcommand{\nsq}{N^2}

% reference-state derivatives
\newcommand{\dlnrho}{\frac{d\ln\rhoover}{dr}}
\newcommand{\dlntmp}{\frac{d\ln\tmpover}{dr}}
\newcommand{\dlnprs}{\frac{d\ln\prsover}{dr}}
\newcommand{\dsdr}{\frac{d\overline{S}}{dr}}

\newcommand{\dlnrholine}{d\ln\rhoover/dr}
\newcommand{\dlntmpline}{d\ln\tmpover/dr}
\newcommand{\dlnprsline}{d\ln\prsover/dr}
\newcommand{\dsdrline}{d\overline{S}/dr}

\newcommand{\hrho}{H_\rho}
\newcommand{\hprs}{H_{\rm{p}}}

% FLUID VARIABLES

% vector fields
\newcommand{\vecu}{\bm{u}}
\newcommand{\vecb}{\bm{B}}
\newcommand{\vecom}{\bm{\omega}}
\newcommand{\vecj}{\bm{\mathcal{J}}}

\newcommand{\upol}{\vecu_{\rm{pol}}}
\newcommand{\bpol}{\vecb_{\rm{pol}}}

\newcommand{\urad}{u_r}
\newcommand{\ut}{u_\theta}
\newcommand{\up}{u_\phi}
\newcommand{\ul}{u_\lambda}
\newcommand{\uz}{u_z}

\newcommand{\omr}{\omega_r}
\newcommand{\omt}{\omega_\theta}
\newcommand{\omp}{\omega_\phi}
\newcommand{\oml}{\omega_\lambda}
\newcommand{\omz}{\omega_z}

\newcommand{\br}{B_r}
\newcommand{\bt}{B_\theta}
\newcommand{\bp}{B_\phi}
\newcommand{\bl}{B_\lambda}
\newcommand{\bz}{B_z}

\newcommand{\jr}{\mathcal{J}_r}
\newcommand{\jt}{\mathcal{J}_\theta}
\newcommand{\jp}{\mathcal{J}_\phi}
\newcommand{\jl}{\mathcal{J}_\lambda}
\newcommand{\jz}{\mathcal{J}_z}

% spherical coordinates
\newcommand{\cost}{\cos\theta}
\newcommand{\sint}{\sin\theta}
\newcommand{\cott}{\cot\theta}
\newcommand{\rsint}{r\sint}
\newcommand{\orsint}{\frac{1}{\rsint}}
\newcommand{\orsintline}{(1/\rsint)}
\newcommand{\rt}{r\theta}


% SIMULATION GEOMETRY
%s subscripts
\newcommand{\minn}{_{\rm{min}}}
\newcommand{\maxx}{_{\rm{max}}}
\newcommand{\inn}{_{\rm{in}}}
\newcommand{\out}{_{\rm{out}}}
\newcommand{\bott}{_{\rm{bot}}}
\newcommand{\midd}{_{\rm{mid}}}
\newcommand{\topp}{_{\rm{top}}}
\newcommand{\bcz}{_{\rm{bcz}}}
\newcommand{\ov}{_{\rm{ov}}}
\newcommand{\rms}{_{\rm{rms}}}
\newcommand{\const}{_{\rm{const}}}

\newcommand{\lmax}{{\ell_{\rm{max}}}}

% SOLAR AND STELLAR VARIABLES
\newcommand{\rsun}{R_\odot}
\newcommand{\rtach}{r_t}
\newcommand{\lsun}{L_\odot}
\newcommand{\omsun}{\Omega_\odot}

\newcommand{\msun}{M_\odot}
\newcommand{\rstar}{R_*}
\newcommand{\lstar}{L_*}
\newcommand{\mstar}{M_*}
\newcommand{\omstar}{\Omega_*}

\newcommand{\rearth}{R_\oplus}
\newcommand{\omearth}{\Omega_\oplus}
\newcommand{\mearth}{M_\oplus}

% TORQUE DEFINITIONS
\newcommand{\taurs}{\tau_{\rm{rs}}}
\newcommand{\taurad}{\tau_{\rm{rad}}}
\newcommand{\taums}{\tau_{\rm{ms}}}
\newcommand{\taumm}{\tau_{\rm{mm}}}
\newcommand{\taumc}{\tau_{\rm{mc}}}
\newcommand{\tauv}{\tau_{\rm{v}}}
\newcommand{\taumag}{\tau_{\rm{mag}}}

% stellar time scales
\newcommand{\pes}{{P_{\rm{ES}}}}
\newcommand{\pessun}{{P_{ {\rm ES}, \odot}}}
\newcommand{\pnu}{{P_{\nu}}}
\newcommand{\pkappa}{{P_{\kappa}}}
\newcommand{\peta}{{P_{\eta}}}
\newcommand{\prot}{{P_{\rm{rot}}}}
\newcommand{\pequil}{{P_{\rm{eq}}}}
\newcommand{\tequil}{{t_{\rm{eq}}}}
\newcommand{\tmax}{{t_{\rm{max}}}}
\newcommand{\pcyc}{{P_{\rm{cyc}}}}
\newcommand{\omcyc}{{\omega_{\rm{cyc}}}}
\newcommand{\pcycm}{{P_{{\rm cyc}, m}}}
\newcommand{\omcycm}{{\omega_{{\rm{cyc}}, m}}}
% NON-DIMENSIONAL NUMBERS

% input non-d
\newcommand{\nrho}{N_\rho}

\newcommand{\ra}{{\rm{Ra}}}
\newcommand{\ramod}{\ra^*}
\newcommand{\raf}{\ra_{\rm{F}}}
\newcommand{\rafmod}{\raf^*}

\newcommand{\pr}{{\rm{Pr}}}
\newcommand{\prm}{{\rm{Pr_m}}}

\newcommand{\di}{{\rm{Di}}}

\newcommand{\ek}{{\rm{Ek}}}
\newcommand{\ta}{{\rm{Ta}}}

%\newcommand{\he}{{\rm{He}}}

\newcommand{\bu}{{\rm{Bu}}}
\newcommand{\bumod}{{\rm{Bu^*}}}
\newcommand{\bvisc}{{\rm{Bu_{visc}}}}
\newcommand{\brot}{{\rm{Bu_{rot}}}}

% output non-d
\newcommand{\ro}{{\rm{Ro}}}
\newcommand{\lo}{{\rm{Lo}}}
\newcommand{\roc}{{\rm{Ro_c}}}

\newcommand{\re}{{\rm{Re}}}
\newcommand{\rem}{{\rm{Re_m}}}

% FLUX ALIASES
\newcommand{\flux}{{\bm{\mathcal{F}}}}
\newcommand{\fcond}{\flux_{\rm{cond}}}
\newcommand{\frad}{\flux_{\rm{rad}}}
\newcommand{\fluxscalar}{{\mathcal{F}}}
\newcommand{\fcondscalar}{\fluxscalar_{\rm{cond}}}
\newcommand{\fenthscalar}{\fluxscalar_{\rm{enth}}}

% UNITS
\newcommand{\gram}{{\rm{g}}}
\newcommand{\cm}{{\rm{cm}}}
\newcommand{\second}{{\rm{s}}}
\newcommand{\gauss}{{\rm{G}}}
\newcommand{\kelv}{{\rm{K}}}
\newcommand{\unitent}{{\rm{erg\ g^{-1}\ K^{-1}}}}
\newcommand{\uniten}{\rm{erg}\ \cm^{-3}}
\newcommand{\unitprs}{\rm{dyn}\ \cm^{-2}}
\newcommand{\unitrho}{\gram\ \cm^{-3}}
\newcommand{\stoke}{\rm{cm^2\ s^{-1}}}

% MEAN FIELD THEORY
\newcommand{\meanb}{\overline{\bm{B}}}
\newcommand{\flucb}{\bm{B}^\prime}
\newcommand{\totb}{\bm{B}}

\newcommand{\meanv}{\overline{\bm{v}}}
\newcommand{\flucv}{\bm{v}^\prime}
\newcommand{\totv}{\bm{v}}

\newcommand{\emf}{\bm{\mathcal{E}}}
\newcommand{\meanemf}{\overline{\bm{\mathcal{E}}}}
\newcommand{\meanbpol}{\overline{\bm{B}_{\rm{pol}}}}

% SIMULATION CODES
\newcommand{\rayleigh}{\texttt{Rayleigh}}
\newcommand{\rayleigha}{\texttt{Rayleigh 0.9.1}}
\newcommand{\rayleighb}{\texttt{Rayleigh 1.0.1}}

\newcommand{\eulag}{\texttt{EULAG}}
\newcommand{\eulagmhd}{\texttt{EULAG-MHD}}
\newcommand{\ash}{\texttt{ASH}}
\newcommand{\rsst}{\texttt{RSST}}
\newcommand{\rtdt}{\texttt{R2D2}}
\newcommand{\pencil}{\texttt{Pencil}}

% other macros
\newcommand{\vecrot}{\bm{\Omega}}
\newcommand{\vecr}{\bm{r}}
\newcommand{\xpar}{x_\parallel}
\newcommand{\epar}{\e_\parallel}
\newcommand{\upar}{u_\parallel}
\newcommand{\uperp}{\vecu_\perp}
\newcommand{\vecf}{\bm{F}}
\newcommand{\veck}{\hat{\bm{k}}}

\newcommand{\uover}{\overline{\vecu}}

% background atmosphere values
\newcommand{\csa}{c_{{\rm s}a}}
\newcommand{\cpa}{C_{{\rm p}a}}

% total energies and energy densities
\newcommand{\etot}{E_{\rm tot}}
\newcommand{\anetot}{E_{\rm tot}^{\rm an}}
\newcommand{\eggetot}{E_{\rm tot}^{\rm EGG}}

\newcommand{\wtot}{W_{\rm tot}}
\newcommand{\wke}{W_{\rm KE}}
\newcommand{\wnke}{W_{\rm NKE}}
\newcommand{\wint}{W_{\rm int}}
\newcommand{\wpot}{W_{\rm pot}}

\newcommand{\wptot}{W^\prime_{\rm tot}}
\newcommand{\wpke}{W^\prime_{\rm KE}}
\newcommand{\wpnke}{W^\prime_{\rm NKE}}
\newcommand{\wpheat}{W^\prime_{\rm heat}}
\newcommand{\wpcomp}{W^\prime_{\rm comp}}
\newcommand{\wppot}{W^\prime_{\rm pot}}
\newcommand{\wpelast}{W^\prime_{\rm elast}}

\newcommand{\anwtot}{W^{\rm an}_{\rm tot}}
\newcommand{\anwke}{W^{\rm an}_{\rm KE}}
\newcommand{\anwheat}{W^{\rm an}_{\rm heat}}

\newcommand{\eggwtot}{W^{\rm EGG}_{\rm tot}}
\newcommand{\eggwke}{W^{\rm EGG}_{\rm KE}}
\newcommand{\eggwheat}{W^{\rm EGG}_{\rm heat}}

% date, author, title
\date{\today}
\author{Loren Matilsky}
\title{An energy-conserving anelastic approximation for strongly stably-stratified fluids}
\begin{document}
	\maketitle
	\section{Introduction}\label{sec:intro}
	Abstract: When acoustic oscillations are believed to be irrelevant to the dynamics of a fluid, it is useful to employ simplifying approximations to the equations of motion. The two most common of these (which are usually used to treat convection problems) are the Boussinesq approximation (when the background density does not significantly vary across the fluid layer) and the anelastic approximation (when the background density does vary significantly). There are many distinct forms of the anelastic approximation in the literature, and it has often been remarked that they do not properly conserve energy when the fluid is stable to convection. Here we show that the anelastic equations derived by Gough (1969) in fact do conserve energy for arbitrary motions of the fluid, even for strongly stratified background stratification. The key properties of these equations that allow them to conserve energy are (1) the absence of the Lantz-Braginsky-Roberts (LBR) approximation in the momentum equation and (2) the inclusion of a historically neglected term in the internal energy equation. These two properties allow the proper conversion between kinetic and internal energy at the correct order of the formal asymptotic expansion of the equations. We show that the scaling analysis of Gough (1969), which implicitly assumed a single typical value of the background entropy gradient, can be valid even for convective overshoot, where the entropy gradient changes from slightly unstable in the convecting region to stable (sometimes strongly so) in the overshoot region. The requirement for the anelastic equations to be valid for convective overshoot is that the buoyancy frequency be significantly less than the acoustic cutoff frequency. 
	\\
	%(i.e., $\Div(\rhoover\vecu)\equiv0$, where $\rhoover$ is the background density and $\vecu$ the fluid velocity; this takes the place of the $\Div\vecu=0$ condition from the Boussinesq approximation)
	
	The anelastic equations consist of an approximation to the continuity and momentum equations, originally derived by assuming small thermal perturbations about a nearly adiabatically stratified hydrostatic reference atmosphere \citep{Batchelor1953,Charney1960}. The thermodynamics of the problem thus become ``linear," in the sense that products of thermodynamic variables reduce to linear expressions in the first-order perturbations. The two key consequences of linearized thermodynamics are divergenceless mass flux and the first-order buoyancy force (associated with the first-order perturbed density and pressure) being the primary driver of the flow. \citet{Ogura1962} formalized the approximation by expanding the fluid equations in a small parameter $\epsilon$, representing the relative variation of potential temperature across the fluid layer, and hence the relative magnitude of the thermal perturbations. They recovered the equations of \citet{Batchelor1953} and \citet{Charney1960} and showed an assumption about the \textit{time scale} of the motion was necessary, in addition to the assumption of small thermal perturbations. Namely, the dynamical time scale of the buoyantly driven flows must be $O(\epsilon^{-1/2})$ times \textit{larger} than the sound crossing time of the region. Sound waves, which imply rapid temporal variations on the order of the sound crossing time, are thus absent from the anelastic equations, making them ideal for numerical integration, where large time steps are required to capture significant evolution of the system. 
	
	In the original asymptotic expansion of \citet{Ogura1962}, the internal energy equation was replaced by a heat (or entropy) equation for the evolution of potential temperature, \textit{before} non-dimensionalizing the equations. The approach of considering the entropy equation instead of the energy equation before nondimensionalization is repeated in all modern implementations of the anelastic approximation that we are aware of (e.g., \citealt{Gilman1981,Lipps1982,Glatzmaier1984,Lantz1992,Braginsky1995,Lantz1999,Clune1999,Rogers2005,Brown2012,Vasil2013,Wilczyski2022}). The resulting energy equation is also used in all numerical codes we are aware of that utilize the anelastic equations, for example, the {\ash} code \citep{Brun2004}, the \texttt{MagIC} code \citep{Gastine2012}, the {\rayleigh} code \citep{Featherstone2016a,Featherstone2023}, the {\eulag} code \citep{Smolarkiewicz2004}, and the \texttt{Dedalus} code \citep{Burns2020,Brown2020}. 
	
	While nondimensionalizing the entropy equation instead of the internal energy equation may at first appear to be an arbitrary (and harmless) choice, we show in the present work that it leads to an asymptotically inconsistent set of equations that do not conserve energy when the background is stably stratified. \citet{Gough1969}, by contrast, took a different approach than \citet{Ogura1962} and performed a formal asymptotic expansion in $\epsilon$ after nondimensionalizing the internal energy equation. We show that this equation set, which we dub the ``Energy-conserving Generalized Gough" (AnEGG) anelastic equations, conserve energy for arbitrary fluid motions and for all hydrostatic background states (whether stably or unstably stratified).
	
	\section{Energy conservation under the fully compressible fluid equations}\label{sec:fullycompressible}
	We begin by writing down the unapproximated fully compressible equations of motion for a nonrotating nonmagnetic fluid. These are \citet{Gough1969}'s Equations (2.1)--(2.5), consisting of the continuity equation 
	\begin{align}\label{eq:cont}
		\pderiv{\rho}{t}=-\Div(\rho\vecu) 
	\end{align}
	the momentum equation,
	\begin{subequations}\label{eq:mom}
	\begin{align}
		\pderiv{}{t}(\rho\vecu)&=-\Div(\rho\vecu\vecu) - \nabla P +\rho\vecg +\Div \overleftrightarrow{D},\label{eq:mommain}\\
		\where D_{ij}& = \mu\left[\pderiv{u_i}{x_j} + \pderiv{u_j}{x_i} - \frac{2}{3}(\Div\vecu)\delta_{ij}\right],\label{eq:momvisc}
	\end{align}
	\end{subequations}
	the internal energy equation,
	\begin{align}\label{eq:en}
		\pderiv{}{t}(\rho U) + \Div(\rho U\vecu) + P\Div\vecu = D_{ij}\pderiv{u_i}{x_j} + Q - \Div\vecf,
	\end{align}
	and a general equation of state,
	\begin{align}\label{eq:eos}
		U = U(P,T).
	\end{align}
	Here, $t$ is the time, the $x_i$ are Cartesian spatial coordinates, $\rho$ is the density, $P$ the pressure, $T$ the temperature, $U$ the internal energy per unit mass, $\mu$ the dynamic viscosity, $\vecg\define -\nabla\Phi$ the gravitational acceleration field, $\Phi$ the gravitational potential, $Q$ an internal heat source, and $\vecf$ the combined conductive and radiative heat flux. The subscripts $i$ and $j$ (taking on the values 1, 2, and 3) denote vector or tensor components in any of the Cartesian coordinate directions. The gravity $\vecg$ is assumed to point in the vertical direction $\veck$ (either the upward Cartesian direction for a plane-parallel fluid layer or the radial direction for spherical shell). Additionally, $\vecg$ is assumed to depend only on the vertical coordinate $q$ (either the upward Cartesian coordinate $x_3$ or the radial coordinate $r$) and to be time-independent (meaning that, among other consequences, self-gravity is ignored). The symbol ``$\leftrightarrow$'' in the viscous stress tensor $\overleftrightarrow{D}$ denotes a second-order tensor, as does the dyadic notation $\vecu\vecu$.   We use the Einstein summation convention and $\delta_{ij}$ denotes the Kronecker delta. These equations are not written in the exact form of \citet{Gough1969} and use slightly different notation but are mathematically equivalent. 
	
	An equation for the evolution of kinetic energy can be formed from $\vecu$ dotted into Equation \eqref{eq:mom},
	\begin{align}\label{eq:ke}
		\pderiv{}{t}\left(\frac{1}{2}\rho u^2\right) = - \Div\left(\frac{1}{2}\rho u^2\vecu \right) + \ugrad P - \rho\ugrad\Phi+ u_i\pderiv{D_{ij}}{x_j}.
	\end{align}
	Equation \eqref{eq:cont} multiplied by $\Phi$ yields an equation for the evolution of potential energy,
	\begin{align}\label{eq:pote}
		\pderiv{}{t}(\rho\Phi) &= - \Phi \Div(\rho\vecu).
	\end{align}
	Adding Equations \eqref{eq:en}, \eqref{eq:ke}, and \eqref{eq:pote} yields an equation for the evolution of total energy,
	\begin{align}\label{eq:tote}
		\pderiv{}{t}\left[\rho\left(\frac{1}{2} u^2 + U + \Phi\right)\right] = -\Div\left\{\left[\rho\left(\frac{1}{2} u^2 + U + \Phi\right) + P\right]\vecu- \vecu\cdot\overleftrightarrow{D} + \vecf\right\} + Q.
	\end{align}
	Integrating Equation \eqref{eq:tote} over the volume $V$ of the fluid layer, using the divergence theorem, and assuming that the sum of the surface-integrated fluxes balances the volume-integrated heating yields\footnote{Throughout this work, we use the symbol ``$\define$" for definitions (i.e., for ``is defined to be equal to") and ``$\equiv$" for homogeneous equivalence (i.e., for ``is equal to [some constant value] everywhere and for all time").}
	\begin{subequations}\label{eq:econst}
	\begin{align}
		\etot &\define \int_V \wtot dV \equiv \text{constant},\\
		\where \wtot &\define \wke + \wint + \wpot,\label{eq:wtot}\\
		\wke &\define \frac{1}{2}\rho u^2,\\
		\wint &\define \rho U,\\
		\andd \wpot &\define \rho \Phi. 
	\end{align}
	\end{subequations}
	
	We additionally define
	\begin{align}\label{eq:wnke}
		\wnke &\define \wtot - \wke = \wint + \wpot 
	\end{align}
	as the combined internal and potential energy density (i.e., the ``nonkinetic" energy). 
	
	Equation \eqref{eq:econst} expresses the conservation of total energy that is implicit to the fully compressible fluid equations \eqref{eq:cont}--\eqref{eq:eos}. The volume-integrated total energy is conserved, as long as the total surface fluxes balance the total internal heat sources. Locally by contrast, fluid parcels can exchange total energy density with one another, and these exchanges are further partitioned into kinetic, internal, and potential energy densities. 
	
	One of the main goals of this work is to elucidate how Equation \eqref{eq:econst} is modified under the anelastic approximation for a general equation of state and nonadiabatic stratification. It will be found that \textit{only} exchange between internal energy (due to heating) and kinetic energy is possible for an energetically constistent anelastic approximation. Conversion to and from potential energy, by contrast, is fundamentally impossible. This is essentially a new result, which has been overlooked for several decades, mainly because of the pathological coincidence that for a perfect, adiabatically stratified gas, the internal energy due to heating \textit{looks} exactly like a potential energy.\footnote{We use the nomenclature that an ``ideal gas" refers to a fluid for which the internal energy and specific enthalpy depend only on the temperature (i.e., $U=U(T)$ and $h=h(T)$). It can be shown from Equations \eqref{eq:firstlaw} these two conditions yield the ideal gas law, $P=\mathcal{R}\rho T$, where $\mathcal{R}\define\cpcap-\cvcap$ and the individual specific heats depend only on temperature. A``perfect gas" is a gas that is not only ideal, but has specific heats that are constants, independent of temperature.} For non-adiabatic stratification, or for an imperfect gas, the coincidence no longer holds, and the more general law for conservation of total energy that is derived here should be considered. 
	
The other new and related result is that total energy \textit{should} be conserved under the anelastic approximation, for arbitrary fluid motions and equations of state and for all stratifications, even for strongly subadiabatic ones. The anelastic equations of \citet{Gough1969} satisfy such a general conservation law, while modern implementations of the anelastic approximation do not. Thus, a major part of this paper is devoted to translating \citet{Gough1969}'s equations into more conventional notation and elucidating the origin and form of the term (which should be present in the internal energy equation for asymptotic consistency) that is missing from modern anelastic implementations. 

Finally, we note that \citet{Gough1969} assumed zero net vertical transport of mass for the ``true" fully compressible fluid whose motion the anelastic equations are intended to approximate. This may at first seem like an arbitrary choice to make the small-parameter expansion simpler. It is concluded in this work, however, that the ``zero mean mass flux" assumption is in fact fundamental to ensure the energetic consistency of the anelastic approximation. 

	\section{The anelastic approximation of \citet{Gough1969}}
	To set the stage, we define some new thermodynamic variables, namely, the specific enthalpy,
\begin{align}\label{eq:enthalpy}
	h\define U + \frac{P}{\rho}
\end{align}
and the specific entropy,
\begin{align}\label{eq:eosentr}
	S = S(P,T).
\end{align}
Equation \eqref{eq:eosentr} is also an equation of state, containing the equivalent information of Equation \eqref{eq:eos}. 

We define several first derivatives of thermodynamic quantities derivable from the generalized equation of state \eqref{eq:eos} or \eqref{eq:eosentr}: the specific heat at constant volume,
\begin{align}\label{eq:cv}
	\cvcap=\cvcap(P,T)\define T\left(\pderiv{S}{T}\right)_\rho,
\end{align}
the specific heat at constant pressure, 
\begin{align}\label{eq:cp}
	\cpcap=\cpcap(P,T)\define T\left(\pderiv{S}{T}\right)_P,
\end{align}
the squared adiabatic sound speed,
\begin{align}\label{eq:cs2}
	\cs^2=\cs^2(P,T)\define \left(\pderiv{P}{\rho}\right)_S,
\end{align}
the (isobaric) thermal expansion coefficient,
\begin{align}\label{eq:delta}
	\delta=\delta(P,T)\define-\left(\pderiv{\ln\rho}{\ln T}\right)_P,
\end{align}
and the (isochoric) thermal pressure coefficient, 
\begin{align}\label{eq:eta}
	\eta=\eta(P,T)\define\left(\pderiv{\ln P}{\ln T}\right)_\rho.
\end{align}
The first law of thermodynamics\footnote{We assume local thermodynamic equilibrium holds for the fluid.} then takes the following forms:
\begin{subequations}\label{eq:firstlaw}
	\begin{align}
		TdS &= dU - \frac{P}{\rho^2}d\rho \label{eq:firstlawinterho} \\
		&= dh - \frac{dP}{\rho}\label{eq:firstlawenthprs}\\
		&= \cvcap dT - \frac{P\eta}{\rho^2}d\rho\label{eq:firstlawtmprho}\\		
		&= \cpcap dT - \frac{\delta}{\rho}dP\label{eq:firstlawtmpprs}\\
		&=\frac{\cpcap T}{\rho\delta}\left[\frac{dP}{\cs^2}-d\rho\right]\label{eq:firstlawprsrho}.
	\end{align}
\end{subequations}

	From Equations \eqref{eq:cont} and \eqref{eq:firstlaw}, the left-hand side of Equation \eqref{eq:en} can be written in several equivalent forms:
	\begin{subequations}\label{eq:enlhs}
		\begin{align}
			\pderiv{}{t}(\rho U) + \Div(\rho U\vecu) + P\Div\vecu &= \rho\matderiv{U} - \frac{P}{\rho}\matderiv{\rho}\\
			&= \rho\matderiv{h} -\matderiv{P}\\
			&=\rho T \matderiv{S},
		\end{align}
	\end{subequations}
	where 
	\begin{align}\label{eq:matderiv}
		\matderiv{}{}\define \pderiv{}{t} + \ugrad
	\end{align}
	is the material (or Lagrangian) derivative,
	
	\subsection{\citet{Gough1969}'s anelastic scale analysis for convection}
	We will not repeat the full asymptotic expansion in $\epsilon$ performed by \citet{Gough1969} of Equations \eqref{eq:cont}, \eqref{eq:mom}, \eqref{eq:en}, and \eqref{eq:eos} here. Instead, we reiterate the salient assumptions in the case where the layer depth is thicker than the typical pressure scale height.\footnote{\citet{Gough1969} also considers thin layers, in which the anelastic equations become the Boussinesq equations.} The main assumption is that the thermodynamic perturbations from the horizontally averaged background state are small,
	\begin{align}\label{eq:smallthermalpert}
		f = \overline{f}(q,t) + f_1(x_i,t)\five &\with f_1/\overline{f}=O(\epsilon)\ll 1,
	\end{align} 
	where $f$ denotes either $\rho$, $P$, $T$, $U$, $h$, $\cpcap$, $\mu$, $\cs^2$, $\delta$, $\eta$, $\vecf$, or $Q$. Here, the overbars denote horizontal averages (taken at a particular instant in time) and the ``1" subscripts denote the perturbations about this average. Note that it is \textit{not} correct to write ``$S=\entrover(q,t)+\entrone(x_i,t)$ with $\entrone/\entrover=O(\epsilon)$." The fully compressible equations of motion contain only differences in entropy and so no meaningful absolute value of $\entrover$ can be defined. Instead, we must write
	\begin{align}\label{eq:linearent}
	S = \entrover(q,t) + \entrone(x_i,t) \five &\with \entrone/\cpover=O(\epsilon)\ll 1.
	\end{align}
	
	The second central assumption is that the coordinate system can be chosen such that there is no mass flux across any horizontal surface, i.e.,
	\begin{align}\label{eq:nomassflux}
		\overline{\rho u_i}\equiv0.
	\end{align}
	In a spherical system (e.g., a spherical shell), the horizontal average would be a spherically symmetric average and the coordinates would point along the spatially varying curvilinear coordinate directions. 

	In writing $\overline{f}=\overline{f}(q,t)$ in Equation \eqref{eq:smallthermalpert}, we make a third central assumption not made by \citet{Gough1969}, namely, that the mean state may be time-dependent.\footnote{\citet{Gough1969} also considered time-dependent atmospheres, but this was in the different context of pulsations, where the atmosphere itself was moving. In the present work, we consider ``Eulerian-only" time-dependence of the mean state.} In order to still be consistent with \citet{Gough1969}'s asymptotic expansion, we must assume that the time-variation of the mean state is sufficiently slow, i.e., comparable in magnitude to the time variation of the fluctuations:
\begin{align}\label{eq:meanvariationslow}
	\left|\pderiv{\overline{f}}{t}\right| = O\left(\left|\pderiv{f_1}{t}\right|\right).
\end{align}
%= O\left(|f_1|\sqrt{\frac{\epsilon g_a}{H_a}}\right)
%By writing, e.g., $\rhoover=\rhoover(q)$, \citet{Gough1969} implicitly assumes that the horizontally averaged state is time-independent. We keep this assumption for now but relax it later, since the asymptotic expansion is unaffected by temporally varying background states, as long as the evolution is slow enough. 
% This can also be interpreted to mean that the convection has established statistically steady mean profiles in the thermal variables, and the goal of the anelastic approximation is to solve for small thermal perturbations about these mean profiles. The mean state is thus unknown a priori, but can be solved for after statistically steady convection has been established. 

	The rest of the assumptions of \citet{Gough1969} concern scale analysis, which is appropriate for convection in the presence of mean density stratification $\pderivline{\rhoover}{q}\neq0$. The characteristic length scale of variation of the fluid is assumed to be a typical value $H_a$ for the pressure scale height. The flow is assumed to be buoyantly driven by the $O(\epsilon)$ thermal perturbations, i.e., 
	\begin{align}\label{eq:scaleu}
		|\vecu| = O(\sqrt{\epsilon g_a H_a}) = O(\sqrt{\epsilon}\csa),
	\end{align}
	where ``a" subscripts denote typical atmospheric background-state values. Thus, the squared Mach number of the flow is assumed to be $O(\epsilon)$. The characteristic time scale of variation of the fluid is assumed to be advective, i.e., 
	\begin{align}\label{eq:scaleu}
	\left|\pderiv{}{t}\right|  = O\left(\sqrt{\frac{\epsilon g_a}{H_a}}\right) = O\left(\sqrt{\epsilon} \frac{1}{H_a/\csa}\right).
	\end{align}
	Thus, the characteristic time scale of variation for the convection is $O(\epsilon^{-1/2})$ times longer than the time it takes a sound wave to cross a pressure scale height. From Equation \eqref{eq:meanvariationslow}, the characteristic time scale of variation for the mean state is $O(\epsilon^{-3/2})$ times longer than the sound crossing time. 
	
	Finally, the vertical convective heat flux, which maximally could transport an energy flux of order $\rhoover \tmpover w \Delta\entrover$, where 
	\begin{align}\label{eq:defw}
		w\define\veck\cdot\vecu
	\end{align} is the vertical velocity and $\Delta \entrover $ is the total drop in background entropy across the convecting layer, is assumed to be limited primarily by the thermal diffusion $\vecf$. This will be true if the conductive heating $-\Div \vecf$ in Equation \eqref{eq:en} is at at least as large as the viscous and internal heatings. In the case of negligible heatings (high Rayleigh number), one expects
	\begin{align}
		\frac{|\Delta\entrover|}{\cpa} = O(\epsilon),
	\end{align}
	i.e., the convecting layer should be nearly adiabatically stratified for vigorous convection. 
	
	One consequence of Equation \eqref{eq:nomassflux} is that the horizontally averaged velocity $\overline{\vecu}$ is $O(\epsilon)$ \textit{smaller} than the perturbed velocity $\vecu_1$. To see this, we write
	\begin{align}
		0 &= \overline{(\rhoover+\rhoone)(\overline{\vecu}+\vecu_1)} = \rhoover\,\uover + \overline{\rhoone\vecu_1},\nonumber\\
		\orr |\uover| &= \left|-\frac{\overline{\rhoone\vecu_1}}{\rhoover}\right| = O(\epsilon|\vecu_1|).
	\end{align}
	For the total mass flux (or equivalently, momentum density), $\bm{m}\define \rho\vecu=(\rhoover+\rhoone)(\overline{\vecu}+\vecu_1)$, we can thus write
		\begin{align}\label{eq:momdensityorig}
		\bm{m} = \bm{m}-\overline{\bm{m}}=\rhoover\vecu_1 + \rhoone\vecu_1 - \overline{\rhoone\vecu_1} + O(\epsilon^2).
	\end{align}
	Hence, at $O(\epsilon)$, only the perturbation velocity $\vecu_1$ appears in the equations, so we subsequently use $\vecu$ as shorthand for $\vecu_1$ and drop the subscript ``1." Under this convention,
\begin{align}\label{eq:nomeanu}
	\overline{\vecu}\equiv0
\end{align}
	and Equation \eqref{eq:momdensityorig} becomes
	\begin{align}\label{eq:momdensity}
		\bm{m} = \rhoover\vecu + \rhoone\vecu - \overline{\rhoone\vecu} + O(\epsilon^2).
	\end{align}
	Each of the two terms $\rhoone\vecu$ and $-\overline{\rhoone\vecu}$ are $O(\epsilon)$. In most cases, we can thus write $\bm{m} \approx \rhoover \vecu$ to translate from \citet{Gough1969} to the current notation (in which we use $\vecu$ as the primary field variable), except when multiplying by potentially zeroth-order quantities.
	
	 One other change in notation is that \citet{Gough1969} uses the superadiabatic mean background temperature gradient
	\begin{align}\label{eq:superad}
		\beta &\define -\frac{1}{\cpover}\veck\cdot\left[\nabla\overline{h}-\frac{1}{\rhoover}\nabla\prsover\right]\nonumber\\
		&=-\frac{\tmpover}{\cpover}\veck\dotgrad\entrover + O(\epsilon^2),
	\end{align}
	whereas we will use $\nabla \entrover$. 
	
	\subsection{\citet{Gough1969}'s anelastic equations}
	Once all of the above scaling assumptions have been made, Equations \eqref{eq:cont}, \eqref{eq:mom}, \eqref{eq:en}, and \eqref{eq:eos} are nondimensionalized, each term is expanded in powers of $\epsilon$, terms up to zeroth-order in the continuity equation and first-order in the other equations are retained, and redimensionalization then yields the anelastic equations. Specifically, we discuss \citet{Gough1969}'s Equations (4.3)--(4.7) and (4.15)--(4.22). We translate these equations using the change of variables outlined in Equations \eqref{eq:momdensity} and \eqref{eq:superad}. 
	
	Under \citet{Gough1969}'s anelastic approximation (with the additional assumption \eqref{eq:meanvariationslow}), the continuity equation \eqref{eq:cont} becomes
	\begin{align}\label{eq:ancont}
		\Div(\rhoover\vecu)\equiv 0,
	\end{align}
	the momentum equation \eqref{eq:mom} becomes 
	\begin{subequations}\label{eq:anmom}
	\begin{align}
		\pderiv{}{t}(\rhoover\vecu)&=-\Div(\rhoover\vecu\vecu) - \nabla \prsone +\rhoone\vecg +\Div \overleftrightarrow{D}+[-\nabla\prsover + \rhoover\vecg],\label{eq:anmommain}\\
		\text{where now}\five D_{ij}& = \overline{\mu}\left(\pderiv{u_i}{x_j} + \pderiv{u_j}{x_i} - \frac{2}{3}(\Div\vecu)\delta_{ij}\right)\label{eq:anmomvisc},
	\end{align}
	\end{subequations}
	the energy equation \eqref{eq:en} becomes 
	\begin{align}\label{eq:anen}
		\rhoover\cpover\left(\pderiv{\tmpone}{t}+\pderiv{\tmpover}{t}\right)- \deltaover\left(\pderiv{\prsone}{t} + \pderiv{\prsover}{t}\right) = &-\rhoover\vecu\cdot\left(\nabla \enthone-\frac{1}{\rhoover}\nabla \prsone\right) - \rhoover \tmpover \ugrad \entrover\nonumber\\
		& D_{ij}\pderiv{u_i}{x_j} + Q _1- \Div\vecf_1 -\rhoone\vecu\cdot\vecg - \tmpover(\rhoone\vecu-\overline{\rhoone\vecu})\cdot\nabla\entrover\nonumber\\
		&+ [\heatover - \Div\overline{\vecf}],
	\end{align}
	%-\nabla(\rhoover \overline{w^2}) 
  and the equation of state \eqref{eq:eos} or \eqref{eq:eosentr} becomes linearized via the first law of thermodynamics \eqref{eq:firstlaw}, with
  \begin{subequations}\label{eq:anfirstlawpert}
  \begin{align}
  	\tmpover \entrone &= \cpover \tmpone - \frac{\deltaover}{\rhoover} \prsone\label{eq:anfirstlawperttmpprs}\\
  	&= U_1 - \frac{\prsover}{\rhoover^2}\rhoone \label{eq:anfirstlawpertinterho}\\
  	&= \enthone - \frac{\prsone}{\rhoover}\label{eq:anfirstlawwpthprs}\\
  	&= \cvover \tmpone - \frac{\prsover\etaover}{\rhoover^2}\rhoone\label{eq:anfirstlawperttmprho}\\		
  	&= \frac{\cpover\, \tmpover}{\deltaover\rhoover}\left[\frac{\prsone}{\cssqover}-\rhoone\right].\label{eq:anfirstlawpertprsrho}
  \end{align}
  \end{subequations}
  We have used Equation \eqref{eq:momdensity} to yield the term $- \tmpover(\rhoone\vecu-\overline{\rhoone\vecu})\cdot\nabla\entrover$ in Equation \eqref{eq:anen}. Note that the continuity equation \eqref{eq:ancont} is unchanged from \citet{Gough1969}'s Equation (4.16), despite $\pderivline{\rhoover}{t}$ being nonzero. This is because $\pderivline{\rhoover}{t}$ shows up only at $O(\epsilon)$ by Equation \eqref{eq:meanvariationslow}. Similarly, the momentum equation \eqref{eq:anmom} takes the same form regardless of whether the mean state is time-dependent, since we can replace $\pderivline{\bm{m}}{t}$ with $\pderivline{(\rhoover\vecu)}{t}$ by virtue of Equations \eqref{eq:meanvariationslow} and \eqref{eq:momdensity}. The left-hand side of the internal energy equation \eqref{eq:anen} and the right-hand side of the mean internal energy equation \eqref{eq:anenmean} below, where we must keep the time-variation of the mean state, is the only place our equations differ materially from those of \citet{Gough1969}, specifically, his Equations (4.17) and (4.7).
  
   The differentials in Equation \eqref{eq:firstlaw} can be converted into gradients (e.g., $T\nabla S= \nabla h - \nabla P/\rho$) and the horizontally averaged form of these relations yields 
  \begin{subequations}\label{eq:anfirstlawmean}
  	\begin{align}
  		\tmpover\nabla\entrover &= \cpover\nabla\tmpover - \frac{\deltaover}{\rhoover}\nabla\prsover + O(\epsilon^2)\\
  		&= \frac{\cpover\,\tmpover}{\deltaover\rhoover}\left[\frac{\nabla\prsover}{\cssqover}-\nabla\rhoover\right]+O(\epsilon^2).
  	\end{align}
  \end{subequations}
  
   The horizontal averages of Equations \eqref{eq:anmom} and \eqref{eq:anen} satisfy 
  \begin{align}\label{eq:anmommean}
  	-\nabla\prsover + \rhoover\vecg = \nabla(\rhoover \overline{w^2})
  \end{align}
  and
  \begin{align}\label{eq:anenmean}
  	\heatover - \Div\overline{\vecf} = \rhoover\overline{\vecu\cdot\left(\nabla \enthone-\frac{1}{\rhoover}\nabla \prsone\right)} + \vecg\cdot\overline{\rhoone\vecu} - \overline{D_{ij}\pderiv{u_i}{x_j}} +	\rhoover\cpover\pderiv{\tmpover}{t}- \deltaover \pderiv{\prsover}{t}
  \end{align}
   Equations \eqref{eq:ancont}--\eqref{eq:anenmean} are mathematically equivalent to \citet{Gough1969}'s Equations (4.3)--(4.7) and (4.15)--(4.22), with the minor exceptions of Equations \eqref{eq:anen} and \eqref{eq:anenmean}. Together, they form a complete system that can be solved for the evolution of both the fluctuating thermal variables and velocity, as well as the mean thermal profiles. %Strictly, however, the approximation is only consistent if the mean state does not evolve. We show that this restriction can be relaxed slightly in Section \ref{sec:meantoref}. 

  
Dotting $\vecu$ into Equation \eqref{eq:anmommain} and using Equation \eqref{eq:anmommean} yields the anelastic kinetic energy equation,
	\begin{align}\label{eq:anke}
		\pderiv{}{t}\left(\frac{1}{2}\rhoover u^2\right)&=-\Div\left(\frac{1}{2}\rhoover u^2\vecu \right) - \ugrad \prsone + \rhoone\vecu\cdot\vecg + u_i\pderiv{D_{ij}}{x_j} + \ugrad(\rhoover \overline{w^2}).
	\end{align}

Converting the differentials in Equation \eqref{eq:firstlaw} into Eulerian differentials in time, the left-hand side of Equation \eqref{eq:anen} can be written in terms of the total entropy $\entrone + \entrover$, 
\begin{align}\label{eq:anenlhs}
	\rhoover\cpover\left(\pderiv{\tmpone}{t}+\pderiv{\tmpover}{t}\right)- \deltaover\left(\pderiv{\prsone}{t} + \pderiv{\prsover}{t}\right) = \rhoover\tmpover\pderiv{\entrone}{t} +  \rhoover\tmpover\pderiv{\entrover}{t}.
\end{align}

Using Equations \eqref{eq:ancont}, \eqref{eq:h1}, and \eqref{eq:anenlhs} and then adding Equations \eqref{eq:anen} and \eqref{eq:anke} yields \citet{Gough1969}'s total energy equation,
\begin{align}\label{eq:antote}
			\pderiv{}{t}\left\{\rhoover\left[\frac{1}{2} u^2 + \tmpover (\entrone+\entrover)\right]\right\} = &-\Div\left\{\left[\rhoover\left(\frac{1}{2} u^2 + \tmpover \entrone\right) + \prsone\right]\vecu- \vecu\cdot\overleftrightarrow{D} + \overline{\vecf} + \vecf_1\right\}+ \heatover + Q_1 \nonumber\\
			 &+\left[-\rhoover\tmpover\ugrad\entrover +\ugrad (\rhoover \overline{w^2})- \tmpover(\rhoone\vecu-\overline{\rhoone\vecu})\cdot\nabla\entrover\right].
\end{align}
Because of Equation \eqref{eq:nomeanu}, or more specifically, the condition
\begin{align}\label{eq:nomeanw}
	\overline{w}\equiv0,
\end{align}
 each of the rightmost terms in brackets in Equation \eqref{eq:antote} vanishes after volume integration over $V$ (i.e., cannot transport any net energy). Note that Equation \eqref{eq:nomeanw} is also a consequence of integrating Equation \eqref{eq:ancont} over volumes bounded by horizontal surfaces (and also assuming $w$ vanishes on the boundaries). Thus, the vanishing of the horizontal components of $\overline{\vecu}$ is not strictly necessary for the bracketed terms to conserve energy. 
 
 Integrating Equation \eqref{eq:antote} over $V$, using the divergence theorem, and assuming that the sum of the surface-integrated fluxes balances the volume-integrated heating yields
 \begin{subequations}\label{eq:aneconst}
 	\begin{align}
 		\anetot &\define \int_V \anwtot dV \equiv \text{constant},\label{eq:anetot}\\
 		\where \anwtot &\define \anwke + \anwheat,\label{eq:anwtot}\\
 		\anwke &\define \frac{1}{2}\rhoover u^2,\label{eq:anwke}\\
 		\andd \anwheat &\define \rhoover \tmpover (\entrone+\entrover).\label{eq:anwnke}
 	\end{align}
 \end{subequations}
 We have used the superscript ``an" to denote energies and densities under the anelastic approximation and defined $\anwheat$ to be internal energy density of fluid parcels associated with irreversible heating processes (entropy increases). The conservation law \eqref{eq:aneconst} holds for arbitrary fluid motions that obey Equations \eqref{eq:ancont}, \eqref{eq:anmom}, \eqref{eq:anen}, and \eqref{eq:anfirstlawpert} and for arbitrary magnitudes of $|\nabla \entrover|$. Note that the partition of energy density is fundamentally different than in the fully compressible conservation law \eqref{eq:econst}; most notably, the potential energy density is absent under the anelastic approximation. We return to this central point in Section \ref{sec:meaningtote}. %However, whether the approximation remains \textit{consistent} (i.e., whether the thermal perturbations remain small) \textit{does} depend on the magnitude of $|\nabla\entrover|$.

\section{\citet{Gough1969}'s anelastic equations in terms of pressure and entropy}\label{sec:goughps}
The anelastic equations are often written using the pressure and entropy ($\prsone$, $\entrone$, and $\entrover$) in place of the quantities $\tmpone$, $\tmpover$, $\prsone$, $\prsover$, $\rhoone$, and $\enthone$ that appear in Equations \eqref{eq:anmom} and \eqref{eq:anmommean}. This is more than just a matter of convention, since transforming the variables to pressure and entropy isolates terms in the equations that depend on the mean entropy gradient $\nabla\entrover$, which can be neglected if $\nabla\entrover$ is small.

  In each of Equations \eqref{eq:anmommean} and \eqref{eq:anenmean}, each term on the right-hand side is $O(\epsilon)$ compared to each term on the left-hand side. In particular, we can make the mean hydrostatic approximation, 
\begin{align}\label{eq:anhydrostatic}
	\nabla\prsover\approx \rhoover\vecg,
\end{align}
when multiplying by terms that are already of first order in $\epsilon$. 

We can change variables using Equations \eqref{eq:anfirstlawpert} and \eqref{eq:anfirstlawmean} and the approximation \eqref{eq:anhydrostatic}. In the momentum equation, we find
\begin{align}\label{eq:anrhsmom}
	-\nabla \prsone + \rhoone\vecg = -\rhoover\nabla\left(\frac{\prsone}{\rhoover}\right) - \deltaover \rhoover \left(\frac{\entrone}{\cpover}\right) \vecg + \frac{\deltaover\prsone}{\cpover}\nabla\entrover.
\end{align}
In the internal energy equation, we find
\begin{align}\label{eq:anrhsen}
	-\rhoover\vecu\cdot\left(\nabla \enthone-\frac{1}{\rhoover}\nabla \prsone\right) = -\rhoover\tmpover\ugrad \entrone - \rhoover \entrone\ugrad\tmpover + \left(\frac{\prsone}{\rhoover}\right)\ugrad \rhoover.
\end{align}
We then compute, with effort,
\begin{align}\label{eq:anrhsen2}
	- \rhoover \entrone\ugrad\tmpover + \left(\frac{\prsone}{\rhoover}\right)\ugrad \entrone -\rhoone\vecu\cdot\vecg = -\rhoover \tmpone\ugrad\entrover + O(\epsilon^2).
\end{align}

Substituting Equation \eqref{eq:anrhsmom} into Equation \eqref{eq:anmom}, and Equations \eqref{eq:anenlhs}, \eqref{eq:anrhsen}, and \eqref{eq:anrhsen2} into Equation \eqref{eq:anen}, we find
	\begin{align}\label{eq:anmommod}
		\pderiv{}{t}(\rhoover\vecu)= &-\Div(\rhoover\vecu\vecu) -\rhoover\nabla\left(\frac{\prsone}{\rhoover}\right) - \deltaover \rhoover \left(\frac{\entrone}{\cpover}\right) \vecg + \underbrace{\frac{\deltaover \prsone}{\cpover} \nabla\entrover}_{\define \bm{f}_{\rm NLBR}}+\Div \overleftrightarrow{D}\nonumber\\
		&+[-\nabla\prsover + \rhoover\vecg],
	\end{align}
and
\begin{subequations}\label{eq:anenmod}
\begin{align}
	\rhoover\tmpover\left(\pderiv{\entrone}{t}+\pderiv{\entrover}{t}\right)= &-\rhoover\tmpover\ugrad \entrone - \rhoover \tmpover \ugrad \entrover \underbrace{- \rhoover \tmpone\ugrad\entrover}_{\define Q_{\rm NLBR}}+D_{ij}\pderiv{u_i}{x_j} + Q _1- \Div\vecf_1  \nonumber\\
	& +[\heatover - \Div\overline{\vecf} - \tmpover(\rhoone\vecu-\overline{\rhoone\vecu})\cdot\nabla\entrover],\\
	\where \rhoone =& \frac{\prsone}{\cssqover}-\frac{\deltaover\rhoover \entrone}{\cpover}\\
	\andd \tmpone = & \frac{\tmpover \entrone}{\cpover} + \frac{\deltaover \prsone}{\rhoover\cpover}.
\end{align}
\end{subequations}
Finally, the mean internal energy equation \eqref{eq:anenmean} becomes
\begin{align}\label{eq:anenmean2}
	\heatover-\Div\overline{\vecf} = \rhoover\tmpover\, \overline{\ugrad \entrone}-\overline{D_{ij}\pderiv{u_i}{x_j}} + \rhoover (\overline{\tmpone\vecu})\dotgrad\entrover + \rhoover\tmpover\pderiv{\entrover}{t}.
\end{align}
%The final terms in brackets in these equations may be dropped without affecting energy conservation, since they do not transport any net energy. However, dropping them still may still makes the equations an asymptotically inconsistent approximation to the true fully compressible motion, and so we retain them.

The essential terms required for energy conservation when $\nabla\entrover\neq0$ are thus the ``non-LBR" force density
\begin{align}\label{eq:nlbrmom}
	\bm{f}_{\rm NLBR} \define \frac{\deltaover \prsone}{\cpover} \nabla\entrover
\end{align}
and the ``non-LBR" heating
\begin{align}\label{eq:nlbrheat}
	Q_{\rm NLBR} \define -\rhoover \tmpone\ugrad\entrover.
\end{align}
Both these terms vanish for a nearly adiabatic background state (where $\nabla\entrover=O(\epsilon)$), which is expected for a fully (and sufficiently vigorously) convecting fluid layer.  Neglecting $\bm{f}_{\rm NLBR}$ was first done independently by \citet{Lantz1992} and \citet{Braginsky1995} and is referred as the ``Lantz-Braginsky-Roberts" (LBR) approximation. The term $Q_{\rm NLBR}$ in the internal energy equation, which was implicitly contained in the equations of \citet{Gough1969}, seems to be absent in the other forms of the anelastic equations currently in use. Its neglect seems to have not been explicitly considered. 

\section{The partial equivalence between horizontally averaged atmospheres to fixed reference atmospheres}\label{sec:meantoref}
In the formalism of \citet{Gough1969}, the horizontally averaged atmosphere, denoted by the overbars, was originally assumed to be time-independent. We have relaxed that assumption here, but we must consider how the mean state depends on the flow via Equations \eqref{eq:anmommean} and \eqref{eq:anenmean2}. Many anelastic numerical codes (e.g., the {\rayleigh}, {\eulag}, and \texttt{MagIC} codes, which simulate the anelastic equations in spherical shells) treat the background state as a fixed-in-time, spherically symmetric, hydrostatic ``reference" state and let the thermal perturbations about this reference state develop small but nonzero horizontal means. Some codes (e.g., the {\ash} code) alternatively solve for both the perturbations about the horizontal average and the mean state itself, choosing some initial mean state consistent with zero flow.\footnote{This initial state, satisfying Equations \eqref{eq:anmommean}, \eqref{eq:anfirstlawmean}, and \eqref{eq:anenmean2} with $\vecu\equiv0$, would resemble a 1-D stellar structure model, for example.} In the latter approach, horizontally averaged terms like the bracketed terms in Equations \eqref{eq:anmom} and \eqref{eq:anen} are retained on the right-hand sides of the momentum and energy equations. As we now show, these two approaches are exactly equivalent to the order of the anelastic equations, provided we make a fourth central assumption: that the horizontal means of the thermal variables wander by no more than $O(\epsilon)$ away from their preordained reference-state values. 

We formalize this new assumption by defining the horizontally averaged profiles as the sum of the reference-state profile (denoted by a tilde) and a horizontally symmetric deviation (denoted by a hat):
\begin{align}\label{eq:meanvariationsmall}
\overline{f}=\overline{f}(q,t) &= \tilde{f}(q) + \hat{f}(q,t)\five\with \hat{f}/\tilde{f} = O(\epsilon),
\end{align}
where again $f$ denotes either $\rho$, $P$, $T$, $U$, $h$, $\cpcap$, $\mu$, $\delta$, $\cs^2$, $\vecf$, or $Q$. In physical terms, Equation \eqref{eq:meanvariationslow} requires that the time variation of the mean state is slow (e.g., altered by the convection on an advective time scale), while Equation \eqref{eq:meanvariationsmall} additionally requires that the total variation of the mean state away from the reference state (integrated over all time) remains small. 

%Note that the horizontal mean profiles are now time-dependent. This does not affect the asymptotic expansion leading to Equations \eqref{eq:ancont}, \eqref{eq:anmommod}, and \eqref{eq:anenmod}, provided the time derivatives of the mean state are the same order as the time derivatives of the deviations [e.g., $\pderivline{\rhoover}{t}=O(\pderivline{\rhoone}{t})$]. 
We denote the (temporally and horizontally dependent) deviations from the reference state by primes and note that 
\begin{align}\label{eq:defprime}
	f^\prime & \define f - \tilde{f} =\hat{f} +  f_1,
\end{align}
The primed quantities, being the sum of two assumed-small quantities, thus remain small as long Equations \eqref{eq:meanvariationslow} and \eqref{eq:meanvariationsmall} are both satisfied. 
%	&= h^\prime - \frac{\prsprime }{\rhotilde}\\
% 		&= U^\prime - \frac{\tilde{P}}{\rhotilde^2}\rhoprime \\

%Strictly, \citet{Gough1969} assumes a time-independent mean state in deriving the continuity equation \eqref{eq:ancont} and the left-hand side of the internal energy equation \eqref{eq:anenmod}. However, these equations are essentially unaltered (to the order of the asymptotic expansion from which they are derived) as long as the slow-variation condition \eqref{eq:meanvariationslow} holds. To see this, note that $\pderivline{\rhoover}{t}$ would be $O(\epsilon)$ and thus the zeroth-order continuity equation remains unchanged. The left-hand side of the internal energy equation must now include the $O(\epsilon)$ term $\rhoover\tmpover \pderivline{\entrhat}{t}$ and is thus transformed from
%\begin{align}\label{eq:anenmodlhs}
%	\rhoover\tmpover\pderiv{\entrone}{t}\five\text{to}\five  \rhoover\tmpover\pderiv{\entrprime}{t}.
%\end{align}
%The mean energy equation \eqref{eq:anenmean2} also must be transformed to

We assume that the reference state has been chosen to be hydrostatic,
\begin{align}\label{eq:refhydro}
	\nabla \tilde{P} = \rhotilde\vecg,
\end{align}
to be in thermal equilibrium,
\begin{align}\label{eq:refthermalequil}
	\tilde{Q} = \Div\tilde{\vecf},
\end{align}
and to satisfy the first law of thermodynamics for gradients (compare to Equation \eqref{eq:anfirstlawmean}),
  \begin{subequations}\label{eq:reffirstlawmean}
	\begin{align}
		\tmptilde \nabla\entrtilde &= \cptilde \nabla\tmptilde - \frac{\deltatilde}{\rhotilde} \nabla\prstilde\label{eq:reffirstlawmeantmpprs}\\
		&= \nabla\intetilde - \frac{\prstilde}{\rhotilde^2}\nabla\rhotilde \label{eq:reffirstlawmeaninterho}\\
		&= \nabla \enthtilde - \frac{\nabla\prstilde}{\rhotilde}\label{eq:reffirstlawmeanenthprs}\\
		&= \cvtilde \nabla\tmptilde - \frac{\prstilde\etatilde}{\rhotilde^2}\nabla\rhotilde\label{eq:reffirstlawmeantmprho}\\		
		&= \frac{\cptilde\, \tmptilde}{\deltatilde\rhotilde}\left[\frac{\nabla\prstilde}{\cssqtilde}-\nabla\rhotilde\right].\label{eq:reffirstlawmeanprsrho}
	\end{align}
\end{subequations}
%Note that this is simply a \textit{definition} for the reference-state term $\tilde{Q}-\Div\tilde{\vecf}$, which can be time-dependent. This is the one reference-state term that is assumed to not be set a priori. Instead, it is moved to the left-hand side of the internal energy equation and the evolution of the sum $\entrprime=\entrhat + \entrone$ is solved for. The approximation remains consistent as long as $\tilde{Q}-\Div\tilde{\vecf}$ remains $O(\epsilon)$ for all time (i.e., it satisfies Equations \eqref{eq:meanvariationslow} and \eqref{eq:meanvariationsmall}). 
%  \define -\rhotilde\tmptilde\pderiv{\entrhat}{t}.
The linearized equation of state for the primed quantities is exactly analogous to Equation \eqref{eq:anfirstlawpert},
  \begin{subequations}\label{eq:reffirstlawpert}
	\begin{align}
		\tmptilde \entrprime &= \cptilde \tmpprime - \frac{\deltatilde}{\rhotilde} \prsprime\label{eq:reffirstlawperttmpprs}\\
		&= \inteprime - \frac{\prstilde}{\rhotilde^2}\rhoprime \label{eq:reffirstlawpertinterho}\\
		&= \enthprime - \frac{\prsprime}{\rhotilde}\label{eq:reffirstlawwpthprs}\\
		&= \cvtilde \tmpprime - \frac{\prstilde\etatilde}{\rhotilde^2}\rhoprime\label{eq:reffirstlawperttmprho}\\		
		&= \frac{\cptilde\, \tmptilde}{\deltatilde\rhotilde}\left[\frac{\prsprime}{\cssqtilde}-\rhoprime\right].\label{eq:reffirstlawpertprsrho}
	\end{align}
\end{subequations}
as is the linearized equation of state for the hatted quantities. Note that in Equation \eqref{eq:anfirstlawpert}, the overbars may be replace by tildes, making Equation \eqref{eq:anfirstlawpertref} applicable to the ``subscript 1" quantities as well. 

To zeroth order in $\epsilon$, Equation \eqref{eq:ancont} becomes simply
\begin{align}\label{eq:refcont}
	\Div(\rhotilde\vecu)\equiv0.
\end{align}
Because the right-hand side of Equations \eqref{eq:anmom} is $O(\epsilon)$ compared to the left-hand side we can write, using Equations \eqref{eq:egghydro} and \eqref{eq:anfirstlawpert} for the hatted variables,
\begin{align}\label{eq:refturbpressure}
	[-\nabla \prsover + \rhoover\vecg] &= -\nabla \prshat + \rhohat\vecg\nonumber\\
	&= -\rhotilde\nabla\left(\frac{\prshat}{\rhotilde}\right) - \deltatilde  \rhotilde \left(\frac{\entrhat}{\cptilde}\right) \vecg + \frac{\deltatilde \rhotilde}{\cptilde} \left(\frac{\prshat}{\rhotilde}\right)\nabla\entrtilde.
\end{align}
%Note that the non-LBR force density from Equation \eqref{eq:nlbrmom} is only significant, i.e., $O(\epsilon)$, when $\nabla\entrover$ is large (in which case, $\nabla\entrover=\nabla\entrtilde+O(\epsilon)$ by Equations \eqref{eq:meanvariationslow} and \eqref{eq:meanvariationsmall}), otherwise it is $O(\epsilon^2)$. For all magnitudes of $|\nabla\entrover|$, we can thus write
%\begin{align}\label{eq:nlbrmom2}
%	\bm{f}_{\rm NLBR} = \frac{\deltaover \prsone}{\cpover} \nabla\entrtilde + O(\epsilon^2)
%\end{align}
Substituting Equation \eqref{eq:anrhsmomref} into Equation \eqref{eq:anmommod} and noting that all terms are of $O(\epsilon)$ (so that we can replace overbars with tildes), we find
\begin{subequations}\label{eq:refmom}
	\begin{align}\label{eq:refmommain}
	\pderiv{}{t}(\rhotilde\vecu)= &-\Div(\rhotilde\vecu\vecu) -\rhotilde\nabla\left(\frac{\prsprime}{\rhotilde}\right) - \deltatilde \rhotilde \left(\frac{\entrprime}{\cptilde}\right) \vecg +\frac{\deltatilde  \prsprime}{\cptilde} \nabla\entrtilde+\Div \overleftrightarrow{D},\\
	\text{where now}\five D_{ij}& = \tilde{\mu}\left[\pderiv{u_i}{x_j} + \pderiv{u_j}{x_i} - \frac{2}{3}(\Div\vecu)\delta_{ij}\right]\label{eq:refmomvisc},
\end{align}
\end{subequations}
%In the internal energy equation \eqref{eq:anenmod}, we write 

Using Equation \eqref{eq:nomeanu}, Equation \eqref{eq:momdensity} can be additionally written
  \begin{align}\label{eq:momdensity2}
\rhoover\vecu + \rhoone\vecu - \overline{\rhoone\vecu} = \rhotilde\vecu + \rhoprime\vecu - \overline{\rhoprime\vecu} + O(\epsilon^2).
\end{align}

We finally write 
\begin{align}
	Q_{\rm NLBR} = -\rhotilde \tmpone\ugrad\entrtilde + O(\epsilon^2)
\end{align}
and 
\begin{align}\label{eq:anenrhsfinal}
	-\rhoover\tmpover \ugrad \entrone - \rhoover\tmpover \ugrad\entrover &= - \rhoover\tmpover\ugrad \entrprime -\rhoover\tmpover\ugrad\entrtilde \nonumber\\
	&=-\rhotilde\tmptilde\ugrad \entrprime - \tmptilde(\rhoover \vecu)\cdot\nabla\entrtilde - \rhotilde\tmphat\ugrad\entrtilde + O(\epsilon^2).
\end{align}
Substituting Equations \eqref{eq:eggfirstlaw} and \eqref{eq:momdensity2}--\eqref{eq:anenrhsfinal} into Equation \eqref{eq:anenmod} yields
\begin{subequations}\label{eq:refen}
	\begin{align}
		\rhotilde\tmptilde\pderiv{\entrprime}{t}= &-\rhotilde\tmptilde\ugrad \entrprime - \rhotilde\tmptilde \ugrad \entrtilde - \rhotilde \tmpprime\ugrad\entrtilde+D_{ij}\pderiv{u_i}{x_j} + Q^\prime- \Div\vecf^\prime  \nonumber\\
		& - \tmptilde(\rhoprime \vecu-\overline{\rhoprime\vecu})\cdot\nabla\entrtilde,\label{eq:refenmain}\\
		\where \rhoprime =& \frac{\prsprime}{\tilde{\cs^2}}-\frac{\deltatilde \rhotilde S ^\prime}{\cptilde},\label{eq:refenrhopert}\\
		 \tmpprime = & \frac{\tmptilde \entrprime}{\cptilde} + \frac{\deltatilde  \prsprime}{\rhotilde\cptilde},\label{eq:refentmppert}
	\end{align}
\end{subequations}
and $D_{ij}$ is now given by Equation \eqref{eq:refmomvisc}. 

Taking the horizontal means of Equations \eqref{eq:refmom} and \eqref{eq:refen} and then using Equations \eqref{eq:egghydro}, \eqref{eq:eggfirstlaw}, and \eqref{eq:anrhsmomref} recovers the mean momentum and energy equations \eqref{eq:anmommean} and \eqref{eq:anenmean2}. The anelastic formulation with fixed reference states (Equations \eqref{eq:eggcont}, \eqref{eq:refmom} and \eqref{eq:refen}) is thus seen to be asymptotically equivalent to the formulation with horizontally averaged background states (Equations \eqref{eq:ancont}, \eqref{eq:anmommod}, and \eqref{eq:anenmod}, combined with Equations \eqref{eq:anmommean} and \eqref{eq:anenmean2}), provided the restriction \eqref{eq:meanvariationsmall} holds.

Intuitively, none of the results of this section should seem too surprising and may even seem obvious. However, formalizing the equivalence between mean-state expansions of the anelastic equations and reference-state expansions via \eqref{eq:meanvariationslow} and \eqref{eq:meanvariationsmall} is important. The mean-state anelastic approximation of (e.g.) \citet{Gough1969} can consistently allow for time-varying mean states, subject only to the slow-variation constraint \eqref{eq:meanvariationslow}. Furthermore, the mean-state equations can allow for secular change in the mean state (e.g., a growing or shrinking convection zone) self-consistently, as long as the evolution is slow. The reference-state formulation, by contrast, does not allow for secular changes in the mean state, because the total change must remain small as per Equation \eqref{eq:meanvariationsmall}.

One pernicious aspect of the reference-state formulation is that the equations have no way to ``inform" the evolution if the reference state was chosen unwisely. For example, a fluid layer with a reference state not in hydrostatic balance will evolve basically as normal under Equations \eqref{eq:ancont}, \eqref{eq:anmommod}, and \eqref{eq:anenmod} (a physically unreasonable result). Under the mean-state formalism by contrast, a fluid layer with initial mean state not in hydrostatic balance will explode or collapse via Equation \eqref{eq:anmommean} (a physically reasonable result). On the other hand, during the explosion or collapse, the fluid will develop very large thermal perturbations and the original assumption of anelasticity (either via the mean-state or reference-state formalism) will have been made unjustly. 

To summarize: as long as the true fully compressible system is not expected to change its mean state by more than $O(\epsilon)$ during evolution, either the mean-state or reference-state anelastic equations may be used with asymptotically equivalent results. If secular variation of the real system \textit{is} expected, it must be slow enough to satisfy Equation \eqref{eq:meanvariationslow}. Only then is the use of the mean-state anelastic equations justified.

\section{The Anelastic Energy-conserving Generalized Gough (AnEGG) approximation}
To arrive at the final form of an energy-conserving set of anelastic equations, we make one final argument: that the last term in Equation \eqref{eq:refenmain} is negligible. We do so because this term has zero horizontal mean and therefore cannot affect conservation of energy energy or the mean stratification $\pderivline{\entrover}{q}$. Furthermore, pointwise it should always be much smaller than the background advection term $-\rhotilde\tmptilde\ugrad\entrtilde$, provided $\rhoprime$ remains $\ll \rhotilde$. 

The final equations, representing what we call the Anelastic Energy-conserving Generalized Gough (AnEGG) approximation, written for time-independent reference states, are thus
\begin{subequations}\label{eq:egg}
\begin{empheq}[box=\fbox]{align}
	\Div(\rhotilde\vecu)&\equiv 0\label{eq:eggcont},\\
	\pderiv{}{t}(\rhotilde\vecu)&=-\Div(\rhotilde\vecu\vecu) -\rhotilde\nabla\left(\frac{\prsprime}{\rhotilde}\right) - \deltatilde \rhotilde \left(\frac{\entrprime}{\cptilde}\right) \vecg \underbrace{+ \frac{\deltatilde  \prsprime}{\cptilde} \nabla\entrtilde}_{\define \tilde{\bm{f}}_{\rm NLBR}}+\Div \overleftrightarrow{D},\label{eq:eggmom}\\	
		\andd \rhotilde\tmptilde\pderiv{\entrprime}{t}&= -\rhotilde\tmptilde\ugrad \entrprime - \rhotilde\tmptilde \ugrad \entrtilde \underbrace{ -  \left(\frac{\tmptilde \entrprime}{\cptilde} + \frac{\deltatilde  \prsprime}{\rhotilde\cptilde}\right)  \rhotilde \ugrad\entrtilde }_{\define \tilde{Q}_{\rm NLBR}} \nonumber\\
		&\ \ \ \ +D_{ij}\pderiv{u_i}{x_j} + Q^\prime- \Div\vecf^\prime,\label{eq:eggen}
\end{empheq}
\end{subequations}
where $\overleftrightarrow{D}$ is defined in Equation \eqref{eq:refmomvisc}. %and it is additionally assumed that the fixed reference state satisfies the hydrostatic condition \eqref{eq:egghydro} and the first law of thermodynamics in gradient form, Equation \eqref{eq:anfirstlawref}. 

The AnEGG kinetic energy equation, derived from $\vecu$ dotted into Equation \eqref{eq:eggmom}, is 
%\begin{equation}\label{eq:ankeref}
	\begin{empheq}[box=\fbox]{align}\label{eq:eggke}
	\pderiv{}{t}\left(\frac{1}{2}\rhotilde u^2\right)&=-\Div\left(\frac{1}{2}\rhotilde u^2\vecu \right) - \Div(\prsprime\vecu)  - \deltatilde \rhotilde \left(\frac{\entrprime}{\cptilde}\right) \vecu\cdot \vecg +\frac{\deltatilde  \prsprime}{\cptilde} \vecu\cdot \nabla\entrtilde  + u_i\pderiv{D_{ij}}{x_j},
	\end{empheq}
%\end{equation}
Adding Equations \eqref{eq:eggen} and \eqref{eq:eggke} yields the AnEGG total energy equation,
\begin{empheq}[box=\fbox]{align}\label{eq:eggtote}
	\pderiv{}{t}\left[\rhotilde\left(\frac{1}{2} u^2 + \tmptilde \entrprime \right)\right] = &-\Div\left\{\left[\rhotilde\left(\frac{1}{2} u^2 + \tmptilde \entrprime\right) + \prsprime \right]\vecu- \vecu\cdot\overleftrightarrow{D} + \vecf^\prime \right\} + Q^\prime \nonumber\\ & -\rhotilde\tmptilde\ugrad\entrtilde. 
\end{empheq}
Again using condition \eqref{eq:nomeanu} (or equivalently, Equation \eqref{eq:eggcont} and assuming $w\equiv0$ on the boundaries), integrating Equation \eqref{eq:eggtote} over $V$, using the divergence theorem, and assuming that the sum of the surface-integrated fluxes balances the volume-integrated heating yields
\begin{subequations}\label{eq:eggeconst}
\begin{empheq}[box=\fbox]{align}
\eggetot &\define \int_V \eggwtot dV = \text{constant},\label{eq:eggetot}\\
\where \eggwtot &\define \eggwke + \eggwheat,\label{eq:eggwtot}\\
 \eggwke &\define \frac{1}{2} \rhotilde u^2,\label{eq:eggwke}\\
 \andd \eggwheat &\define \rhotilde\tmptilde \entrprime.\label{eq:eggwheat}
\end{empheq}
\end{subequations}
This conservation law holds for arbitrary fluid motions obeying Equations \eqref{eq:egg} and for all magnitudes of $|\nabla\entrover|$. 

Note that in practice when simulating stiff systems (large $|\nabla\entrtilde|$) numerically (e.g., \citealt{Guerrero2016a,Matilsky2022,Matilsky2024}), the term 
\begin{align}
	\tilde{Q}_{\rm adv}\define  -\rhotilde\tmptilde\ugrad\entrtilde
\end{align}
may pointwise be quite large (though it can never alter the mean stability $\pderivline{\entrhat}{q}$). The degree to which energy is conserved numerically may thus be limited by the precision of the condition \eqref{eq:nomeanw}. In the streamfunction formulation of {\ash} and {\rayleigh} for example (e.g., \citealt{Clune1999,Featherstone2016a}), Equation \eqref{eq:nomeanw} holds to near machine precision. 

To summarize this section, necessary conditions for anelastic codes implementing background stable layers to conserve energy are (1) including the non-LBR force density,
\begin{align}\label{eq:nlbrmomref}
	\tilde{\bm{f}}_{\rm NLBR} \define\frac{\deltatilde  \prsprime}{\cptilde} \vecu\cdot \nabla\entrtilde 
\end{align}
in the momentum equation (i.e., not making the LBR approximation) and (2) including the non-LBR heating term,
\begin{align}\label{eq:nlbrheatref}
	\tilde{Q}_{\rm NLBR} \define -  \left(\frac{\tmptilde \entrprime}{\cptilde} + \frac{\deltatilde  \prsprime}{\rhotilde\cptilde}\right)  \rhotilde \ugrad\entrtilde
\end{align}
in the internal energy equation. Alternatively, if the LBR approximation is made, (setting $\tilde{\bm{f}}_{\rm NLBR}\equiv0$ in Equation \eqref{eq:refmom}), total energy will still be conserved if $\tilde{Q}_{\rm NLBR}$ in Equation \eqref{eq:refen} is replaced by
\begin{align}\label{eq:lbrheating}
	\tilde{Q}_{\rm LBR}\define - \left(\frac{\rhotilde\tmptilde}{\cptilde} \right) \entrprime\ugrad\entrtilde.
\end{align}
However, doing this would yield an asymptotically inconsistent equation set if $|\nabla\entrtilde|$ is large. 

\section{The meaning of total energy under the anelastic approximation}\label{sec:meaningtote}
In this section, we discuss how the expression for total energy under the anelastic approximation $\eggwtot$ (or equivalently, $\anwtot$) arises. The form of $\eggwtot$ places unique constraints on possible avenues for the transfer of energy under the anelastic approximation, and these fundamentally differ from the avenues available for the true compressible motion. It is found that all compressible effects, including the potential energy, are entirely eliminated from the energy budget. 

One perhaps surprising corollary is that \textit{no transfer between potential and kinetic energy is possible for anelastic convection.} Given the prevalence of the concept of ``available potential energy" (APE), especially in atmospheric physics (e.g., \citealt{Vallis2017}, p. 138), these may seem like fighting words, and so we tread with extra mathematical caution in this section. The true available energy (which fundamentally is due to changes in internal energy from heating or cooling, i.e., entropy changes) can resemble a potential energy because of a pathological coincidence that is unique to adiabatically stratified background states and constant $\cptilde/\deltatilde$. Since these conditions are satisfied for a fully and vigorously convecting perfect gas (which is thus nearly adiabatically stratified), in many cases the distinction between potential and heat energy is a philosophical one. However, in more general circumstances (such as convective overshoot into a stably stratified region), the distinction is important. 

We note a priori that the zero mean mass flux condition \eqref{eq:nomassflux} immediately eliminates possible conversion to and from potential energy for the fully compressible flow (since no net mass can be transferred vertically). Thus, it may not be surprising after all that potential energy disappears from the anelastic energy budget. Perhaps one of the main new results of this work is that the assumption of zero mean mass flux is \textit{fundamental} for the energetic consistency of the anelastic approximation. 

\subsection{General expression for $\wpnke$}
From Equations \eqref{eq:wnke} and \eqref{eq:linuref}, the perturbed nonkinetic energy per unit volume is 
\begin{align}\label{eq:wp1}
	\wpnke &=  \rhotilde U^\prime + \tilde{U}\rhoprime + \Phi \rhoprime \nonumber\\
	&= \rhotilde\tmptilde \entrprime + \left(\tilde{U}+\frac{\tilde{P}}{\rhotilde}+\Phi\right)\rhoprime.
\end{align}
%\begin{align}\label{eq:wp1}
%	\wpnke &=  \rhotilde U^\prime + \tilde{U}\rhoprime + \Phi \rhoprime \nonumber\\
%	&= \underbrace{\rhotilde\tmptilde \entrprime}_{\define \wpheat} + \underbrace{\left(\tilde{U}+\frac{\tilde{P}}{\rhotilde}+\Phi\right)\rhoprime}_{\define\wpcomp}.
%\end{align}

Equation \eqref{eq:linuref} is really an expression of the first law of thermodynamics, which states that changes in the internal energy of fluid parcels come from the sum of irreversible heating processes and pressure work done by the environment during compression. Comparing to Equation \eqref{eq:eggeconst}, it is clear that \textit{only} changes in internal energy from heat sources can effect the total energy. The changes in internal energy from density variations $\intetilde\rhoprime$, the pressure work $\prstilde(\rhoprime/\rhotilde)$, and the changes in potential energy $\Phi\rhoprime$ are all absent. Since all three of these latter effects are due to density variations (i.e., compression or expansion of the fluid), we rewrite Equation \eqref{eq:wp1} as
\begin{subequations}\label{eq:wpnke}
	\begin{align}
		\wpnke &= \wpheat + \wpcomp,\\
		\where \wpheat&\define \rhotilde\tmptilde \entrprime\\
		\andd \wpcomp &\define \left(\intetilde+\frac{\prstilde}{\rhotilde}+\Phi\right)\rhoprime
	\end{align}
\end{subequations}

We also write the linear form of Equation \eqref{eq:wtot} as
\begin{subequations}\label{eq:wpke}
	\begin{align}
		\wptot &\define \wpke + \wpnke,\\
		\where \wpke &\define \frac{1}{2}\rhotilde u^2 = \eggwke.
	\end{align}
\end{subequations}

$\wptot$ represents the total energy density whose volume integral the true fully compressible fluid would conserve (at first order in $\epsilon$) under the assumption \eqref{eq:smallthermalpert} of small thermal perturbations. Comparing Equations \eqref{eq:eggeconst}, \eqref{eq:wpnke}, and \eqref{eq:wpke}, we see that the fully compressible motion is formally able to convert kinetic energy to and from internal energy from heating $\wpheat$ \textit{and} to and from the compression effects accounted for by $\wpcomp$. By contrast, the anelastic motion is \textit{only} able to convert kinetic energy to and from internal energy from heating.
%	&=  -\rhotilde\Phi\left(\frac{\entrprime}{\cpcap}\right)+\frac{\prsprime}{\gamma-1}
% where $\cpcap$ and $\cvcap$ are constants and 
%\begin{subequations}\label{eq:idgas}
%	\begin{align}
	%	\prstilde &= (\gamma-1)\cvcap\rhotilde\tmptilde, \label{eq:idgasprs}\\
	%	 \tilde{U} &= \cvcap \tmptilde, \label{eq:idgasinte}\\
	%	 \andd \cssqtilde &= (\gamma-1)\cpcap\tmptilde, \label{eq:idgascs2}
	%\end{align}
	%\end{subequations}
	%where we define $\gamma\define\cpcap/\cvcap$. 

\subsection{Special form  of $\wpnke$ for adiabatically stratified perfect gases}
In contrast to Equation \eqref{eq:wpnke}, \citet{Ogura1962} write
\begin{subequations}\label{eq:wpogura}
\begin{align}
	\wpnke &\define \wppot + \wpelast,\\
	\where \wppot &\define -\rhotilde\Phi\left(\frac{\entrprime}{\cpcap}\right)\\
	\and \wpelast &\define \rhotilde\cvcap\tmptilde_0\left(\frac{\prsprime}{\prstilde}\right)
\end{align}
\end{subequations}
Here, $\tilde{\cpcap}\equiv\cpcap$ and $\tilde{\cvcap}\equiv\cvcap$ are constants, since \citet{Ogura1962} assumed a perfect gas. The gravitational potential $\Phi$ is measured from the location $q_0$ of the reference pressure for the potential temperature (in \citealt{Ogura1962}, this is the bottom of the fluid layer) and $\tmptilde_0\define \tmptilde(q_0)$. The reference state is assumed to be adiabatic,
\begin{align}\label{eq:refadiabatic}
	\frac{d\entrtilde}{dq}\equiv0,
\end{align}
and thus the background potential temperature is constant and equal to $\tmptilde_0$.\footnote{See the original equation immediately following \citet{Ogura1962}'s Equation (31). To see the equivalence to our Equation \eqref{eq:wpogura}, note that \citet{Ogura1962} use $\theta$ for the non-dimensional order-unity perturbed potential temperature, $gz^\prime$ for the dimensional gravitational potential (equal to $\Phi$ here), and $\Theta$ for the constant dimensional potential temperature (equal to $\tmptilde_0$ here) of the adiabatically stratified background. Also note the relation $\entrprime=\epsilon\cpcap\theta$.}

\citealt{Eckart1960} (p. 54) discussed the energy density associated with small perturbations about a background static atmosphere and identified a term with pressure perturbations, $\prsprime^2/(2\rhotilde\cssqtilde)$, as the ``elastic" energy density from acoustics (for the acoustics definition of the term, see \citealt{Morse1948}, p. 237). \citet{Ogura1962} identify $\wppot$ as a potential energy and note that the energy density $\wpelast$, which also contains the pressure perturbation, disappears from the equation of total energy conservation (see \citealt{Ogura1962}'s Equation (31)). This disappearance is apparently the origin of the term ``anelastic," coined by Jule Charney.% The potential energy term also disappears, which is a direct consequence of the assumption of zero mean mass flux, Equation \eqref{eq:nomassflux}. 

However, in light of Equation \eqref{eq:wpnke}, which must be the more general form of Equation \eqref{eq:wpogura}, it is clear that the partition $\wpnke = \wppot + \wpelast$ can only hold under very specific circumstances, namely one in which $\tmptilde$ is proportional to $\Phi$. The latter condition holds only for an adiabatically stratified fluid where also $\cptilde/\deltatilde\equiv\text{constant}$. Furthermore, it turns out that the partition of Equation \eqref{eq:wpogura} \textit{only} holds for the unique circumstance of a gas that is not only adiabatically stratified, but also perfect (we assume up front that the gas is hydrostatically stratified). This should not be a surprise (since \citealt{Ogura1962} assumed a perfect gas to begin with), however, it is useful to go through the proof in order to understand that the appearance of $\wppot$ as a potential energy (which, it will be found, it fundamentally is not) is almost exclusively unique to adiabatically stratified perfect gases. 

We use hydroastic balance \eqref{eq:refhydro} and the first law of thermodynamics for the reference-state gradients \eqref{eq:reffirstlawmean} to yield
\begin{align}\label{eq:tmpphi1}
	\frac{d\Phi}{dq} = -\frac{\cptilde}{\deltatilde}\frac{d\tmptilde}{dq} + \frac{\tmptilde}{\deltatilde}\frac{d\entrtilde}{dq}.
\end{align}
If and only if 
\begin{align}\label{eq:cpdeltaconst}
	\frac{\cptilde}{\deltatilde} \equiv \text{constant} \define \frac{\cpcap}{\delta}
\end{align}
(the constancy of the individual coefficients comes later; we also write $\delta/\cpcap\define(\cpcap/\delta)^{-1}$), Equation \eqref{eq:tmpphi1} can be integrated to yield
\begin{align}\label{eq:tmpphi2}
	\Phi = -\frac{\cpcap}{\delta}(\tmptilde-\tmptilde_0) + \int_{q_0}^q\frac{\tmptilde(q^\prime)}{\deltatilde(q^\prime)}\frac{d\entrtilde}{dq^\prime}dq^\prime
\end{align}
Then, if and only if the background is adiabatically stratified ($d\entrtilde/dq\equiv0$), there exists a simple proportionality between temperature and gravitational potential:\footnote{We exclude the case $\tmptilde d\entrtilde = (\cptilde + K \deltatilde)d\tmptilde$, where $K$ is a constant, which would yield $d\Phi = K d\tmptilde$ in Equation \eqref{eq:tmpphi1} even with nonzero $d\entrtilde/dq$. However, from Equation \eqref{eq:reffirstlawmeantmpprs}, this would be possible if and only if $\deltatilde\equiv0$, in which case hydrostatic balance would decouple from the rest of the thermodynamics. We also exclude the case of degenerate matter ($\deltatilde\equiv\infty$).}
\begin{align}\label{eq:tmpphi3}
	\Phi = -\frac{\cpcap}{\delta}(\tmptilde-\tmptilde_0).
\end{align}
Using Equations \eqref{eq:reffirstlawpertprsrho} and \eqref{eq:tmpphi3}, Equation \eqref{eq:wpnke} becomes
\begin{align}\label{eq:wpnke2}
	\wpnke = \rhotilde\left[-\frac{\delta}{\cpcap}\Phi - \frac{\delta}{\cpcap}\left(\enthtilde - \frac{\cpcap}{\delta}\tmptilde\right)\right] \entrprime +\left[\left(\enthtilde-\frac{\cpcap}{\delta}\tmptilde\right) + \frac{\cpcap}{\delta}\tmptilde_0\right]\frac{\prsprime}{\cssqtilde}.
\end{align}
This reduces to Equation \eqref{eq:wpogura} if and only if 
\begin{align}\label{eq:enthfunctmp}
	\enthtilde = \frac{\cpcap}{\delta}\tmptilde = \enthtilde(\tmptilde). 
\end{align}
Comparing Equations \eqref{eq:reffirstlawmeanenthprs} and \eqref{eq:reffirstlawmeantmpprs}, Equation \eqref{eq:enthfunctmp} implies $\deltatilde\equiv\delta=1$ and thus $\cptilde\equiv\cpcap=\text{constant}$ by Equation \eqref{eq:cpdeltaconst}.  

From Equations \eqref{eq:reffirstlawmeantmprho}, \eqref{eq:reffirstlawmeanprsrho}, and \eqref{eq:refadiabatic}, it can be shown that
\begin{align}\label{eq:cpcssq}
	\frac{\cpcap}{\cssqtilde} = \frac{\rhotilde\cvtilde}{\etatilde\prstilde}.
\end{align}

Substituting Equations \eqref{eq:enthfunctmp} and \eqref{eq:cpcssq} into Equation \eqref{eq:wpnke2} then yields
\begin{align}\label{eq:wpnke3}
	\wpnke =-\rhotilde\Phi\left(\frac{\entrprime}{\cpcap}\right) + \frac{\rhotilde\cvtilde\tmptilde_0}{\etatilde}\left(\frac{\prsprime}{\prstilde}\right),
\end{align}
which finally reduces to Equation \eqref{eq:wpogura} if and only if $\cvtilde\equiv\cvcap=\text{constant}$ and $\etatilde\equiv1$. It can be shown that the conditions $\deltatilde\equiv\etatilde\equiv1$ imply the ideal gas law (and thus $\intetilde=\intetilde(\tmptilde)$), and the condition $\cptilde=\text{constant}$ then implies that the gas is perfect.

\subsection{The absence of potential energy and the importance of zero mean mass flux}
Technically, the same conclusion reached by \citet{Ogura1962} (that the term $\wppot$ comes from residual potential energy) is suggested by Equation \eqref{eq:wpnke3}, even for the slightly more general case of adiabatically stratified but nonideal gases ($\cvtilde\neq\text{constant}$, $\etatilde\neq1$, $\intetilde\neq\intetilde(\tmptilde)$). However, given that $\enthtilde=\enthtilde(\tmptilde)$, $\deltatilde\equiv1$, and $\cptilde\equiv\text{constant}$ are all necessary conditions for the term $\wppot$ to appear, it seems likely that adiabatically stratified perfect gases are the most common systems where the coincidence occurs. 

The main point is that the term $\wppot$ in no way derives from a potential energy and the pressure-fluctuation term $\wpelast$ is not the term eliminated from anelastic energy conservation. The seeming partition of $\wpnke$ into potential and ``elastic" components (either Equation \eqref{eq:wpogura} or Equation \eqref{eq:wpnke3}) is a coincidence unique to adiabatic stratification and arises almost exclusively in perfect gases. The true partition of $\wpnke$ (Equation \eqref{eq:wpnke}) yields two components: one due due changes in internal energy due to to irreversible heating ($\wpheat$) and one due to compression effects ($\wpcomp$). Under the anelastic approximation, $\wpcomp$ (which contains the potential energy) is entirely eliminated from the energy budget. In light of this discussion, a better name for the ``anelastic" approximation might be the ``incompressible" approximation, although such a rebranding is not possible for obvious reasons. 

For the true fully compressible fluid, Equation \eqref{eq:nomassflux} applied to the continuity equation \eqref{eq:cont} yields the obvious result
\begin{align}\label{eq:drhodt0}
	\pderiv{\rhoover}{t}\equiv0,
\end{align}
i.e., there can be no net increase or decrease in mass at any particular vertical level (here the vanishing of the horizontal components of $\overline{\rho\vecu}$, where in this section $\vecu$ again refers to the total velocity, becomes necessary\footnote{Or the vanishing of the horizontal components of $\vecu$ on the horizontal boundaries, or semi-periodic boundaries, as in a spherical shell. The only requirement is that $\Div(\overline{\rho\vecu})\equiv0$.}). As such, no increase or decrease in energy due to compression effects (including potential energy) is possible. Thus, the compressible motion is energetically consistent with the anelastic approximation if the zero mean mass flux condition holds. 

Furthermore, the anelastic approximation only conserves energy (for all magnitudes of $|\nabla\entrtilde|$) if at least the vertical component of mass flux, $\overline{\rho w}$, is assumed to vanish up to $O(\epsilon)$. To see this, first note that if we do not assume any of the components of $\overline{\rho \vecu}$ vanish a priori (and recall that $\vecu$ and $w$ now refer to the total---mean plus perturbed---velocity), the anelastic condition is still $\nabla\cdot(\rhotilde\vecu)\equiv0$ (e.g., \citealt{Ogura1962}), which can be integrated to yield $\overline{w}\equiv0$, as long as $w$ vanishes on the boundaries. We then expand the term representing Lagrangian evolution of heat energy as
\begin{align}\label{eq:enlhsfirstorder}
	\rho T\frac{DS}{Dt} =\rhotilde\tmptilde\pderiv{\entrprime}{t}  + \rhotilde\tmptilde\ugrad \entrprime + \underbrace{\rhotilde\tmptilde \ugrad \entrtilde}_{=-\tilde{Q}_{\rm adv}}+ \underbrace{ \rhoprime\tmptilde \ugrad\entrtilde }_{\define- \tilde{Q}_{\rm massflux}} + \underbrace{ \tmpprime\rhotilde \ugrad\entrtilde }_{=- \tilde{Q}_{\rm NLBR}} + O(\epsilon^2). 
\end{align}

Since the continuity and momentum equations \eqref{eq:eggcont} and \eqref{eq:eggmom} are unmodified even if there may be mean mass flux (with the exception that $\vecu$ is now taken to mean the full velocity) and  $\tilde{Q}_{\rm massflux}$ is the only new term that would appear on the right-hand side of Equation \eqref{eq:eggen}, total energy conservation remains possible if and only if the volume integral of $\tilde{Q}_{\rm massflux}$ vanishes. This is guaranteed if and only if we demand $\overline{\rho^\prime w}\equiv0$, i.e., the vertical mass flux in both equation sets (fully compressible and anelastic) vanishes to first order in $\epsilon$.

From Equation \eqref{eq:enlhsfirstorder}, we begin to see why the AnEGG equations include the term $\tilde{Q}_{\rm NLBR}$ (which is crucial for energy conservation), but other anelastic equation sets do not (and hence do not conserve energy for nonadiabatic stratification). \citet{Gough1969} nondimensionalized the internal energy equation and then linearized by expanding to first order in $\epsilon$. The other approximations instead linearized the entropy (or potential temperature) equation (e.g., \citealt{Batchelor1953,Ogura1962,Rogers2005}). If this is done, the Lagrangian evolution of entropy, 
\begin{align}
	content...
\end{align}
%\footnote{Strictly, this is not an ``and only if" statement, since $\tilde{Q}_{\rm massflux}$ could vanish if $\overline{\rho^\prime w}$ is nonzero but cancels itself out when integrated vertically. However, there is no reason to expect such cancellation from the equations.}
%For a perfect gas [$U=\cvcap T$ and $P=(\cpcap-\cvcap)\rho T)$, with constant specific heats] with a hydrostatic, adiabatic mean background state satisfying Equations \eqref{eq:anfirstlaw} and \eqref{eq:anhydrostatic}, we have
%\begin{align}\label{eq:tmppot}
%	\Phi = -\cpcap \tmpover + K,
%\end{align}
%where $K$ is a constant. Equation \eqref{eq:wp1} then becomes
%\begin{align}\label{eq:wp2}
%	W_1 = - \frac{\rhoover\Phi}{\cpcap}\entrone + K\frac{\prsone}{\cssqover}
%\end{align}
%The disappearance of the compressive term, which was called the ``elastic energy" in \citet{Eckart1956}, is the origin of the term ``anelastic," coined by Jule Charney (see \citealt{Ogura1962}). The potential energy term also disappears, which is a direct consequence of the assumption of zero mean mass flux, Equation \eqref{eq:nomassflux}. 

%The latter compressive term $(\prsover/\rhoover^2)\rhoone$ is thus eliminated under the anelastic approximation and we instead identify $\tmpover \entrone$ with the internal energy. 
\section{Conservation of mass and the second law of thermodynamics}
Although there is no mean mass flux in both the fully compressible and anelastic equations (see Equations \eqref{eq:nomassflux} and either \eqref{eq:momdensity} or \eqref{eq:momdensity2}), total mass is \textit{not} conserved at first order under the anelastic approximation.\footnote{In this section, $\vecu$ again refers to $\vecu_1$} Instead of Equation \eqref{eq:drhodt0}, we have from (e.g.) Equation \eqref{eq:refenrhopert}, 
\begin{align}\label{eq:drhodtnot0}
 \pderiv{\rhoover}{t} &= \frac{1}{\tilde{\cs^2}}\pderiv{\prsover}{t}-\frac{\deltatilde \rhotilde}{\cptilde}\pderiv{\entrover}{t}\nonumber\\
 &=  \frac{1}{\tilde{\cs^2}}\pderiv{\prshat}{t}-\frac{\deltatilde \rhotilde}{\cptilde}\pderiv{\entrhat}{t}.\
\end{align}
$\pderivline{\prshat}{t}$ can be written in terms of $\pderivline{\overline{w^2}}{t}$ and $\pderivline{\entrhat}{t}$ via Equations \eqref{eq:anmommean} and \eqref{eq:refturbpressure} and $\pderivline{\entrhat}{t}$ comes directly from the mean of Equation \eqref{eq:eggen}. 

%One slightly surprising result concerns the second law of thermodynamics. In the absence of heat sources or sinks, when the true motion should be adiabatic, Equation \eqref{eq:eggen} becomes
%\begin{align}\label{eq:dsdtref}
%	\frac{D}{Dt}(\entrtilde + \entrprime) = \frac{Q_{\rm NLBR}}{\rhotilde\tmptilde}.
%\end{align}
%Thus, the AnEGG equations do not conserve total entropy for adiabatic motion, violating the second law of thermodynamics if $\entrprime$ is strictly interpreted as the entropy. Apparently making the anelastic approximation allows for the conservation of energy or the second law of thermodynamics to be satisfied, but not both. 

\section{Conditions for anelastic overshoot}
[THIS SECTION IS A STUB]. What I have so far is that a downflowing plume should have kinetic energy 
\begin{align}
	w^2 \sim g_a H_a \left(\frac{\entrprime}{\cpa}\right) \sim \csa^2 \epsilon
\end{align}
If it descends into a stable stratification with buoyancy frequency
\begin{align}
	N_a^2 =   \frac{g_a}{\cpa}|\nabla S_a|,
\end{align}
where $|\nabla S_a|$ is a typical value for $|\nabla\entrtilde|$ in the stable layer. At most, the plume should reach a depth $d$ before it decelerates to $w=0$, with $d$ given by
\begin{align}
	w^2 \sim \left(\frac{\entrprime}{\cpa}\right)gd \sim \frac{g_a}{\cpa}|\nabla S_a| d^2 \sim N_a^2 d^2
\end{align}
or 
\begin{align}
	d\sim \frac{w}{N_a}. 
\end{align}
Thus, $|\entrprime|$ should reach a maximum during overshoot of 
\begin{align}
	\max{\left(\frac{\entrprime}{\cpa}\right)_{\rm stable}} &\sim \left(\frac{|\nabla S_a|}{\cpa}\right)d \sim \left(\frac{N_a^2}{g_a}\right)\frac{w}{N_a}\nonumber\\
	&\sim \left(\frac{N_a}{g_a}\right)\csa \epsilon^{1/2}\sim \left(\frac{N_a}{\csa^2/H_a}\right)\csa \epsilon^{1/2} = \left(\frac{N_a}{\csa/H_a}\right)\epsilon^{1/2}.
\end{align}
We thus expect the thermal perturbations to remain small in the stable layer, provided 
\begin{subequations}
\begin{align}
	\frac{N_a^2}{\omega_{\rm ac}^2} &\lesssim \epsilon,\label{eq:conditionovershoot}\\
	\where \omega_{\rm ac} &\define \frac{\csa}{2H_a}
\end{align}
\end{subequations}
is the acoustic cutoff frequency. 

In the solar radiative zone for example, $\omega_{\rm ac}\sim\sn{3}{-2}\ \rm rad\ s^{-1}$ and $N_a\sim \sn{1.4}{-3}\ \rm rad\ s^{-1}$, so condition \eqref{eq:conditionovershoot} should be satisfied if $\epsilon \sim 10^{-3}$. On the other hand, maybe the scale analysis of \citet{Gough1969} breaks down, since buoyantly decelerated flows (i.e., overshoot and g-modes) may have very different dynamical balances than buoyantly accelerated flows (i.e., convection)? So maybe another expansion is necessary that takes into account not only small thermal perturbations, but anisotropic velocity and length scales. 

%Sufficient conditions for conserving energy will likely rest in the numerical precision with which the AnEGG equations \eqref{eq:egg} are actually solved. 
	\newpage
	%\bibliography{/Users/loren/Desktop/Paper_Library/000_bibtex/library_propstyle, 
		\bibliography{/Users/loren/Desktop/Paper_Library/000_bibtex/library, 
			/Users/loren/Desktop/Paper_Library/000_bibtex/proceedings,
			/Users/loren/Desktop/Paper_Library/000_bibtex/books}
		\bibliographystyle{aasjournal}
	%, therefore getting the properties of internal gravity waves badly wrong.
\end{document}