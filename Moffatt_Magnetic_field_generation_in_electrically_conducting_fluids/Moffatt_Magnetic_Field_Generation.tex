\documentclass[12pt]{article} % document type and language

\usepackage{amsmath, amssymb, bm, mathtools, cancel, empheq, ulem, mathrsfs, natbib}
\setcitestyle{aysep={}} 
%\usepackage{newtxmath} 
\usepackage[margin=1in]{geometry}

\usepackage[colorlinks]{hyperref}
\hypersetup{
	colorlinks = true,
	linkcolor=blue,
	citecolor=blue
}

% Allow option to set color when hyperlinking
\newcommand{\MYhref}[3][blue]{\href{#2}{\color{#1}{#3}}}

\date{\today}
\author{Loren Matilsky}
\title{Moffatt's Magnetic Field Generation in Electrically Conducting Fluids, Notes}
\newcommand{\pderiv}[2]{\frac{\partial#1}{\partial#2}}
\newcommand{\ppderiv}[2]{\frac{\partial^2#1}{\partial#2^2}}
\newcommand{\av}[1]{\langle#1\rangle}
\newcommand{\bigav}[1]{\bigg{\langle}#1\bigg{\rangle}}
\newcommand{\bigfrac}[2]{\bigg{(}\frac{#1}{#2}\bigg{)}}
\newcommand{\mbigfrac}[2]{\bigg{(}{-\frac{#1}{#2}}\bigg{)}}
\newcommand\numberthis{\addtocounter{equation}{1}\tag{\theequation}}
\newcommand{\pomega}{\varpi}
\newcommand{\ugrad}{\bm{u}\cdot\nabla}
\newcommand{\cv}{c_{\rm{v}}}
\newcommand{\cp}{c_{\rm{p}}}
\newcommand{\orr}{\text{or}\ \ \ \ \ }
\newcommand{\andd}{\text{and}\ \ \ \ \ }
\newcommand{\tz}{\tilde{Z}}
\newcommand{\tw}{\tilde{W}}
\newcommand{\five}{\ \ \ \ \ }
\newcommand{\e}{\hat{\bm{e}}}
\newcommand{\tr}{\tilde{r}}
\newcommand{\rhobar}{\overline{\rho}}
\newcommand{\curl}{\nabla\times}
\newcommand{\er}{\hat{\bm{e}}_r}
\newcommand{\Div}{\nabla\cdot}
\newcommand{\ri}{r_{\rm{i}}}
\newcommand{\ro}{r_{\rm{o}}}
\allowdisplaybreaks
\begin{document}
\maketitle
\section*{Ch 2: Magnetokinematic Preliminaries}
\subsection*{1. Identity relating angular momentum operator $L^2$ and Laplacian operator $\nabla^2$}
We compute
\begin{align*}
(\bm{x}\wedge\nabla)^2\psi &= (\epsilon_{ijk}r_j\partial_k)(\epsilon_{ilm}r_l\partial_m)\psi\\
&= (\delta_{jl}\delta_{km} - \delta_{jm}\delta_{kl})r_j(\delta_{lk}\partial_m\psi + r_l\partial_k\partial_m\psi)\\
&= r_k\partial_k\psi + r_jr_j\partial_m\partial_m\psi - 3r_m\partial_m\psi - r_jr_k\partial_k\partial_j\psi\\
&= r^2\nabla^2\psi - 2\bm{x}\cdot\nabla\psi - \bm{x}\cdot(\bm{x}\cdot\nabla)\nabla\psi,\\
\end{align*}
which verifies (2.23). Noting that $\bm{x}\cdot\nabla = r(\partial/\partial r)$, and (since all unit vectors are independent of $r$) that
\begin{align*}
\bm{x}\cdot(\bm{x}\cdot\nabla)\nabla\psi &= (r\e_r)\cdot r\pderiv{}{r}\bigg{(}\pderiv{\psi}{r}\e_r + \dots\bigg{)}\\
&= r^2\e_r\cdot\bigg{(}\frac{\partial^2\psi}{\partial r^2}\e_r + \dots\bigg{)} = r^2\frac{\partial^2\psi}{\partial r^2},
\end{align*}
we then compute
\begin{align*}
r^2\nabla^2\psi - 2\bm{x}\cdot\nabla\psi - \bm{x}\cdot(\bm{x}\cdot\nabla)\nabla\psi &=\\
 r^2\bigg{(}\frac{1}{r^2}\pderiv{}{r}r^2\pderiv{\psi}{r} + \frac{1}{r^2\sin\theta}\pderiv{}{\theta}\sin\theta\pderiv{\psi}{\theta} &+ \frac{1}{r^2\sin^2\theta}\frac{\partial^2\psi}{\partial\phi^2}\bigg{)} - 2r\pderiv{\psi}{r} - r^2\frac{\partial^2\psi}{\partial r^2}\\
 \cancel{2r\pderiv{\psi}{r} + r^2\frac{\partial^2\psi}{\partial r^2}} + \frac{1}{\sin\theta}\pderiv{}{\theta}\sin\theta\pderiv{\psi}{\theta} &+
  \frac{1}{\sin^2\theta}\frac{\partial^2\psi}{\partial\phi^2} - \cancel{2r\pderiv{\psi}{r} - r^2\frac{\partial^2\psi}{\partial r^2}}\\
  &= \frac{1}{\sin\theta}\pderiv{}{\theta}\sin\theta\pderiv{\psi}{\theta} + \frac{1}{\sin^2\theta}\frac{\partial^2\psi}{\partial\phi^2} = L^2\psi.
\end{align*}

\subsection*{2. How to invert $L^2$}
We now derive (2.32), which seems trivial, but really should require a bit of thought. From (2.25), we have
\begin{align*}
f(r,\theta,\phi) &= \sum_n f_n(r)S_n(\theta,\phi),\\
\text{where}\five S_n(\theta,\phi) &= \sum_m A_n^m Y_n^m(\theta,\phi)\\
\andd Y_n^m(\theta,\phi) &\coloneqq P_n^m(\cos\theta)e^{im\phi}.
\end{align*}

Now suppose
\begin{align*}
L^2\psi = f(r,\theta,\phi)
\end{align*}
We can similarly expand $\psi$ in spherical harmonics:
\begin{align*}
\psi(r,\theta,\phi) &= \sum_n \psi_n(r)\tilde{S}_n(\theta,\phi),\\
\text{where}\five \tilde{S}_n(\theta,\phi) &= \sum_m \tilde{A}_n^m Y_n^m(\theta,\phi).\\
\end{align*}
and write
\begin{align*}
L^2\psi &= L^2\sum_n \psi_n(r)\tilde{S}_n(\theta,\phi)\\
			&= \sum_n \psi_n(r)L^2\tilde{S}_n(\theta,\phi)\\
			&= \sum_n \psi_n(r)n(n+1)\tilde{S}_n(\theta,\phi)\\
			&= \sum_n n(n+1)\psi_n(r)\tilde{S}_n(\theta,\phi) = f(r,\theta,\phi) = \sum_n f_n(r)S_n(\theta,\phi).
\end{align*}
From this last equality, it is tempting to identify $f_n = n(n+1)\psi_n$, and be done with it, thereby arriving at (2.32). But this is not really allowed, since we have not shown yet that the $\tilde{S}_n$ are the same as the $S_n$. So we really need to decompose further with respect to $m$ and derive the relationship between the $A$-coefficients:
\begin{align*}
\sum_n n(n+1)\psi_n(r)\sum_m \tilde{A}_n^mY_n^m(\theta,\phi)  = \sum_n f_n(r)\sum_mA_n^mY_n^m(\theta,\phi)\Longrightarrow\\
\sum_{n,m} n(n+1)\psi_n(r) \tilde{A}_n^mY_n^m(\theta,\phi)  = \sum_{n,m} f_n(r)A_n^mY_n^m(\theta,\phi)
\end{align*}
We know that the $Y_n^m$ are all orthogonal, and so we identify
\begin{align*}
n(n+1)\psi_n(r) \tilde{A}_n^m = f_n(r)A_n^m.
\end{align*}
Thus,
\begin{align*}
\psi(r,\theta,\phi) &= \sum_n\psi_n(r)\sum_m\tilde{A}_n^m Y_n^m(\theta,\phi)\\
&= \sum_{n,m}(\psi_n(r)\tilde{A}_n^m)Y_n^m(\theta,\phi)\\
&= \sum_{n,m}\bigg{(}\frac{1}{n(n+1)}f_n(r)A_n^m\bigg{)}Y_n^m(\theta,\phi)\\
&= \sum_n \frac{1}{n(n+1)}f_n(r)\sum_m A_n^mY_n^m(\theta,\phi)\\
&= \sum_n\frac{1}{n(n+1)}f_n(r)S_n(\theta,\phi).
\end{align*}

\subsection*{3. Curl of a Poloidal Field is a Toroidal Field}
A ``poloidal" field is written
\begin{align*}
\bm{B}_P = \nabla\wedge[\nabla\wedge(P\bm{x})] = -\nabla \wedge(\bm{x}\wedge\nabla P) = -\nabla^2(P\bm{x}) + \nabla[\nabla\cdot(P\bm{x})].
\end{align*}
We compute
\begin{align*}
\nabla\wedge \bm{B}_P &= \nabla\wedge\{ -\nabla^2(P\bm{x}) + \cancel{\nabla[\nabla\cdot(P\bm{x})]}\}\\
&= - \nabla\wedge[\nabla^2(P\bm{x})]
\end{align*}
Now, 
\begin{align*}
\nabla^2(P\bm{x})_i &= \partial_j\partial_j(Pr_i)\\
&= \partial_j[(\partial_jP)r_i + P\delta_{ij}]\\
&= (\partial_j^2P)r_i + (\partial_jP)\delta_{ij} + (\partial_jP)\delta_{ij} + 0\\
&= r_i\partial_j^2P + 2\partial_iP\Longrightarrow\\
\nabla^2(P\bm{x}) &= (\nabla^2P)\bm{x} + 2\nabla P.
\end{align*}
Thus,
\begin{align*}
\nabla\wedge \bm{B}_P &= - \nabla\wedge[(\nabla^2P)\bm{x} + 2\cancel{\nabla P}]\\
&= \nabla\wedge[(-\nabla^2P)\bm{x}],
\end{align*}
and so is a toroidal field with toroidal streamfunction $-\nabla^2P$. 

Note that the same holds true even if we define the streamfunctions in the alternate way
\begin{align*}
\bm{B} = \bm{B}_P + \bm{B}_T = \nabla\wedge[\nabla\wedge(P\e_r)] + \nabla\wedge(T\e_r),
\end{align*}
where $\e_r = \bm{x}/r$. That the curl of the toroidal field is a poloidal field is trivial, and the curl of the poloidal field is
\begin{align*}
\nabla\wedge\bm{B}_P &= -\nabla\wedge[\nabla^2(P\e_r)].
\end{align*}
Now, following the calculations from before,
\begin{align*}
\nabla^2(P\e_r) = \nabla^2\bigg{(}\frac{P}{r}\bm{x}\bigg{)} = \nabla^2\bigg{(}\frac{P}{r}\bigg{)}\bm{x} + 2\nabla\bigg{(}\frac{P}{r}\bigg{)},
\end{align*}
so
\begin{align*}
\nabla\wedge\bm{B}_P &= -\nabla\wedge\bigg{[}\nabla^2\bigg{(}\frac{P}{r}\bigg{)}\bm{x}\bigg{]}\\
&= \nabla\wedge\bigg{\{}\bigg{[}-r\nabla^2\bigg{(}\frac{P}{r}\bigg{)}\bigg{]}\e_r\bigg{\}},
\end{align*}
which is by (the new) definition, a toroidal field. 

\subsection*{4. $\bm{x}\cdot \bm{B}$ and $\bm{x}\cdot\nabla\wedge\bm{B}$ in Terms of Angular Momentum Operator}
We write
\begin{align*}
\bm{B} = \bm{B}_P + \bm{B}_T = \nabla\wedge[\nabla\wedge(P\bm{x})] + \nabla\wedge(T\bm{x})
\end{align*}
Since the Toroidal fields are orthogonal to $\bm{x}$, we compute
\begin{align*}
\bm{x}\cdot\bm{B} = \bm{x}\cdot \bm{B}_P &= \bm{x}\cdot\{\nabla\wedge[\nabla\wedge(P\bm{x})]\} \\
&= x_i\{\epsilon_{ijk}\partial_j[\epsilon_{klm}\partial_l(Pr_m)]\}\\
&= (\epsilon_{ijk}x_i\partial_j)\{\epsilon_{klm}[(\partial_lP)r_m + P\cancel{\delta_{lm}}]\}\\
&= -\underbrace{(\epsilon_{ijk} x_i\partial_j)}_{(\bm{x}\wedge\nabla)_k} \underbrace{(\epsilon_{kml}r_m\partial_l)}_{(\bm{x}\wedge\nabla)_k}P\\
&= -(\bm{x}\wedge\nabla)^2P
\end{align*}
Since we have already shown that $\nabla\wedge \bm{B}_P$ is a toroidal field (and thus orthogonal to $\bm{x}$), we have
\begin{align*}
\bm{x}\cdot(\nabla\wedge\bm{B}) = \bm{x}\cdot(\nabla\wedge \bm{B}_T) &= \bm{x}\cdot\{\nabla\wedge[\nabla\wedge(T\bm{x})]\}\\
&=  -(\bm{x}\wedge\nabla)^2T, 
\end{align*}
by an identical calculation to the one manipulating $\bm{x}\cdot\bm{B}$. 

\subsection*{5. Purely poloidal form for the term $T\bm{x} + \nabla U$ in vector potential}
We identify the vector potential $\bm{A}$ in terms of the streamfunctions in (2.37):
\begin{align*}
\bm{A} = \nabla\wedge(P\bm{x}) + T\bm{x} + \nabla U,
\end{align*}
which is (2.40). If we pick the Coulomb gauge, then 
\begin{align*}
0 = \nabla\cdot\bm{A} &= \nabla\cdot[\cancel{\nabla\wedge(P\bm{x})} + T\bm{x} + \nabla U]\\
&= \nabla\cdot(T\bm{x} + \nabla U). 
\end{align*}
Thus, using the formalism just laid out, we can write $T\bm{x} + \nabla U$ as the sum of a yet another set of toroidal and poloidal fields. But note that 
\begin{align*}
\bm{x}\cdot[\nabla\wedge(T\bm{x} + \cancel{\nabla U})] = \bm{x}\cdot[-\bm{x}\wedge\nabla T] = 0,
\end{align*} 
and so the toroidal part of $T\bm{x} + \nabla U$ is zero by virtue of 2.39(b). 

We thus have (for the Coulomb gauge)
\begin{align*}
\bm{A} &= \bm{A}_P + \bm{A}_T\\
\text{where}\five \bm{A}_P &= T\bm{x} + \nabla U = \nabla\wedge[\nabla\wedge(S\bm{x})]\\
\andd \bm{A}_T &= \nabla\wedge(P\bm{x})
\end{align*}. 

Finally, since the curl of a poloidal field is a toroidal field, the curl of a toroidal field is a poloidal field, and $\bm{B} = \nabla\wedge \bm{A}$, we have
\begin{align*}
\bm{B}_P &= \nabla\wedge\bm{A}_T,\\
\bm{B}_T &= \nabla\wedge\bm{A}_P.
\end{align*}

\subsection*{6. Axisymmetric Fields}
For axisymmetric fields, $\partial_\phi\equiv0$, so
\begin{align*}
\bm{B}_T &= \frac{1}{r^2\sin\theta}
 \begin{vmatrix} \e_r & r\e_\theta & r\sin\theta\e_\phi\\
\partial_r & \partial_\theta & 0\\
Tr & 0 & 0
\end{vmatrix}\\
&=  \frac{1}{r^2\sin\theta}\bigg{[}-\pderiv{}{\theta}(Tr)\bigg{]}(r\sin\theta\e_\phi)\\
&= -\pderiv{T}{\theta}\e_\phi.
\end{align*}

Similarly, $$\bm{A}_T = -\pderiv{P}{\theta}\e_\phi,$$ and thus
\begin{align*}
\bm{B}_P = \nabla\wedge\bm{A}_T &= \frac{1}{r^2\sin\theta}\begin{vmatrix} \e_r & r\e_\theta & r\sin\theta\e_\phi\\
\partial_r & \partial_\theta & 0\\
0 & 0 & -r\sin\theta\pderiv{P}{\theta}
\end{vmatrix}\\
&= \frac{1}{r^2\sin\theta}\bigg{[}-r\pderiv{}{\theta}\bigg{(}\sin\theta\pderiv{P}{\theta}\bigg{)}\e_\theta + \bigg{(}\pderiv{}{r}r\sin\theta\pderiv{P}{\theta}\bigg{)}(r\e_\theta) + 0\bigg{]}\\
&= -\frac{1}{r^2\sin\theta}\pderiv{}{\theta}\bigg{(}\underbrace{r\sin\theta\pderiv{P}{\theta}}_{\coloneqq\chi}\bigg{)}\e_r +
 \frac{1}{r\sin\theta}\pderiv{}{r}\bigg{(}\underbrace{r\sin\theta\pderiv{P}{\theta}}_{\coloneqq\chi}\bigg{)}\e_\theta,
\end{align*}
from whence we recover what Moffatt meant by (2.47). Note however, that Moffatt's (2.47) is completely wrong. It should read
\begin{align*}
\chi = -r\sin\theta A_\phi, \five A_\phi = -\pderiv{P}{\theta},
\end{align*}
so that
\begin{align*}
\chi = r\sin\theta\pderiv{P}{\theta}.
\end{align*}

Now, the $\bm{B}_P$-lines satisfy 
\begin{align*}
dr &= C (B_P)_r\\
r d\theta &= C (B_P)_\theta,
\end{align*}
where $C$ is some constant of proportionality, possibly a function of position along the streamline, but the same between both the above equations. Dividing the two equations above thus yields
\begin{align*}
\frac{dr}{rd\theta} &= \frac{(B_P)_r}{(B_P)_\theta}\Longrightarrow
0 = (B_P)_\theta dr - (B_P)_r rd\theta &= \frac{1}{r\sin\theta}\pderiv{\chi}{r}dr - \bigg{(}-\frac{1}{r^2\sin\theta}\pderiv{\chi}{\theta}\bigg{)}(rd\theta)\\
&= \frac{1}{r\sin\theta}\bigg{(}\pderiv{\chi}{r}dr + \pderiv{\chi}{\theta}d\theta\bigg{)} = \frac{1}{r\sin\theta}d\chi.
\end{align*}
The defining condition for $\bm{B}_P$-lines is thus
\begin{align*}
d\chi = 0\five \orr \chi(r,\theta) = \text{constant}.
\end{align*}

Consider the flux through a ring made by rotating the infinitesimal line element between $(r,\theta)$ and $(r+dr,\theta + d\theta)$. This line element as length $dl=\sqrt{dr^2+(rd\theta)^2}$, and the ring has area $a = 2\pi r\sin\theta dl$. The normal to this area has associated unit vector $\e_n = (-rd\theta, dr, 0)/dl$. Thus, the flux of $\bm{B}_P$ across the ring is
\begin{align*}
\bm{B}_P\cdot(a \e_n) &= \frac{-(B_P)_rrd\theta+ (B_P)_\theta dr}{dl}(2\pi r\sin\theta dl)\\
&= 2\pi r\sin\theta\bigg{[}\frac{1}{r^2\sin\theta}\pderiv{\chi}{\theta} r d\theta + \frac{1}{r\sin\theta}\pderiv{\chi}{r}dr\bigg{]}\\
&= 2\pi\bigg{[}\pderiv{\chi}{\theta} d\theta + \pderiv{\chi}{r}dr\bigg{]} = 2\pi d\chi.
\end{align*}

\subsection*{7. Two-dimensional analogues}
Moffatt seems to switch here to the second convection for streamfunctions (e.g., $P\bm{x}\rightarrow P\e_r$, etc.) He then zooms in so that the ``sphere is flat" and takes $(r,\theta,\phi)\rightarrow(z,x,y)$, leaving
\begin{align*}
\bm{B} = \bm{B}_P + \bm{B}_T = \nabla\wedge[\nabla\wedge(P\e_z)] + \nabla\wedge(T\e_z).
\end{align*}
The vector (field?) $\e_z$ is just a constant, which makes things easier, e.g., proving that the curl of the poloidal field is a toroidal field, so no need to do it again!

Now, clearly $\bm{B}_T$ is orthogonal to $\e_z$, and so
\begin{align*}
\e_z\cdot\bm{B} = \e_z\cdot\bm{B}_P &= \e_z\cdot\{-\nabla^2(P\e_z) + \nabla[\underbrace{\nabla\cdot(P\e_z)}_{\nabla P\cdot\e_z = \pderiv{P}{z}}]\}\\
&= \e_z\cdot\bigg{[} -(\nabla^2P)\e_z + \nabla\bigg{(}\pderiv{P}{z}\bigg{)}\bigg{]}\\
&=  -\bigg{(}\ppderiv{P}{x} + \ppderiv{P}{y} + \cancel{ \ppderiv{P}{z}\bigg{)}}+ \cancel{\ppderiv{P}{z}}\\
&= -\nabla_2^2P.
\end{align*}
Thus, the two-dimensional Laplacian operator $$\nabla_2^2\coloneqq \ppderiv{}{x} + \ppderiv{}{y}$$ is the analogue of the angular momentum operator. 

The fact that the curl of $\bm{B}_P$ is toroidal (and thus $\perp\e_z$) and an identical calculation to the preceding one immediately yields
\begin{align*}
\e_z\cdot \nabla\wedge\bm{B} = \e_z\cdot\nabla\wedge\bm{B}_T = -\nabla_2^2T.
\end{align*}

For ``axisymmetric fields" $P=P(x,z)$, $T=T(x,z)$, $\bm{B}_T = \nabla T\wedge \e_z = (\partial_xT\e_x + \partial_z\cancel{\e_z})\wedge\e_z = -\partial_xT\e_y$ (eq. (2.53)). Similar arguments to the ones before show that $\bm{A}$ has a toroidal part given purely by $\nabla\wedge(P\e_z)$, so that $\bm{A}_T = -\partial_xP\e_y$ and $\bm{B}_P = \nabla \wedge\bm{A}_T=\nabla\wedge(A\e_y)$, where $A=-\partial_xP$.

Thus, 
\begin{align*}
\bm{B}_P = \begin{vmatrix}\e_x & \e_y & \e_z\\
\partial_x & 0 & \partial_z\\
0 & A & 0\end{vmatrix}
= -(\partial_zA)\e_x + (\partial_xA)\e_z. 
\end{align*}
The $\bm{B}_P$-lines thus satisfy 
\begin{align*}
\frac{dx}{dy} = \frac{-\partial_zA}{\partial_xA}\Longrightarrow \partial_xAdx + \partial_zAdz = dA = 0.
\end{align*}
Finally, the flux per unit length through a strip parallel to $y$, i.e., one obtained by sliding the line segment joining $(x,z)$ and $(x+dx,z+dz)$ along $y$, is
\begin{align*}
\bm{B}_P\cdot(dz\e_x - dx\e_z) = (B_P)_xdz - (B_P)_zdx = \pderiv{A}{z}dz - \pderiv{A}{x}dx = -dA,
\end{align*}
making $A$ ($-A$?) the flux-function for $\bm{B}_P$. 
\end{document}